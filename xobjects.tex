We now discuss how to utilize XObjects in pdf\TeX.
One may wonder why we don't first discuss PDF objects in general, and the answer is simple: they're less
useful than form XObjects in pdf\TeX.
This is because objects are in general much more internal to the PDF, and generally don't display content
(unless they're XObjects or the content stream of a page object).

\subsection{Form XObjects}

pdf\TeX{} allows the creation of a form XObject in two steps:
\benum
    \item create the content of the XObject in a box;
    \item save the content to a form XObject, while also potentially altering its stream attributes and
    resources.
\eenum
\noindent The first step is done in the obvious way: by using the \TeX{} primitive \macro\setbox.
The second step is done using the pdf\TeX{} primitive \macro\pdfxform.
Its usage is
\getmacrousage{\pdfxform [<attr spec>] [<resources spec>] <box number>}
\noindent Where {\it attr spec} is \getmacrousageB{attr <general text>}, and {\it resources spec} is
\getmacrousageB{resources <general text>}.

\macro\pdfxform{} creates a form XObject whose attributes dictionary contain {\it attr spec}, and resources
dictionary contains {\it resources spec}, and whose stream is determined by the box numbered by {\it box
number}.
The PDF object number of the last form XObject created is placed in \macro\pdflastxform.
You can then paint a form XObject using the \macro\pdfrefxform{} primitive, whose usage is
\getmacrousage{\pdfrefxform <object number>}
\noindent This essentially does {\tt Do} to the form XObject specified by {\it object number}, but it also
updates the document's position as well.

\bnote
    \TeX{} does not deal in generation numbers.
    That is, all of the generation numbers generated by \TeX{} are 0.
    So if an object's object number is $N$, its object identifier will be $N\ 0$.
\eppbox

\bnote
    Recall that form XObjects have a required {\bf BBox} field in their stream dictionary.
    \TeX{} calculates this from the dimensions of the box supplied to \macro\pdfxform.
    So if anything protrudes from the box, it will be clipped from the form XObject when painted.
    This can be annoying, or useful.
\eppbox

Form XObjects are useful for creating recreatable PDF objects, similar to boxes.
Unlike boxes, the number of form XObjects is not limited by \TeX.
For example, we can create a special symbol point by doing:

\blisting
\bgroup
\setbox0=\hbox to10.3bp{\vrule width0bp height10.15bp depth.15bp%
\kern.15bp%
\pdfliteral{
    1 0 0 rg
    0 0 0 RG
    1 j 1 J
    .3 w
    0 0 m
    10 0 l 10 10 l 0 10 l 3 5 l h B
}\hfil}
\pdfxform0
\xdef\ribbon{\pdfrefxform\the\pdflastxform\relax}
\egroup
\elisting

\bgroup
\setbox0=\hbox to10.3bp{\vrule width0bp height10.15bp depth.15bp%
\kern.15bp%
\pdfliteral{
    1 0 0 rg
    0 0 0 RG
    1 j 1 J
    .3 w
    0 0 m
    10 0 l 10 10 l 0 10 l 3 5 l h B
}\hfil}
\pdfxform0
\xdef\ribbon{\pdfrefxform\the\pdflastxform\relax}
\egroup

\noindent This will create

\centerline{A ribbon: \ribbon}

The kerning and the extra dimensions are to take into account the extra space required by rounded joins.

We can use the fact that form XObjects clip their contents to our advantage: we can create infinitely
extensible operators using this (like those in my pdfMsym package).

\blisting
\bgroup
\setbox0=\hbox{$\displaystyle\sum$}
\setbox1=\hbox to.52\wd0{\copy0\hss}
\setbox2=\hbox to.08\wd0{\kern-.52\wd0\copy0\hss}
\setbox3=\hbox to.4\wd0{\kern-.6\wd0\copy0\hss}
\pdfxform1  \xdef\suumL{\the\pdflastxform}
\pdfxform2  \xdef\suumC{\the\pdflastxform}
\pdfxform3  \xdef\suumR{\the\pdflastxform}
\egroup

\def\suum{%
    \mathop{%
        \pdfrefxform\suumL%
        \xleaders\hbox{\pdfrefxform\suumC}\hfill%
        \pdfrefxform\suumR%
    }\limits%
}
\elisting

\noindent What we've done here is create three form XObjects, the first has the left slice of the sum
character, which does not extend; the second has the middle slice, which is what we use to extend the operator;
and the third has the right slice.
These are created by eyeballing the lengths of each slice, and clipping them using form XObjects.
We use the fact that the contents of \macro\mathop{} take the width of its limits, and we use leaders to
extend the middle slice.
This gives, for example:

\bgroup
\setbox0=\hbox{$\displaystyle\sum$}
\setbox1=\hbox to.52\wd0{\copy0\hss}
\setbox2=\hbox to.08\wd0{\kern-.52\wd0\copy0\hss}
\setbox3=\hbox to.4\wd0{\kern-.6\wd0\copy0\hss}
\pdfxform1  \xdef\suumL{\the\pdflastxform}
\pdfxform2  \xdef\suumC{\the\pdflastxform}
\pdfxform3  \xdef\suumR{\the\pdflastxform}
\egroup

\def\suum{%
    \mathop{%
        \pdfrefxform\suumL%
        \xleaders\hbox{\pdfrefxform\suumC}\hfill%
        \pdfrefxform\suumR%
    }\limits%
}

$$ \suum^{abcdefghijklm}_{nopqrstuvwxyz} $$

\bnote
    \TeX{} does not write the XObject into the PDF until it is referred to by \macro\pdfrefxform.
    Thus if you want to paint it yourself using a {\tt Do} operation, you need to somehow first write it
    into the PDF.
    This can be done by prepending \macro\pdfxform{} with \macro\immediate.
    This will write the XObject into memory immediately, instead of waiting for a reference.
\eppbox

\subsubsection{Shading}

We can also use form XObjects to paint shadings (tiling patterns must be {\bf Pattern} objects, which requires
us to use general PDF objects, not XObjects; so they will be discussed in the next section).
To do this, we must access the resources of the XObject.
So first we create a box with the PDF code we'd like:

\blisting
\bgroup
\setbox0=\hbox to100bp{\vrule width0bp height50bp depth50bp%
\pdfliteral{
    1 0 0 1 50 0 cm
    /Pattern cs
    /Sh scn
    \circle{50} f
}\hfil}
\elisting

Now we need to define the pattern {\bf Sh} in the {\bf Pattern} subdictionary of the form XObject's resources
dictionary:

\blisting
\pdfxform resources {
    /Pattern << /Sh <<
        /Type /Pattern
        /PatternType 2  % Shading pattern
        /Shading <<
            /ShadingType 2  % axial shading
            /ColorSpace /DeviceRGB
            /Domain [0 1]
            /Coords [0 0 100 0]    % the target coordinate space is not affected by changes to the CTM
            /Function <<
                /FunctionType 2     % exponential interpolation
                /Domain [0 1]
                /C0 [1 0 0]
                /C1 [0 1 1]
                /N 1    % linear
            >>
        >>
    >> >>
}0
\xdef\shadecircle{\pdfrefxform\the\pdflastxform\relax}
\egroup
\elisting

Now we can draw a shaded circle with the macro \macro\shadecircle, which results in:

\bgroup
\setbox0=\hbox to100bp{\vrule width0bp height50bp depth50bp%
\pdfliteral{
    1 0 0 1 50 0 cm
    /Pattern cs
    /Sh scn
    \circle{50} f
}\hfil}
\pdfxform resources {
    /Pattern << /Sh <<
        /Type /Pattern
        /PatternType 2  % Shading pattern
        /Shading <<
            /ShadingType 2  % axial shading
            /ColorSpace /DeviceRGB
            /Domain [0 1]
            /Coords [0 0 100 0]    % the target coordinate space is not affected by changes to the CTM
            /Function <<
                /FunctionType 2     % exponential interpolation
                /Domain [0 1]
                /C0 [1 0 0]
                /C1 [0 1 1]
                /N 1    % linear
            >>
        >>
    >> >>
}0
\xdef\shadecircle{\pdfrefxform\the\pdflastxform\relax}
\egroup
\centerline{\shadecircle}

\subsubsection{Transparency}

Transparency is also governed by XObjects (transparency group objects), so we will discuss that here.
Suppose we want to draw four overlapping circles, each with an opacity of $.5$.
First we draw our circles:

\blisting
\bgroup
\setbox0=\hbox to100bp{\vrule width0bp height100bp depth0bp%
\pdfliteral{
    /Gs gs
    .75 g
    q
    1 0 0 1 40 60 cm
    \circle{30} f
    Q
    q
    1 0 0 1 60 60 cm
    \circle{30} f
    Q
    q
    1 0 0 1 40 40 cm
    \circle{30} f
    Q
    q
    1 0 0 1 60 40 cm
    \circle{30} f
    Q
}\hfil}
\elisting

\noindent Now we add {\bf Gs} to the resources.
We will make the fill opacity $.5$ and the blend mode {\bf Multiply}.

\blisting
\pdfxform resources{
    /ExtGState << /Gs <<
        /ca 1
        /BM /Multiply
    >> >>
}0
\xdef\fourcircs{\pdfrefxform\the\pdflastxform\relax}
\egroup
\elisting

\bgroup
\setbox0=\hbox to100bp{\vrule width0bp height100bp depth0bp%
\pdfliteral{
    /Gs gs
    .75 g
    q
    1 0 0 1 40 60 cm
    \circle{30} f
    Q
    q
    1 0 0 1 60 60 cm
    \circle{30} f
    Q
    q
    1 0 0 1 40 40 cm
    \circle{30} f
    Q
    q
    1 0 0 1 60 40 cm
    \circle{30} f
    Q
}\hfil}
\pdfxform resources{
    /ExtGState << /Gs <<
        /ca 1
        /BM /Multiply
    >> >>
}0
\xdef\fourcircs{\pdfrefxform\the\pdflastxform\relax}
\egroup

\centerline{\fourcircs}

We can also make a more complicated drawing.
Say we start with a gradient background, and draw four circles on it with an opacity of $.5$.
But the four circles will form an isolated transparency group, and within the transparency group they are
drawn with full opacity.

\blisting
\bgroup
\setbox0=\hbox to100bp{\vrule width0bp height100bp depth0bp%
\pdfliteral{
    .75 g
    q
    1 0 0 1 40 60 cm
    \circle{30} f
    Q
    q
    1 0 0 1 60 60 cm
    \circle{30} f
    Q
    q
    1 0 0 1 40 40 cm
    \circle{30} f
    Q
    q
    1 0 0 1 60 40 cm
    \circle{30} f
    Q
}\hfil}
\immediate\pdfxform attr{
    /Group <<
        /S /Transparency
        /CS /DeviceGray
        /I true
    >>
}0
\edef\fourcircsform{\the\pdflastxform}

\setbox0=\hbox to100bp{\vrule width0bp height100bp depth0bp%
\pdfliteral{
    /Sh sh
    /Gs gs
    /4circs Do
}\hfil}
\pdfxform resources {
    /Shading << /Sh <<
        /ShadingType 2
        /ColorSpace /DeviceRGB
        /Coords [0 0 100 0]
        /Domain [0 1]
        /Function <<
            /FunctionType 2
            /Domain [0 1]
            /C0 [1 0 0]
            /C1 [0 1 1]
            /N 1
        >>
    >> >>
    /ExtGState << /Gs <<
        /ca 1
        /BM /Multiply
    >> >>
    /XObject <<
        /4circs \fourcircsform\space0 R
    >>
}0
\xdef\fourcircs{\pdfrefxform\the\pdflastxform\relax}
\egroup
\elisting

\bgroup
\setbox0=\hbox to100bp{\vrule width0bp height100bp depth0bp%
\pdfliteral{
    .75 g
    q
    1 0 0 1 40 60 cm
    \circle{30} f
    Q
    q
    1 0 0 1 60 60 cm
    \circle{30} f
    Q
    q
    1 0 0 1 40 40 cm
    \circle{30} f
    Q
    q
    1 0 0 1 60 40 cm
    \circle{30} f
    Q
}\hfil}
\immediate\pdfxform attr{
    /Group <<
        /S /Transparency
        /CS /DeviceGray
        /I true
    >>
}0
\edef\fourcircsform{\the\pdflastxform}

\setbox0=\hbox to100bp{\vrule width0bp height100bp depth0bp%
\pdfliteral{
    /Sh sh
    /Gs gs
    /4circs Do
}\hfil}
\pdfxform resources {
    /Shading << /Sh <<
        /ShadingType 2
        /ColorSpace /DeviceRGB
        /Coords [0 0 100 0]
        /Domain [0 1]
        /Function <<
            /FunctionType 2
            /Domain [0 1]
            /C0 [1 0 0]
            /C1 [0 1 1]
            /N 1
        >>
    >> >>
    /ExtGState << /Gs <<
        /ca 1
        /BM /Multiply
    >> >>
    /XObject <<
        /4circs \fourcircsform\space0 R
    >>
}0
\xdef\fourcircs{\pdfrefxform\the\pdflastxform\relax}
\egroup

This creates the following:

\centerline{\fourcircs}

\subsection{Image XObjects}

Say we want to include an image in a \TeX{} file.
We could use an inline image for images that we know the samples of, or an image XObject by defining it
explicitly (using \macro\pdfobj).
But say we have an image file we'd like to include.
We can use pdf\TeX{} primitives for this.

The main character of the show is \macro\pdfximage, which reads an image file into memory.
It supports the following formats: JPEG ({\tt.jpg} or {\tt.jpeg}), JBIG2 ({\tt.jbig2} or {\tt.jb2}), PNG
({\tt.png}), or PDF ({\tt.pdf}).
Its usage is
\getmacrousage{\pdfximage [<rule spec>] [<attr spec>] [page <number>] <general text>}
\noindent {\it general text} is the name of the file to import.

The dimensions of the image can be controlled via {\it rule spec}.
If some dimensions (but not all) are given, the others will be scaled to yield the same ratio between
width and ${\rm height}+{\rm depth}$ as the image's ratio between height and depth.
If no {\it rule spec} is given, the image will have its natural dimensions.

If {\it general text} refers to a PDF file, \getmacrousageB{page <number>} may be specified, which imports
that specific page from the PDF file.

{\it attr spec} defines additional attributes which will be added to the image XObject.

An image imported by \macro\pdfximage{} is not written to the PDF unless \macro\pdfximage{} is preceeded by
\macro\immediate, or \macro\pdfrefximage{} is used on the object number of the image XObject (which is
returned by \macro\pdflastximage).

