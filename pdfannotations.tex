We can add annotation objects using the \macro\pdfannot{} primitive.
Its usage is
\getmacrousage{\pdfannot reserveobjnum | [useobjnum <number>] [<rule spec>] <general text>}
\noindent Where {\it rule spec} is \getmacrousageB{(width | height | depth) <dimen> [<rule spec>]}.
This should illustrate the reason for using \macro\pdfannot{} over \macro\pdfobj{} and with {\tt /Type /Annot}:
you can give the annotation's activation area.

Everything here is pretty self-explanatory.
We can create a text annotation with different states like so:

\blisting
\bgroup
\setbox1=\hbox{Rollover}
\setbox0=\hbox to\wd1{\hfil Normal\hfil}
\setbox2=\hbox to\wd1{\hfil Down\hfil}
\immediate\pdfxform0 \edef\normal{\the\pdflastxform}
\immediate\pdfxform1 \edef\rollover{\the\pdflastxform}
\immediate\pdfxform2 \edef\down{\the\pdflastxform}
\pdfannot width\wd0 height\ht0 depth\dp0 {
    /Subtype /Text
    /Contents (This is a text annotation with different states)
    /AP <<
        /N \normal\space0 R
        /R \rollover\space0 R
        /D \down\space0 R
    >>
    /Border [3 3 1]
    /C [0 1 1]
}
\egroup
\elisting

This creates:

\bgroup
\setbox3=\hbox{Rollover}
\setbox1=\copy3
\setbox0=\hbox to\wd1{\hfil Normal\hfil}
\setbox2=\hbox to\wd1{\hfil Down\hfil}
\immediate\pdfxform0 \edef\normal{\the\pdflastxform}
\immediate\pdfxform1 \edef\rollover{\the\pdflastxform}
\immediate\pdfxform2 \edef\down{\the\pdflastxform}
\centerline{\pdfannot width\wd3 height\ht3 depth\dp3 {
    /Subtype /Text
    /Contents (This is a text annotation with different states)
    /AP <<
        /N \normal\space0 R
        /R \rollover\space0 R
        /D \down\space0 R
    >>
}}
\egroup

\bnote
    Many PDF viewers don't support multiple-state annotations.
    Try viewing this PDF on Adobe Acrobat.
\eppbox

\subsection{Links and destinations}

Consider the following issue: suppose we have an explicit destination we want to jump to.
But the link itself spans multiple lines.
\TeX{} will break the link over multiple lines, but how should the link border look?
Well pdf\TeX{} actually provides a primitive for creating link annotations so that they can break across
lines or even pages.
It does this by inserting a new annotation for each line.

This is done via the primitives \inlinecode|\pdfstartlink...\pdfendlink|, whose usage is
\getmacrousage{\pdfstartlink [<rule spec>] [attr <general text>] <action spec>}
\noindent where {\it action spec} is
\getmacrousage{user <general text> | goto <goto action spec>}
\noindent where {\it goto action spec} is
\getmacrousage{name <general text> | page <number> <general text> | named <general text> <general text>}

{\it attr spec} defines additional attributes for the annotation (e.g. {\bf Border}, {\bf C}).
{\it action spec} defines the action to be done when the annotation's activation region is clicked.
If {\it action spec} is {\tt user}, {\it general text} is added directly to the annotation's dictionary.
We will discuss {\tt goto} actions later.
For example, we may have a URL hyperlink:

\blisting
\pdfstartlink
    attr {
        /Border [3 3 1]
        /C [1 0 0]
    }
    user {
        /Subtype /Link
        /A <<
            /S /URI
            /URI (https://tug.org/)
        >>
    }
\TeX{} User Group%
\pdfendlink
\elisting

This creates

\centerline{%
\pdfstartlink
    attr {
        /Border [3 3 1]
        /C [1 0 0]
    }
    user {
        /Subtype /Link
        /A <<
            /S /URI
            /URI (https://tug.org/)
        >>
    }
\TeX{} User Group%
\pdfendlink}

If the {\it goto action spec} is \getmacrousageB{page <number>}, when clicked, the document goes to the page
specified by {\it number}, and the magnification factor is given by {\it general text}.

\subsubsection{Destinations}

You can create an indirect destination using the primitive \macro\pdfdest.
Its usage is
\getmacrousage{\pdfdest name <general text> (xyz [zoom <number>] | fitr <rule spec> | fith | fitv | fit)}
\noindent This creates a destination at the current point whose name is {\it general text}.
The type of destination is specified by the rest of the input (e.g. {\tt xyz} for {\bf/XYZ}).
This destination can be refered to by its name as a string; its name is put in the name tree defined by
the {\bf Dests} subdictionary of the document catalog's {\bf Name} dictionary.

If {\tt xyz} is specified, an optional {\tt zoom} argument can be given.
A zoom of $1000$ gives the normal page view.

You can then then create a link to this destination like so:

\blisting
\pdfdest name {link name} xyz zoom 2000
...
\pdfstartlink
    attr {
        /Border [3 3 1]
        /C [1 0 0]
    }
    goto name {link name}%
This is a link%
\pdfendlink
\elisting

