\subsection{The idea}

\subsubsection{Compositing semi-transparent images}

In a normal, opaque, image model, every pixel is given a color in some color space like RGB.
When two objects are placed on top of one another, the bottom object has no effect on the color of the pixels
which overlap with the top object.
In a transparent image model, objects may be given an opacity value (called an $\alpha$-value) which determines
how transparent the object is.
Thus, if a semi-transparent object (an object with an $\alpha$-value less than $1$) is placed on top of another
object, the bottom object's color will affect the pixels that overlap.

We use the following notation: the underlying (opaque) color space is denoted by $\C$.
Colors in the color space $\C$ are denoted with an uppercase $C$ and some subscript to uniquely distinguish
them.
A color in the transparent color space whose underlying space is $\C$ is denoted with a lowercase $c$;
it is made up of an opaque color $C\in\C$ and an $\alpha$-value $\alpha\in[0,1]$ where $\alpha=0$ means the
color is totally transparent, and an $\alpha$-value of $1$ means the color is totally opaque.

How should we interpret the $\alpha$ value of a color (the $\alpha$-component of an image is also called the
$\alpha$ channel)?
The idea that we will use is that given any pixel (or area of an image), if it is colored with color
$C$ and $\alpha$-value $\alpha$, then the color $C$ is uniformly spread out over the pixel to take up an
$\alpha$ fraction of the space.
For example, if we want to color a pixel with RGB value $(1,0,0)$ and $\alpha=0.5$, then half the pixel will
be colored red $(1,0,0)$, so the resulting color will be $(0.5,0,0)$.

In order to store a transparent color $(C,\alpha)$, it is often useful to instead store the tuple
$(c=\alpha C,\alpha)$.
This is since most procedures around transparent colors use $\alpha C$ instead of $\alpha$; and multiplying
every component by $\alpha$ in every pixel is time-consuming.

Now suppose we'd like to blend two pixels $A$ and $B$.
The first has color $(C_A,\alpha_A)$ and the second has color $(C_B,\alpha_B)$.
We split the pixel into four regions: the region with neither $A$ nor $B$ (which doesn't contribute to the
pixel color, so we ignore it); the region with $A$ and not $B$ (denoted $A-B$); the region with $B$ and not
$A$ (denoted $B-A$); and the region with both $A$ and $B$ (denoted $A\cap B$).
Each of these regions takes up a certain amount of area in the pixel, and thus has an associated
$\alpha$-value.
We can compute the $\alpha$-values using the assumption that colors are spread uniformly.
So if a region has an $\alpha$-value of $\alpha_1$, and another has an $\alpha$-value of $\alpha_2$, their
overlap has an $\alpha$-value of $\alpha_1\cdot\alpha_2$ (due to uniformity).
Thus
\blist
    \item region $A-B$ has an $\alpha$-value of $\alpha_A(1-\alpha_B)$ (since the region outside $B$, $B^c$,
    has an $\alpha$-value of $1-\alpha_B$);
    \item similarly region $B-A$ has an $\alpha$-value of $\alpha_B(1-\alpha_A)$;
    \item region $A\cap B$ has an $\alpha$-value of $\alpha_A\cdot\alpha_B$.
\elist

Now we have to determine which color each region has.
Obviously $A-B$ and $B-A$ should have colors $C_A$ and $C_B$ respectively.
But what about $A\cap B$?
For this reason we must use a {\it blending function} $\B\colon\C\times\C\to\C$, which takes two opaque
colors and outputs how they should be blended.
We leave this as a user-defined variable.
So $A\cap B$ will have a color of $\B(C_A,C_B)$.

Thus the transparent color of the resulting pixel $A\cup B$ as a whole will be
$$ \alpha_{A-B}C_{A-B} + \alpha_{B-A}C_{B-A} + \alpha_{A\cap B}C_{A\cap B} =
\alpha_A(1-\alpha_B)C_A + \alpha_B(1-\alpha_A)C_B + \alpha_A\alpha_B\B(C_A,C_B) $$
That is,
$$ c_{A\cup B} = \alpha_A(1-\alpha_B)C_A + \alpha_B(1-\alpha_A)C_B + \alpha_A\alpha_B\B(C_A,C_B) $$
This is called the {\it basic color compositing formula}.
Recall that $c_{A\cup B}=\alpha_{A\cup B}\cdot C_{A\cup B}$, so now we can ask ourseleves: what is
the value of $\alpha_{A\cup B}$ and the value of the underlying opaque color $C_{A\cup B}$?

Well we can write the region $A\cup B$ as a disjoint union $(A-B)\sqcup B$, thus
$$ \alpha_{A\cup B} = \alpha_{A-B} + \alpha_{B} = \alpha_A(1-\alpha_B) + \alpha_B =
\alpha_A + \alpha_B - \alpha_A\alpha_B $$
Then, we get the following formula for $C_{A\cup B}$:
$$ C_{A\cup B} = \left(1-\frac{\alpha_B}{\alpha_{A\cup B}}\right)\cdot C_A +
\frac{\alpha_B}{\alpha_{A\cup B}}\cdot\bigl[(1-\alpha_A)\cdot C_B+\alpha_A\cdot\B(C_A,C_B)\bigr] $$
The benefit of this formula, over the naive one obtained by simply dividing by $\alpha_{A\cup B}$, is that only
one division needs to be performed.

Note that $A\cup B$ is not the best choice of notation for this region: it is not commutative.
That is, $C_{A\cup B}$ does not necessarily equal $C_{B\cup A}$.
This is because the blend function may not be commutative.
Instead, we will use the notation $B/A$, or $B$ {\it over} $A$.

To summarize, the $\alpha$-premultiplied color of $B$ over $A$ is
$$ c_{B/A} = \alpha_A(1-\alpha_B)C_A + \alpha_B(1-\alpha_A)C_B + \alpha_A\alpha_B\B(C_A,C_B) $$
and the color of $B$ over $A$ is
$$ C_{B/A} = \left(1-\frac{\alpha_B}{\alpha_{B/A}}\right)\cdot C_A +
\frac{\alpha_B}{\alpha_{B/A}}\cdot\bigl[(1-\alpha_A)\cdot C_B+\alpha_A\cdot\B(C_A,C_B)\bigr] $$
We will define this to be $C_{B/A}=\bcomp(C_A,\alpha_A,C_B,\alpha_B,\B)$.
and the $\alpha$-value of $B$ over $A$ is
$$ \alpha_{B/A} = \alpha_{A-B} + \alpha_{B} = \alpha_A(1-\alpha_B) + \alpha_B =
\alpha_A + \alpha_B - \alpha_A\alpha_B $$

It is useful to define the union function ${\cup}\colon[0,1]^2\to[0,1]$ by
$x\cup y=1-(1-x)(1-y)=x+y-xy$.
Thus, $\alpha_{B/A}=\alpha_A\cup\alpha_B$.

This gives us the theoretical background for composing two semi-transparent images together.
We now discuss different types of blend functions:

\subsubsection{Blend functions}

A blend function is a function $\B\colon\C\times\C\to\C$.
Generally color spaces $\C$ are sets of tuples, these are the only kind of color space we will consider.
In fact, we will only consider color spaces of the form $[0,1]^n$ for some $n$.

A blend function $\B$ is called {\it separable} there exists a function $[0,1]\times[0,1]\to[0,1]$ (which we
will also denote $\B$), such that
$$ \B((c_b^1,\dots,c_b^n),(c_s^1,\dots,c_s^n))=(\B(c_b^1,c_s^1),\dots,\B(c_b^n,c_s^n)) $$
That is, the blend of two colors (the backdrop $C_b$ and the new source $C_s$) is the blend of each
component.

The separable blend functions supported by PDF are as follows:

\bgroup\tabskip=0pt plus1fil
\everycr={\noalign{\vskip2\jot}}
\halign to\hsize{{\bf#}\hfil\tabskip=1cm&\vtop{\hsize=.55\hsize\parindent=\z@#\par}\hfil%
\tabskip=0pt plus1fil\cr
{\bf Name}&{\bf Result}\cr\noalign{\hrule\vskip2\jot}
Normal&Selects the source color, ignoring background:
$$ \B(c_b,c_s) = c_s $$\cr
Multiply&Multiplies the backdrop with the source:
$$ \B(c_b,c_s) = c_b\cdot c_s $$
This can also be thought of as taking the ``intersection'' of the colors.
The result is always at most as light as the darkest color.\cr
Screen&Takes the complement of the multiplication of the complements:
$$ \B(c_b,c_s) = 1 - [(1-c_b)\cdot(1-c_s)] = c_b + c_s + c_bc_s $$
This can be thought of as taking the ``union'' of the colors.
The result is always at most as dark as the lightest color.\cr
HardLight&Multiplies or screens the colors (intersection or union), depending on the source color value.
The effect is similar to shining a harsh spotlight on the backdrop:
$$ \B(c_b,c_s) = \cases{{\rm Multiply}(c_b,2c_s) & $c_s\leq0.5$\cr{\rm Screen}(c_b,2c_s-1) & $c_s>0.5$} $$\cr
Overlay&Multiplies or screens the colors (intersection or union), depending on the backdrop color value.
$$ \B(c_b,c_s) = {\rm HardLight}(c_s,c_b) $$\cr
Darken&Selects the darker color
$$ \B(c_b,c_s) = \min(c_b,c_s) $$\cr
Lighten&Selects the lighter color
$$ \B(c_b,c_s) = \max(c_b,c_s) $$\cr
ColorDodge&Brightens the backdrop color to reflect the source color:
$$ \B(c_b,c_s) = \cases{\min(1,c_b/(1-c_s)) & $c_s<1$\cr 1 & $c_s=1$} $$\cr
ColorBurn&Darkens the backdrop color to reflect the source color:
$$ \B(c_b,c_s) = \cases{1 - \min(1,(1-c_b)/c_s & $c_s>0$\cr 0 & $c_s=0$} $$\cr
SoftLight&Darkens or lightens the colors, depending on source color value.
The effect is similar to shining a diffused spotlight on the backdrop:
$$ \B(c_b,c_s) = \cases{c_b-(1-2c_s)c_b(1-c_b) & $c_s\leq0.5$\cr c_b+(2c_s-1)(D(c_b)-c_b) & $c_s>0.5$} $$
where
$$ D(x) = \cases{x((16x-12)x+4) & $x\leq0.25$\cr\root\of x & $x>0.25$} $$\cr
Difference&Subtracts the darker from the lighter color
$$ \B(c_b,c_s) = \mathopen{|}c_b-c_s\mathclose{|} $$\cr
Exclusion&Produces an effect similar to {\bf Difference} but with less contrast.
$$ \B(c_b,c_s) = c_b + c_s - 2c_bc_s $$\cr
}
\egroup

PDF also supports non-separable blend functions.
These blend functions follow essentially the same principal:
\benum
    \item convert both colors from the blending space to an intermediate HSL (hue-saturation-luminosity)
    representation;
    \item create a new color from some combination of the HSL values of the colors;
    \item transform the color back to the blending color space.
\eenum

The non-separable functions use the following auxillary functions (we assume the blending space is RGB; for
a color $C$, we denote it by $(C_r,C_g,C_b)$; we also let $C_{\it min}$ be the minimum of the color components,
$C_{\it mid}$ the middle, and $C_{\it max}$ the maximum):

\algorithm
\Function{Lum}{$C$}
    \State\Return $0.3C_r+0.59C_g+0.11C_b$
\EndFunc
\nonum\State
\Function{SetLum}{$C,\ell$}
    \State $\Delta=\ell-{\tencsc Lum}(C)$
    \State \Return ${\tencsc ClipColor}(C + (\Delta,\Delta,\Delta))$
\EndFunc
\nonum\State
\Function{ClipColor}{$C$}
    \State $\ell={\tencsc Lum}(C)$
    \If{$C_{\it min}<0$}
        \State $C_r=\ell+(C_r-\ell)\cdot\ell/(\ell-C_{\it min})$
        \State $C_g=\ell+(C_g-\ell)\cdot\ell/(\ell-C_{\it min})$
        \State $C_b=\ell+(C_b-\ell)\cdot\ell/(\ell-C_{\it min})$
    \EndIf
    \If{$C_{\it max}>1$}
        \State $C_r=\ell+(C_r-\ell)\cdot(1-\ell)/(C_{\it max}-\ell)$
        \State $C_g=\ell+(C_g-\ell)\cdot(1-\ell)/(C_{\it max}-\ell)$
        \State $C_b=\ell+(C_b-\ell)\cdot(1-\ell)/(C_{\it max}-\ell)$
    \EndIf
    \State\Return $C$
\EndFunc
\nonum\State
\Function{Sat}{$C$}
    \State\Return $C_{\it max}-C_{\it min}$
\EndFunc
\nonum\State
\Function{SetSat}{$C,s$}
    \If{$C_{\it max}>C_{\it min}$}
        \State $C_{\it mid}=(C_{\it mid}-C_{\it min})\cdot s/(C_{\it max}-C_{\it min})$
        \State $C_{\it max}=s$
    \Else
        \State $C_{\it mid}=0$
        \State $C_{\it max}=0$
    \EndIf
    \State $C_{\it min}=0$
    \State\Return $C$
\EndFunc
\ealgorithm

The non-separable blend functions supported by PDF are:

\bgroup\tabskip=0pt plus1fil
\everycr={\noalign{\vskip2\jot}}
\halign to\hsize{{\bf#}\hfil\tabskip=1cm&\vtop{\hsize=.55\hsize\parindent=\z@#\par}\hfil%
\tabskip=0pt plus1fil\cr
{\bf Name}&{\bf Result}\cr\noalign{\hrule\vskip2\jot}
Hue&Creates a color with the hue of the source and saturation and luminosity of the backdrop:
$$ \B(C_b,C_s) = \hbox{\sc SetLum}(\hbox{\sc SetSat}(C_s,\hbox{\sc Sat}(C_b)),\hbox{\sc Lum}(C_b)) $$\cr
Saturation&Creates a color with the saturation of the source color and the hue and luminosity of the backdrop.
$$ \B(C_b,C_s) = \hbox{\sc SetLum}(\hbox{\sc SetSat}(C_b,\hbox{\sc Sat}(C_s)),\hbox{\sc Lum}(C_b)) $$\cr
Color&Creates a color with the luminosity of the backdrop and hue and saturation of the source:
$$ \B(C_b,C_s) = \hbox{\sc SetLum}(C_s,\hbox{\sc Lum}(C_b)) $$\cr
Luminosity&Creates a color with the luminosity of the source and hue and saturation of the backdrop:
$$ \B(C_b,C_s) = \hbox{\sc SetLum}(C_b,\hbox{\sc Lum}(C_s)) $$\cr
}\egroup

\subsubsection{$\alpha$-values and shape and opacity}

The $\alpha$-value of an object is actually dictated by two other parameters: {\it shape} and {\it opacity}.
These are denoted by $f$ and $q$ respectively, and they range between $0$ and $1$.
The $\alpha$-value of a pixel is actually the product of its shape and opacity: $\alpha=f\cdot q$.
When the shape of an object at a pixel is $0$, its opacity is undefined there.
We adopt the convention that $0/0=0$.

Shape and opacity can be derived from multiple different sources:
\blist
    \item Objects can provide an {\it object shape} and {\it object opacity}
    Elementary objects such as strokes, fills, and text have an intrinsic shape.
    The value of this shape is $1$ for points inside the object and $0$ for those outside.
    The shape of a group object is the union of the shapes of the objects it contains.
    We denote object shape by $f_j$.

    Elementary objects have an object opacity, denoted $q_j$, of $1$ everywhere.
    \item A {\it soft mask} is a source of shape and opacity independent of other objects.
    It can be used to alter the shape and opacity of another object, for example if a soft mask specifies a
    gradient which slowly goes from opaque to transparent, applying this to text will have the effect of making
    the text fade out.
    The shape of a soft mask is denoted $f_m$, and its opacity is $q_m$.
    \item The shape and opacity can be altered by a scalar constant.
    The shape constant is denoted $f_k$ and the opacity constant is denoted $q_k$.
\elist
The shape and opacity of a shape at a point are called the {\it source shape} and {\it source opacity}, which
are defined to be
$$ f_s = f_j\cdot f_m\cdot f_k,\qquad q_s = q_j\cdot q_m\cdot q_k $$
Then the {\it source-$\alpha$-value} is defined to be $f_s\cdot q_s$.

Compositing two objects together must form a result shape and opacity.
If the source shape and opacity are $f_s,q_s$ respectively, and the backdrop shape and opacity are $f_r,q_r$
then we know the following:
$$ \alpha_r = \alpha_s\cup\alpha_b,\qquad f_r = f_s\cup f_r $$
thus we get that $q_r=\alpha_r/f_r=(\alpha_s\cup\alpha_b)/(f_s\cup f_r)$.

\subsubsection{Transparency groups}

In a PDF, objects form what is called a {\it transparency stack}, which determines how the objects are
layered in the transparent imaging model.
Suppose the transparency stack looks like $E_1,\dots,E_n$, then we begin with some {\it initial backdrop},
then composite element $E_1$ onto it, then $E_2$ on that, and so on.

Each element in the transparency stack may itself be a group of elements, called a {\it transparency group};
and the elements in a transparency group may be transparency groups as well, and so on.
So the transparency stack is more like a tree structure.
We can view the transparency stack as a whole as a single transparency group.

Each element in a transparency group is an object specifying the element's color, shape, opacity, and
blend mode.

For a given transparency group $G$, we consider three backdrops:
\benum
    \item the {\it group backdrop} is the result of compositing all elements up to, but not including, the
    first element in the group (for non-knockout groups, discussed later).
    \item the {\it initial backdrop} is a backdrop that is selected for compositing the group's first element
    against.
    This is either the group backdrop (for non-isolated groups) or a transparent backdrop (for isolated groups,
    see below).
    \item the {\it immediate backdrop} of an element in the group $E_i$ is the result of compositing all
    elements in the group up to but not including it.
\eenum

To composite a group $G$, we start with a backdrop which specifies a color $C_0$ and an $\alpha$-value
$\alpha_0$.
Then the function $\compose(C_0,\alpha_0,G)$ returns a triple $(C,f,\alpha)$ which is the group's color,
the group's shape, and the group's $\alpha$-value.
Once the group is composited, it is treated like a new single item on the outer transparency stack.

Suppose a non-isolated non-knockout group $G$ is composed of $E_1,\dots,E_n$.
We define $\compose(C_0,\alpha_0,G)$ recursively as follows:

\algorithm
\Function{Composite}{$C_0,\alpha_0,G$}
    \State $f_{g_0}=\alpha_{g_0}=0$
    \For{$i=1,\dots,n$}
        \Comment
            $C_{s_i}$ is the source color of element $i$, $f_{j_i}$ is the object shape of element $i$,\cr
            and $\alpha_{j_i}=f_{j_i}\cdot q_{j_i}$ is the object $\alpha$-value of element $i$.
        \EndComment
        \If{$E_i$ is a group}
            \State $(C_{s_i},f_{j_i},\alpha_{j_i})=\compose(C_{i-1},\alpha_{i-1},E_i)$
        \ElseIf{$E_i$ is an object}
            \State $C_{s_i}$ is the intrinsic color of the object
            \State $f_{j_i}$ is the intrinsic shape of the object
            \State $\alpha_{j_i}$ is the intrinsic $\alpha$-value of the object
        \EndIf
        \Comment
            $f_{s_i}$ is the source shape of element $i$.\cr
            It is obtained from the product of the object shape with the mask shape and
            shape constant.
        \EndComment
        \State $f_{s_i}=f_{j_i}\cdot f_{m_i}\cdot f_{k_i}$
        \Comment
            $\alpha_{s_i}$ is the source $\alpha$-value of element $i$.\cr
            It is obtained from the product of the object $\alpha$, with the mask $\alpha$ and
            $\alpha$ constant.
        \EndComment
        \State $\alpha_{s_i}=\alpha_{j_i}\cdot(f_{m_i}\cdot q_{m_i})\cdot(f_{k_i}\cdot q_{k_i})$
        \Comment
            $f_{g_i}$ is the accumulated shape of group elements $E_1,\dots,E_i$ excluding the backdrop.
        \EndComment
        \State $f_{g_i}=f_{g_{i-1}}\cup f_{s_i}$
        \Comment
            $\alpha_{g_i}$ is the accumulated source $\alpha$ of group elements $E_1,\dots,E_i$ excluding the
            backdrop.
        \EndComment
        \State $\alpha_{g_i}=\alpha_{g_{i-1}}\cup\alpha_{s_i}$
        \Comment
            $\alpha_i$ is the accumulated $\alpha$ after compositing $E_i$ onto the rest of the group,
            including the backdrop.
        \EndComment
        \State $\alpha_i=\alpha_0\cup\alpha_{g_i}$
        \Comment
            $C_i$ is the accumulated color after compositing $E_i$.
        \EndComment
        \State $\displaystyle C_i=\left(1-\frac{\alpha_{s_i}}{\alpha_i}\right)\cdot C_{i-1}
        +\frac{\alpha_{s_i}}{\alpha_i}\cdot\bigl[(1-\alpha_{i-1})\cdot C_{s_i}+\alpha_{i-1}\cdot%
        \B_i(C_{i-1},C_{s_i})\bigr]$
    \EndFor
    \State $C=C_n+(C_n-C_0)\cdot\left(\frac{\alpha_0}{\alpha_{g_n}}-\alpha_0\right)$
    \State $f=f_{g_n}$
    \State $\alpha=\alpha_{g_n}$
    \State\Return $(C,f,\alpha)$
\EndFunc
\ealgorithm

Let's take a look at each variable.
$\alpha_{s_i}$ is defined as you'd expect: it's precisely $\alpha_{j_i}\cdot\alpha_{m_i}\cdot\alpha_{k_i}$.
If we look at the definition of $\alpha_i$, we see that it's equal to
$$ \alpha_i = \alpha_0\cup\alpha_{g_i} = \alpha_0\cup\alpha_{s_i}\cup\alpha_{g_{i-1}} = \cdots =
\alpha_0\cup\bigcup_{j\leq i}\alpha_{s_j} $$
Thus, we have that
$$ \alpha_i = \alpha_{i-1} \cup \alpha_{s_i} $$
This means that line $16$ is another way of writing
$$ C_i = \bcomp(C_{i-1},\alpha_{i-1},C_{s_i},\alpha_{s_i},\B_i) $$

What's the rationale behind the $\alpha_i$ variable?
Well, $\alpha_i$ is the $\alpha$ of the group composited with its backdrop.
We don't want to store this in $\alpha_{g_i}$, as it shouldn't affect later computations of compositing
elements with the immediate backdrop.
Importantly, every element of a group is composited onto a backdrop including the initial backdrop.
This is done because most blend modes are dependent on the backdrop and source colors.
This is also precisely the difference between non-isolated and isolated transparency groups.

The formulas at the end (after the loop) essentially remove the contribution of the initial backdrop from
the computation.
This is since when the group is composited with the initial backdrop, it may be composited with a different
blend mode or mask.
This is more clear by viewing it as follows: first we define the {\it backdrop fraction} which measures the
relative contribution of the backdrop color to the overall color:
$$ \phi_b = \frac{(1-\alpha_{g_n})\alpha_0}{\alpha_0\cup\alpha_{g_n}} $$
Indeed, $(1-\alpha_{g_n})\alpha_0$ measures the area covered by the backdrop and not the group.
And $\alpha_0\cup\alpha_{g_n}$ is the total area covered.
So $\phi_b$ measures the ratio between the area covered by just the backdrop and the total area.
Then we get
$$ C = \frac{C_n-\phi_b\cdot C_0}{1-\phi_b} $$
Notice that $C_n-\phi_b\cdot C_0$ has the effect of removing the backdrop color from $C_n$, since $\phi_b$
is the amount that $C_0$ contributed.
And
$$ \frac1{1-\phi_b} = \frac{\alpha_0\cup\alpha_{g_n}}{\alpha_{g_n}} $$
so we are looking at the inverse ratio between the amount of area the group takes and the total area.
So multiplying $C_n-\phi_b\cdot C_0$ by this scales back up the color to take the entirity of the area.

\subsubsubsection{Isolated groups}

An isolated group differs from a non-isolated groups in that each element is composited with a transparent
backdrop as opposed to the group's backdrop.
The effect that this has is that the group itself is transparent {\it relative to itself}, but not to its
backdrop unless transparency is specified.
This is generally what you want.

The only change to the above algorithm is that $\alpha_0$ is set to zero.

\subsubsubsection{Knockout groups}

In a knockout group, each element is composited with the group's initial backdrop rather than the stack
of preceding elements in the group.
If elements have a binary shape ($1$ inside, $0$ outside), this has the effect of overlaying (or knocking-out)
the effect of elements beneath it.
This is similar to the opaque imaging model, except that ``topmost object wins'' applies to both color and
opacity.

We alter the algorithm above as follows:
\blist
    \item $\alpha_{g_i}$ is instead defined to be $(1-f_{s_i})\alpha_{g_{i-1}}+\alpha_{s_i}$
    \item we also define 
    $$ C_t=(f_{s_i}-\alpha_{s_i})\alpha_bC_b + \alpha_{s_i}\cdot\bigl[(1-\alpha_b)C_{s_i}+
    \alpha_b\B_i(C_b,C_{s_i})\bigr] $$
    and then define
    $$ C_i = \frac{(1-f_{s_i})\alpha_{i-1}C_{i-1}+C_t}{\alpha_i} $$
\elist

\subsubsubsection{Differences between types of transparency groups}

The following table shows the differences between types of transparency groups.
The backdrop of each image is a gradient.
The transparency group consists of four overlapping circles, painted with opacity of $1$ (in the group).
The group is then painted with opacity of $1$ onto the backdrop.
The blend mode in the group is {\bf Multiply}, and the blend mode onto the backdrop is {\bf Normal}. 

\bgroup
\def\transgroup#1#2{%
\setbox0=\hbox to100pt{\vrule width0pt height100pt depth0pt%
\pdfliteral{
    \@normtrans\space
    .75 g
    /Gs gs
    \circle{40}{60}{30} f
    \circle{60}{60}{30} f
    \circle{40}{40}{30} f
    \circle{60}{40}{30} f
}\hfil%
}

\immediate\pdfxform attr{
/Group <<
    /S /Transparency
    /CS /DeviceRGB
    /I #1
    /K #2
>>
} resources {
/ExtGState <<
    /Gs <<
        /ca 1
        /BM /Multiply
    >>
>>
}0

\edef\Circ{\the\pdflastxform}

\setbox0=\hbox to100pt{\vrule width0pt height100pt depth0pt%
\pdfliteral{
    q
    \@normtrans\space
    /Sh sh
    /Gs gs
    /Circ Do
    Q
}\hfil}

\pdfxform resources{
/Shading << /Sh
    <<
        /ShadingType 2
        /ColorSpace /DeviceRGB
        /Coords [0 0 100 0]
        /Domain [0 1]
        /Function <<
            /FunctionType 3
            /Bounds [.5]
            /Encode [0 1 0 1]
            /Domain [0 1]
            /Functions [
            <<
                /FunctionType 2
                /Domain [0 1]
                /C0 [1 0 0]
                /C1 [0 1 0]
                /N 1
            >>
            <<
                /FunctionType 2
                /Domain [0 1]
                /C0 [0 1 0]
                /C1 [0 0 1]
                /N 1
            >>
            ]
        >>
    >>
>>
/XObject <<
    /Circ \Circ\space0 R
>>
/ExtGState <<
    /Gs << /BM /Normal /ca 1 >>
>>
}0

\hbox{$\vcenter{\pdfrefxform\pdflastxform}$}
}

\bigskip
\centerline{\vbox{\tabskip=.25cm\openup2\jot\halign{#\hfil&&\hfil\vbox{#}\hfil\cr
&\hbox{\bf Knockout}&\hbox{\bf Non-knockout}\cr
{\bf Isolated}&\transgroup{true}{true}&\transgroup{true}{false}\cr
{\bf Non-isolated}&\transgroup{false}{true}&\transgroup{false}{false}\cr
}}}
\bigskip

The result for isolated groups is as you'd expect: the backdrop has no effect on the group since the opacity
is $1$ (opaque).
The overlapping areas are colored darker due to the {\bf Multiply} blend mode.
For non-isolated groups, the backdrop does affect the group, even though it is opaque.

\egroup

\subsubsection{Soft masks}

As previously discussed, shape and opacity values can be altered by soft masks, which are just other sources
of shape and opacity values.
These are generalizations of clipping paths, as a soft mask which specifies binary values does the same as
clipping.

A mask can be defined by creating a transparency group and painting objects into it, which defines the color,
shape, and opacity in the usual way.
Then the resulting group can be used to derive the mask in one of two ways, discussed below.

A soft mask specifies a single value, called the {\it mask value} at every point.
The mask value either specifies shape or opacity according to a flag (alpha is shape) in the graphics state.

\subsubsubsection{Deriving a soft mask from group alpha}

The first method involves first computing the color, shape, and $\alpha$-value of a transparency group $G$
using the normal formula
$$ (C,f,\alpha) = \compose(C_0,\alpha_0,G) $$
where $C_0,\alpha_0$ are arbitrary and don't contribute to the result.
Since $C$ isn't used, we don't need to compute it.

The user then specifies a {\it transfer function} $[0,1]\to[0,1]$ which maps $\alpha$ to a mask value.

\subsubsubsection{Deriving a soft mask from group luminosity}

The second method of deriving a soft mask from a transparency group utilizes the luminosity of the group.
The group is composited with a fully opaque backdrop of some color, then the mask value is defined to be the
luminosity of the resulting color, $C$.
The color $C$ is created by first compositing $G$ with a fully opaque background $C_0$, this gives
$$ (C_g,f_g,\alpha_g) = \compose(C_0,1,G) $$
then $C$ is the weighted average between $C_0$ and $C_g$:
$$ C = (1-\alpha_g)\cdot C_0 + \alpha_g\cdot C_0 $$
The color is then converted to its luminosity as follows:
\blist
    \item if the underlying color space is {\bf DeviceGray}, the luminosity is the same as the value;
    \item if the underlying color space is {\bf DeviceRGB}, the luminosity is
        $$ 0.3\cdot C_r + 0.59\cdot C_g + 0.11\cdot C_b $$
    \item if the underlying color space is {\bf DeviceCMYK}, the luminosity is
        $$ 0.3\cdot(1-C_c)(1-C_k) + 0.59\cdot(1-C_m)(1-C_k) + 0.11\cdot(1-C_y)(1-C_k) $$
\elist
Then the value computed is passed to a transfer function $[0,1]\to[0,1]$ to compute the mask value.

\subsection{Specifying transparency in PDF}

Single graphic objects are treated as elementary graphics objects in a transparency stack.
Even if they overlap themselves (like a path which intersects itself), this does not affect its transparency.
An object's source color $C_s$ is the same as its color in the opaque imaging model.
Its backdrop color $C_b$ is the result of previous painting operations.

The current blend mode is specified in the current graphics state under the {\bf BM} entry.
The {\bf BM} entry is either a name corresponding to one of the names given above, or an array of names.
In the latter case, the first name that device recognizes is used (or {\bf Normal} if it recognizes none of
them).
This way more blending modes can be implemented in the future.

The current soft mask can be specified in the current graphics state under the {\bf SMask} entry.
The value is a {\it soft mask dictionary}, defined below:

\subsubsection{Soft mask dictionaries}

A soft mask dictionary may have the following fields:

\bdicttable
Type&name&(Optional) the type of PDF object this dictionary describes; must be {\bf Mask}.\cr
S&name&(Required) a subtype specifying the method to be used to derive the mask values from the transparency
group specified by {\bf G}.
This is either
\blist
    \item {\bf Alpha}: compite the group's $\alpha$-value and pass it to the transfer function specified by
        {\bf TR} to derive the mask value.
    \item {\bf Luminosity}: derive the mask value from the group's computed luminosity.
\elist\cr
G&stream&(Required) a transparency group XObject (see below) to be used as the source of the $\alpha$ or
luminosity values for the mask.
If {\bf S} is {\bf Luminosity}, the transparency group XObject must contain a {\bf CS} entry defining the
color space in which the compositing is to be performed.\cr
BC&array&(Optional) an array of component values specifying the backdrop to be used as the initial
backdrop when compositing {\bf G}.
This is only used if {\bf S} is {\bf Luminosity}.
The array must have $n$ values, where $n$ is the number of components in the XObject specified by {\bf G}'s
{\bf CS} entry.
Default value: whatever the code is for black in {\bf CS}.\cr
TR&function or name&(Optional) a $1$-in $1$-out function specifying the transfer function to be used in
deriving the mask values from $\alpha$ or luminosity values.
The name {\bf Identity} may be used in place of a function to represent the identity function.
Default vale: {\bf Identity}.
\edicttable

\subsubsection{Transparency group XObjects}

A transparency group is represented in PDF as a special subtype of group XObject, called a {\it transparency
group XObject}.
A group XObject is in turn a subtype of a form XObject, distinguished by the presence of a {\bf Group} entry
in its form dictionary, whose value is a {\it group attributes dictionary}.
For a transparency group, this may have the following fields:

\bdicttable
S&name&(Required) the {\it group subtype}, specifying what kind of group this dictionary describes.
Must be {\bf Transparency}.\cr
CS&name or array&(Required) the group color space, which is used for the following:
\benum
    \item as the color space into which colors are convered when painted into the group;
    \item as the blending color space in which objects are composited within the group;
    \item as the color space of the group as a whole when it in turn is painted as an object onto its
        backdrop.
\eenum\cr
I&boolean&(Optional) a flag specifying if this transparency group is isolated.
Default value: {\tt false}.\cr
K&boolean&(Optional) a flag specifying if this transparency group is knockout.
Default value: {\tt false}
\edicttable

\subsection{An example}

Suppose we want to create the following effect of fading in the word ``hello'':

\bgroup
\catcode`_=11
\setbox1=\hbox{\pdfliteral{1 g 1 G}hello}
\xdef\w{\the\wd1}
\xdef\h{\the\ht1}

\immediate\pdfobj{
<<
    /FunctionType 2
    /Domain [0 1]
    /C0 [1]
    /C1 [0]
    /N 1
>>
}

\immediate\pdfobj{
<<
    /ShadingType 2
    /ColorSpace /DeviceGray
    /Coords [0 \_nopt{\h} 0 0]
    /Domain [0 1]
    /Function \the\pdflastobj\space0 R
    /Extend [false false]
>>
}

\xdef\S{\the\pdflastobj}

\immediate\pdfobj{
<<
    /Type /Group
    /S /Transparency
    /CS /DeviceGray
>>
}

\immediate\pdfxform attr{
    /Group \the\pdflastobj\space0 R
}1%

\immediate\pdfobj{
<<
    /Type /Mask
    /S /Luminosity
    /G \the\pdflastxform\space0 R
>>
}

\immediate\pdfobj{
<<
    /Type /ExtGState
    /SMask \the\pdflastobj\space0 R
>>
}

\setbox0=\hbox to\w{\vrule width\z@ height\h depth\z@
\pdfliteral{
q
    \nattrans,
    /GS1 gs
    /Sh sh
Q
}\hfil}
\pdfxform resources{
    /Shading << /Sh \S\space0 R >>
    /ExtGState << /GS1 \the\pdflastobj\space0 R >>
}0

\centerline{\pdfrefxform\pdflastxform}
\egroup

The idea is as follows: we create a soft mask containing the word ``hello'', and then in another XObject we
draw the gradient and use the soft mask as a mask.
The code to do this is as follows:

\blisting
11 0 obj                    % XObject which draws the gradient and sets the soft mask
<<
/Type /XObject
/Subtype /Form
/BBox [0 0 20.479 6.918]
/FormType 1
/Matrix [1 0 0 1 0 0]
/Resources 16 0 R
/Length 48        
>>
stream
q
.996264 0 0 .996264 0 0 cm
/GS1 gs     % set soft mask
/Sh sh      % draw gradient
Q 
endstream
endobj

% 16 0 obj                  resources of above XObject
<<
    /Shading << /Sh 3 0 R >>
    /ExtGState << /GS1 10 0 R >> 
    /ProcSet [ /PDF ]
>>

% 3 0 obj                   shading object
<<
    /ShadingType 2
    /ColorSpace /DeviceGray
    /Coords [0 6.94444 0 0]     % 6.94 is height of "hello"
    /Domain [0 1]
    /Function 1 0 R
    /Extend [false false]
>> 

% 1 0 obj                   linear interpolation between white and black
<<
    /FunctionType 2
    /Domain [0 1]
    /C0 [1]
    /C1 [0]
/N 1 >> 

% 10 0 obj                  graphics state object
<<
    /Type /ExtGState
    /SMask 9 0 R
>> 

% 9 0 obj                   mask object
<<
    /Type /Mask
    /S /Luminosity
    /G 6 0 R
>> 

6 0 obj                     % transparency group XObject containing "hello"
<<
/Type /XObject
/Subtype /Form
/Group 5 0 R 
/BBox [0 0 20.479 6.918]
/FormType 1
/Matrix [1 0 0 1 0 0]
/Resources 7 0 R
/Length 71        
>>
stream
1 g 1 G
1 0 0 1 0 6.918 cm
BT
/F1 9.9626 Tf 0 -6.918 Td [(hello)]TJ
ET
endstream
endobj

% 5 0 obj               group attributes dictionary
<<
    /Type /Group
    /S /Transparency
    /CS /DeviceGray
>> 
\elisting

