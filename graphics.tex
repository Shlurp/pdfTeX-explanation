PDF provides support for drawing and including graphics.
In this section we will cover the code for doing so, and later on we will discuss how to interact with this
code through pdf\TeX.

PDF inherits the postfix syntax from PostScript.
This inheritance is entirely syntactical, as PDF does not support the concept of an argument stack or other
features PostScript provides.

PDF defines the following graphics objects for use within content streams:
\blist
    \item {\it path objects} are arbitrary shapes made up of straight lines, rectangles, and cubic B\'ezier
    curves.
    A path object ends with painting operators which indicate whether the path is opened or closed, stroked,
    filled, etc.
    \item {\it text objects} consist of one or more character strings that identify sequences of glyphs to be
    painted.
    It can also be stroked, filled, or used as a clipping boundary.
    \item {\it external objects} (XObjects) are objects defined outside of the content stream, but can be
    referenced from within the content stream through use of the stream's {\bf Resources}.
    There are different kinds of XObjects:
    \blist
        \item {\it image XObjects} define a rectangular array of color samples to be painted;
        \item {\it form XObjects} define an entire content stream to be treated as a single graphics object;
        \item {\it reference XObjects} are a type of form XObject used to import content from one PDF into
        another;
        \item {\it group XObjects} are a type of form XObject used to group graphical elements together
        (e.g. for use in the transparency model, which uses {\it transparency group XObjects}).
    \elist
    \item {\it inline image objects} use a special syntax to express data for a small image directly within
    the content stream.
    \item {\it shading objects} describes a geometric shape whose color is an arbitrary function of position
    within the shape.
\elist

PDF 1.3 and early use an opaque imaging model, meaning that every object is painted in its entirety and at
every point, only the object at the top has an effect on the color painted.
PDF 1.4 and later use a transparent imaging model, meaning that objects may be specified to have a certain
amount of transparency, so that objects underneath it may also affect the color painted.
By default objects are painted as opaque.



