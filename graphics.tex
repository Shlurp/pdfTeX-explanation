PDF provides support for drawing and including graphics.
In this section we will cover the code for doing so, and later on we will discuss how to interact with this
code through pdf\TeX.

PDF inherits the postfix syntax from PostScript.
This inheritance is entirely syntactical, as PDF does not support the concept of an argument stack or other
features PostScript provides.

PDF defines the following graphics objects for use within content streams:
\blist
    \item {\it path objects} are arbitrary shapes made up of straight lines, rectangles, and cubic B\'ezier
    curves.
    A path object ends with painting operators which indicate whether the path is opened or closed, stroked,
    filled, etc.
    \item {\it text objects} consist of one or more character strings that identify sequences of glyphs to be
    painted.
    It can also be stroked, filled, or used as a clipping boundary.
    \item {\it external objects} (XObjects) are objects defined outside of the content stream, but can be
    referenced from within the content stream through use of the stream's {\bf Resources}.
    There are different kinds of XObjects:
    \blist
        \item {\it image XObjects} define a rectangular array of color samples to be painted;
        \item {\it form XObjects} define an entire content stream to be treated as a single graphics object;
        \item {\it reference XObjects} are a type of form XObject used to import content from one PDF into
        another;
        \item {\it group XObjects} are a type of form XObject used to group graphical elements together
        (e.g. for use in the transparency model, which uses {\it transparency group XObjects}).
    \elist
    \item {\it inline image objects} use a special syntax to express data for a small image directly within
    the content stream.
    \item {\it shading objects} describes a geometric shape whose color is an arbitrary function of position
    within the shape.
\elist

PDF 1.3 and early use an opaque imaging model, meaning that every object is painted in its entirety and at
every point, only the object at the top has an effect on the color painted.
PDF 1.4 and later use a transparent imaging model, meaning that objects may be specified to have a certain
amount of transparency, so that objects underneath it may also affect the color painted.
By default objects are painted as opaque.

\subsection{Coordinate systems}

Positions in the document are determined in terms of coordinates on a plane.
A coordinate space is determined by the following properties relative to the current page:
\blist
    \item the location of the origin;
    \item the orientation of the $x$ and $y$ axes;
    \item the lengths of the units along each axis.
\elist
There are several coordinate spaces defined by the PDF, which will be described in this section.
Transformations between coordinate spaces are done via affine transformations (transformations of the form
$x\mapsto Ax+b$, where $A$ is a matrix and $b$ a vector).

The coordinate space which is native to a specific device is called its {\it device space}.

\subsubsection{User space}

To avoid the issues arising from using device-dependent coordinate spaces, PDF defines a device-independent
coordinate system that remains the same relative to the current page no matter the medium in which it is
displayed or printed.
This is called the {\it user space} coordinate system.

The {\bf CropBox} entry in a page dictionary specifies the rectangle of user space corresponding to the visible
area on the output medium.
The length of a unit along both the $x$ and $y$ axes is set by {\bf UserUnit} (in PDF-1.6).
If the entry is not supplied (or supported), the default value is $1/72$th of an inch.
{\bf CropBox} defines the rectangular region for which the page is to be displayed in the infinite plane that
is the user space.

The transformation from user to device space is defined by the CTM (current transformation matrix).
This is stored in the PDF graphics state (to be dicussed below).
A PDF content stream can modify user space by using the {\bf cm} operator (coordinate transformation
operator).

\subsubsection{Other coordinate spaces}

In addition to device and user space, PDF utilizes a variety of other coordinate systems:
\blist
    \item The coordinates of text are defined in {\it text space}.
    The translation from text space to user space is defined by a {\it text matrix} as well as several
    text-related parameters in the graphics state (see below).
    \item All sampled images are defined in {\it image space}.
    The transformation from image to user space is predefined and cannot be changed.
    All images are one-unit by one-unit in user space.
    To paint them, the CTM must be temporarily changed.
    \item A form XObject as a self-contained content stream is defined in a {\it form space}.
    When painted in another content stream, its space is transformed into user space using the {\it form
    matrix} which is defined in the form XObject.
    \item A pattern (which is content invoked repeatedly to tile an area) is defined in a space called
    {\it pattern space}.
    The transformation from pattern space to user space is defined in a {\it pattern matrix} contained in the
    pattern.
\elist

\subsubsection{Transformation matrices}

A transformation, as discussed previously, is an affine transformation of the form $x\mapsto Ax+b$ where
$A$ is a $2\times2$ matrix and $b$ a vector of size $2$.
Suppose we want to transform $(x,y)$ to $(x',y')$ via such an affine transformation, then
$$ \pmatrix{x'\cr y'} = \pmatrix{a&c\cr b&d}\pmatrix{x\cr y}+\pmatrix{e\cr f} $$
or, we can write this as
$$ \pmatrix{x'\cr y'\cr1} = \pmatrix{a&c&e\cr b&d&f\cr0&0&1}\pmatrix{x\cr y\cr1} $$
The reason we add the final line of the matrix is to make it square.
So if the current CTM is $M$ and we'd like to transform it by the affine transformation described by matrix
$A$, then we must change CTM to be $M\cdot A$.
Thus if we are given an input coordinate $X$, then we first apply $A$ and then $M$.

\bnote
In the PDF reference, they take the convention of multiplying by row vectors on the left.
So the affine transformation is represented by the tranpose of the matrix provided here.
Nevertheless, the results are the same.
\eppbox

The affine transformation represented by
$$ \pmatrix{a&c&e\cr b&d&f\cr0&0&1} $$
is represented in PDF code by $[a\ b\ c\ d\ e\ f]$.

\subsection{Graphics State}

When rendering a PDF, an internal state must be held by the application which determines the current state to
be used when rendering graphics.
The graphics state is initialized at the beginning of each page according to the table below.

\bthreetable{}{}{}
{\bf Parameter}&{\bf Type}&{\bf Value}\cr\noalign{\hrule\vskip2\jot}
CTM&array&The current transformation matrix, which maps user space to device space.
The CTM can be modified by use of the {\bf cm} operator.\cr
clipping path&(internal)&The current {\it clipping path}, which defines the boundary against which all
output is to be cropped.
The initial value is the boundary of the entire imageable portion of the page.\cr
color space&name or array&The current {\it color space} which determines how color values are to be
interpreted (e.g. RGB).
There are two separate color spaces: one for stroking and one for non-stroking operations.
The initial value is {\bf DeviceGray}.\cr
color&(various)&The current color to be used when painting.
The type and interpretation of the color depends on the current color space.
There are two color parameters: one for stroking and one for non-stroking operations.
The initial value is black.\cr
text state&(various)&A set of nine graphics state parameters that affect the painting of text (see below).\cr
line width&number&The thickness, in user space units, of paths to be stroked.
Initial value: $1.0$.\cr
line cap&integer&A code specifying the shape of endpoints for any stroked open paths (see below).
Initial value: $0$ (square caps).\cr
line join&integer&A code specifying the shape of joints between connected segments of a stroked path.
Initial value: $0$ (mitered joints).\cr
miter limit&number&The maximum length of mitered line joins for stroked paths (the length of the spikes when
lines join at sharp angles).
Initial value: $10.0$.\cr
dash pattern&array and number&A description of the dash pattern to be used when paths are stroked (see below).
Initial value: a solid line.\cr
blend mode&name or array&The current {\it blend mode} to be used in the transparent imaging model (see below).
This parameter is reset to its initial value at the beginning of execution of a transparency group XObject.
Initial value: {\bf Normal}.\cr
soft mask&dictionary or name&A {\it soft-mask} dictionary (see below), specifying the mask shape or opacity
values to be used in the transparent imaging model, or {\bf None}.
This parameter is reset to its initial value at the beginning of execution of a transparency group XObject.
Initial value: {\bf None}.\cr
alpha constant&number&The constant shape or constant opacity value to be used in the transparent imaging
model.
There are two alpha constant parameters: one for stroking and another for non-stroking operations.
This parameter is reset to its initial value at the beginning of execution of a transparency group XObject.
Initial value: $1.0$.\cr
alpha source&boolean&A flag specifying whether the current soft mask and alpha constant parameters are to be
interepreted as shape values ({\bf true}) or opacity values ({\bf false}).
Initial value: {\bf false}.
\ethreetable

Some parameters are set with specific operators, while others are set by including a particular entry in a
graphics state parameter dictionary.
Some can be set either way.
For example, the line width can be set using the {\bf w} operator, or with the {\bf LW} entry.

The {\it graphics state stack} allows for local changes to the graphics state, so you can change it without
affecting things outside the current scope.
The stack is a stack (LIFO --- last in first out) data structure.
The {\bf q} operator pushes a copy of the graphics state onto the stack, while {\bf Q} pops from the stack.
Occurrences of {\bf q} and {\bf Q} must be balanced within a content stream.

\subsubsection{Line caps}

\def\showlinecap#1{%
    \hbox to50pt{\pdfliteral{%
        q
        \nattrans,
        #1 J
        10 w
        0 g 0 G
        0 0 m 50 0 l S
        1 J
        1 w
        1 g 1 G
        0 0 m 50 0 l S
        Q
    }\hfil}%
}

We demonstrate in the table below the three styles of line caps.
These are the ends of open subpaths (and dashes) when they are stroked.

\bthreetable{}{}{}
{\bf Style}&{\bf Appearance}&{\bf Description}\cr\noalign{\hrule\vskip2\jot}
0&\showlinecap0&{\it Butt cap}: the stroke is squared off at the endpoint of the path.
There is no projection beyond the end of the path.\cr
1&\showlinecap1&{\it Round cap}: the stroke is rounded off with semicircular ends on both sides of the stroke.
The diameter of the capp is equal to the line width.\cr
2&\showlinecap2&{\it Projecting square cap}: the stroke continues beyond the endpoint of the path for a
distance equal to half the line width and is squared off.
\ethreetable

\subsubsection{Line Joins}

\def\showlinejoin#1{%
    \hbox to50pt{\vrule width\z@ height0pt depth35pt%
    \pdfliteral{%
        q
        \nattrans,
        1 0 0 1 0 -35 cm
        1 J
        #1 j
        10 w
        0 g 0 G
        0 0 m 25 30 l 50 0 l S
        1 j
        1 w
        1 g 1 G
        0 0 m 25 30 l 50 0 l S
        Q
    }\hfil}%
}

\bthreetable{}{}{}
{\bf Style}&{\bf Appearance}&{\bf Description}\cr\noalign{\hrule\vskip2\jot}
0&\showlinejoin0&{\it Miter join}: the outer edges of the strokes for the segments are extended until they meet
at an angle.
If the angles meet at too sharp an angle (as defined by the miter limit parameter, though we won't go into
this), a bevel join is used instead.\cr
1&\showlinejoin1&{\it Round join}: an arc of a circle with diameter equal to the line width is drawn around
the point where the two segments meet.\cr
2&\showlinejoin2&{\it Bevel join}: ythe two segments are finished with butt caps.
\ethreetable

\subsubsection{Line dash pattern}

The {\it line dash pattern} controls the pattern of dashes and gaps used by stroked paths.
It is specified by a {\it dash array} and a {\it dash phase}.
The dash array's elements are numbers that specify the lengths of alternating dashes and gaps (they must all
be nonnegative and not all zero).
The dash phase specifies the distance into the dash pattern at which to start the dash.
When dashing begins, the elements of the dash array are cyclicly summed up, and when the sum equals the dash
phase, the stroking of the phase begins.

\def\showdash#1{%
    \hbox to100pt{\pdfliteral{%
        q
        \nattrans,
        0 j 0 J
        0 g 0 G
        .3 w
        [1] 0 d
        0 -5 100 10 re S
        4 0 0 1 0 0 cm
        #1 d
        10 w
        0 0 m 25 0 l S
        Q
    }\hfil}%
}

\bthreetable{\tt}{}{}
{\bf Dash}&{\bf Appearance}&{\bf Description}\cr\noalign{\hrule\vskip2\jot}
[] 0&\showdash{[] 0}&No dash\cr
[3] 0&\showdash{[3] 0}&3 units on, 3 units off,\dots\cr
[2] 1&\showdash{[2] 1}&1 on, 2 off, 2 on,\dots\cr
[2 1] 0&\showdash{[2 1] 0}&2 on, 1 off, 2 on,\dots\cr
[3 5] 6&\showdash{[3 5] 6}&2 off, 3 on, 5 off,\dots
\ethreetable

\subsubsection{Graphics state operators}

We list the operators which can be used to alter the graphics state.
Like all PDF operators, they are postfix.

\bthreetable{\it}{\bf}{}
{\bf Operands}&{\bf Operator}&{\bf Description}\cr\noalign{\hrule\vskip2\jot}
---&q&Push the current graphics state onto the state stack.\cr
---&Q&Pop from the state stack into the current graphics state.\cr
a b c d e f&cm&Modify the current transformation matrix (CTM) by multiplying with the matrix
$$ \pmatrix{a&b&0\cr c&d&0\cr e&f&1} . $$\cr
lineWidth&w&Sets the line width.\cr
lineCap&J&Sets the line cap style according to the provided code.\cr
lineJoin&j&Sets the line join style according to the provided code.\cr
miterLimit&M&Sets the miter limit.\cr
dashArray dashPhase&d&Set the line dash pattern in the graphics state.\cr
dictName&gs&Set the specified parameters in the grapgics state.
{\it dictName} is the name of a graphics state dictionary in the {\bf ExtGState} subdictionary of the current
{\bf Resources} dictionary.
\ethreetable

As written, you can use the {\bf gs} to alter the graphics state.
This is by providing the name of a graphics state parameter dictionary which is provided in the
{\bf ExtGState} field of the current {\bf Resources} dictionary.
Graphics state parameter dictionaries may have the following fields (all fields are optional):

\bdicttable
Type&name&The type of PDF object that this dictionary describes; must be {\bf ExtGState}.\cr
LW&number&The line width.\cr
LC&integer&The line cap code.\cr
LJ&integer&The line join code.\cr
ML&number&The miter limit.\cr
D&array&The line dash pattern, expressed as an array of the form $[{\it dashArray}\ {\it dashPhase}]$ where
{\it dashArray} is itself an array.\cr
Font&array&An array of the form $[{\it font}\ {\it size}]$, where {\it font} is an indirect reference to a
font dictionary and {\it size} is a number expressed in text space units for the font size.
These can also be altered by the {\bf Tf} operator (see below).\cr
BM&name or array&The current blend mode to be used in the transparent imaging model (see below).\cr
SMask&dictionary or name&The current soft mask, which specifies the mask shape or mask opacity values to be
used in the transparent imaging model.\cr
CA&number&The current stroking alpha constant, specifying the shape or constant opacity value to be used for
stroking operations in the transparent imaging model.\cr
ca&number&Same as {\bf CA} but for non-stroking operations.\cr
AIS&boolean&The alpha source flag (``alpha is shape''), which specifies whether the current soft mask and alpha
constant are to be interpreted as shape values ({\bf true}) or opacity values ({\bf false}).\cr
TK&boolean&The text knockout flag, which determines the behavior of overlapping glyphs within a text object in
the transparent imaging model.
\edicttable

For example, we may have the single-page PDF:

\blisting
10 0 obj % the page object
<<
    /Type /Page
    /Parent 5 0 R
    /Resources 20 0 R
    /Contents 40 0 R
>>
endobj

20 0 obj % resource dictionary for the page
<<
    /ProcSet [/PDF /Text]
    /Font << /F1 25 0 R >>
    /ExtGState <<
        /GS1 30 0 R
        /GS2 35 0 R
    >>
>>
endobj

30 0 obj % first graphics state
<<
    /Type /ExtGState
    /LW .3
    /D [[3] 0]
>>
endobj

35 0 obj % second graphics state
<<
    /Type /ExtGState
    /LC 1
    /LJ 1
    /LW 1
>>
endobj

40 0 obj % contents of the page
<<
    /Length ...
>>
stream
q
/GS1 gs
0 0 m 10 0 l S
/GS2 gs
0 -5 m 10 -5 l S
Q
endstream
endobj
\elisting

We have two graphics states, {\bf GS1} and {\bf GS2}, which both alter the graphics state to change how lines
are drawn.
We now discuss the code for drawing paths.

\subsection{Path construction and painting}

Paths define the boundaries of areas within the PDF.
They can be stroked to draw the boundaries of shapes, or filled, or added to the current clipping path in order
to crop material on the page.

A path is made up of one or more disconnected subpaths, each of which is a sequence of connected segments,
which are lines or curves.
Two segments are connected only if they are defined consecutively: if you define a line from $(0,0)$ to
$(10,0)$, then a line to $(10,10)$, the two lines are connected.
But if between those two lines you draw another line, they are not connected.
This is not entirely true: a subpath may be closed, that is the first and last segments may be connected by the
use of a special operator.

There are three types of path operators:
\benum
    \item {\it path construction operators} which define the geometry of the path;
    \item {\it path painting operators} which define how the path should be painted and/or stroked;
    \item {\it clipping path operators} which say if a path should be added to the current clipping area.
\eenum

\subsubsection{Path construction operators}

\bthreetable{\it}{\bf}{}
{\bf Operands}&{\bf Operator}&{\bf Description}\cr\noalign{\hrule\vskip2\jot}
x y&m&Begin a new subpath by moving the current point to coordinate $(x,y)$ in user space.\cr
x y&l&Append a line from the current point to $(x,y)$ in the current subpath.
The new current point is $(x,y)$.\cr
$x_1\ y_1\ x_2\ y_2\ x_3\ y_3$&c&Append a cubic B\'ezier curve to the current subpath.
The curve extends from the current point to $(x_3,y_3)$ using $(x_1,y_1)$ and $(x_2,y_2)$ as control points.\cr
$x_2\ y_2\ x_3\ y_3$&v&Append a cubic B\'ezier curve to the current subpath.
The curve extends from the current point to $(x_3,y_3)$ using the current point and $(x_2,y_2)$ as control
points.\cr
$x_1\ y_1\ x_3\ y_3$&y&Append a cubic B\'ezier curve to the current subpath.
The curve extends from the current point to $(x_3,y_3)$ using $(x_1,y_1)$ and $(x_3,y_2)$ as control
points.\cr
---&h&CLose the current subpath by appending a straight line segment from the current point to the starting
point of the subpath.
This terminates the current subpath.\cr
x y width height&re&Append a rectangle to the current path as a complete subpath.
\ethreetable

\noindent Doing
\blisting
x y width height re
\elisting
\noindent is equivalent to
\blisting
x y m
(x + width) y l
(x + width) (y + height) l
x (y + height) l
h
\elisting

\subsubsubsection{Cubic B\'ezier curves}

Curved path segments are defined using {\it cubic B\'ezier curves}.
B\'ezier curves are useful since they can be scaled, and subpaths are also B\'ezier curves, which makes them
very useful in computer graphics.

A cubic B\'ezier curve is defined by four points: beginning and end points $p_0,p_3$, as well as two control
points $p_1,p_2$.
Given these points, the cubic B\'ezier curve is parameterized by
$$ {\bf B}(t) = (1-t)^3p_0 + 3t(1-t)^2p_1 + 3t^2(1-t)p_2 + t^3p_3 $$
Note that ${\bf B}(0)=p_0$ and ${\bf B}(1)=p_1$, so this defines a curve between $p_0$ and $p_1$.
${\bf B}$ generally does not intersect $p_1,p_2$, but it always lies within the convex hull of the four points.
Another big reason for the use of B\'ezier curves is that changing the control points can {\it intuitively}
alter the curvature of the curve.

\bgroup
\centerline{%
\hbox to300pt{\vrule width\z@ height200pt depth\z@%
    \place{15}{0}{$p_0$}%
    \place{45}{150}{$p_1$}%
    \place{195}{150}{$p_2$}%
    \place{295}{50}{$p_3$}%
    \pdfliteral{q
        \nattrans,
        0 0 m
        50 150 200 150 300 50 c S
        q
        [5] 0 d
        0 0 m
        50 150 l S
        300 50 m
        200 150 l S
        Q
        .3 w
        \circle{0}{0}{2} f
        \circle{50}{150}{2} f
        \circle{200}{150}{2} f
        \circle{300}{50}{2} f
    Q}%
\hfil}}
\egroup

\bigskip

The above figure demonstrates a useful property of B\'ezier curves: the lines $p_0p_1$ and $p_2p_3$ are
tangent to the curve!
This helps explain how they can intuitively alter the curvature of the curve.

\subsubsection{Path painting operators}

We now describe the operators which can be used for painting (stroking and filling) paths:

\bthreetable{\it}{\bf}{}
{\bf Operands}&{\bf Operator}&{\bf Description}\cr\noalign{\hrule\vskip2\jot}
---&S&Stroke the path.\cr
---&s&Close and stroke the path.
Equivalent to {\tt h S}.\cr
---&f&Fill the path, using the nonzero winding number rule to determine the region to fill (see below).
All open subpaths are implicitly closed.\cr
---&f*&Same as {\bf f}, except using the even-odd rule.\cr
---&B&Fill then stroke the path, using the nonzero winding number rule.\cr
---&B*&Fill and then stroke the path, using the even-odd rule.\cr
---&b&Close, fill, then stroke the path, using the nonzero winding number rule.
(The closing affects only the stroking.)\cr
---&b*&Same as {\bf b} except using the even-odd rule.\cr
---&n&End the path without stroking or filling it.
\ethreetable

\subsubsubsection{The nonzero winding number rule}

The {\it nonzero winding number rule} determines whether a given point is inside the path by drawing a ray
from the given point to infinity in every direction.
For each ray, we count how many times the ray intersects the path.
When it intersects the path as it is going left-to-right, it adds one; and when it intersects the path as it
is going right-to-left, it subtracts one.
If the result is zero (for any ray), the point is outside the path; otherwise it is inside.

Instead of doing this for every ray, an arbitrary ray is chosen.

For convex shapes, this defines the inside and outside as you'd expect.
For more complicated boundaries, such as those which intersect themselves, we can get weirder results.
For example the five-point star:

\bgroup
\centerline{%
\hbox to191pt{\vrule width\z@ height181pt depth\z@%
    \pdfliteral{q
        .6 g 0 G
        1 j 1 J
        \nattrans,
        191 111 m
        37 0 l
        96 181 l
        154 0 l
        0 111 l b
    Q}%
\hfil}}
\egroup

\bigskip

And if we draw a donut shape, the direction we choose to draw it matters.
For example, if both circles are drawn in the same direction we get the left, but if they are drawn in opposite
directions, we get the right:

\bigskip

\bgroup
\centerline{%
\hbox to300pt{\vrule width\z@ height100pt depth\z@%
    \pdfliteral{q
        .6 g 0 G
        1 j 1 J
        \nattrans,
        \circle{50}{50}{50}
        \circle{50}{50}{30} b
        \circle{250}{50}{50}
        \rcircle{250}{50}{30} b
    Q}%
\hfil}}
\egroup

\bigskip

\subsubsubsection{The even-odd rule}

The even-odd rule is similar to the nonzero winding number rule.
Similar to the nonzero winding number rule, a ray is drawn in any direction from the point, and the number of
intersections is counted (without any subtractions).
If the number is odd, the point is inside; and if it is even, outside.

For convex shapes, this produces the same result as the nonzero winding number rule.
But for more complicated shapes, it produces different shapes:

\bigskip

\bgroup
\centerline{%
\hbox to191pt{\vrule width\z@ height181pt depth\z@%
    \pdfliteral{q
        .6 g 0 G
        1 j 1 J
        \nattrans,
        191 111 m
        37 0 l
        96 181 l
        154 0 l
        0 111 l b*
    Q}%
\hfil}\kern100pt%
\hbox to100pt{\vrule width\z@ height100pt depth\z@%
    \pdfliteral{q
        .6 g 0 G
        1 j 1 J
        \nattrans,
        \circle{50}{50}{50}
        \circle{50}{50}{30} b*
    Q}%
\hfil}%
}
\egroup

\subsubsection{Clipping path operators}

The graphics state contains a {\it current clipping path} which defines a boundary of a shape through which
painting operators are cropped.
Anything outside the boundary is cropped from the page.
The initial current clipping path includes the entire page.
After constructing a path and before painting it, a clipping operator ({\bf W} or {\bf W*}) may be added.
The clipping path is altered to be the intersection of the path with the current clipping path, and this is
done only after the path is painted.

\bthreetable{}{\bf}{}
{\bf Operands}&{\bf Operator}&{\bf Description}\cr\noalign{\hrule\vskip2\jot}
---&W&Modify the current clipping path by intersecting it with the current path, using the nonzero winding
number rule to determine what areas lay inside and outside the path.\cr
---&W*&Like {\bf W}, except it uses the even-odd rule.
\ethreetable

In order to alter the clipping path without painting a path, one can use the {\bf n} operator, which just
terminates the path without painting it.
So for example, one could do:

\blisting \setsyntax{TeX}
\pdfliteral{q}\rlap{\pdfliteral{0 0 100 4 re W n}hello}\pdfliteral{Q}
\elisting

\noindent(\macro\pdfliteral{} injects PDF code into the document, we'll discuss this later.)
This will write ``hello'', but only the coordinates which fall inside the rectangle of height 4.
This results in:

\pdfliteral{q}\rlap{\pdfliteral{0 0 100 4 re W n}hello}\pdfliteral{Q}

\bigskip

\noindent It's important to push the color state before setting the clipping path and pop after you've done
your clipping.
Otherwise all subsequent material will also be cropped.

\subsection{Color Spaces}

PDF provides the ability to input color in multiple ways.
For example, it supports both RGB and CMYK color spaces.
The current color space dictates the colors used for stroking and non-stroking operations (there exists
separate color spaces for each).

To provide a color, first a color space must be given.
We will discuss only the {\it device color spaces} now, and later we will discuss some {\it special color
spaces}.
There are three device color spaces: {\bf DeviceGray}, {\bf DeviceRGB}, and {\bf DeviceCMYK}.

{\bf DeviceGray} uses only shades of gray, which range between $0.0$ and $1.0$.
{\bf DeviceRGB} uses combinations of red, green, and blue whose values range between $0.0$ and $1.0$ to
produce color.
Similarly {\bf DeviceCMYK} uses combination of cyan, magenta, yellow, and black whose values range between
$0.0$ and $1.0$.
Importantly, RGB is an additive color system (it mixes light), and CMYK is subtractive (it mixes material;
adding colors makes it darker).

We can use the {\bf CS} and {\bf cs} operators to change the stroking and non-stroking color spaces
respectively.
Their only input is the name of a color space ({\bf DeviceGray}, {\bf DeviceRGB}, {\bf DeviceCMYK}).
Then to set the actual color value, one uses {\bf SC} or {\bf sc} to set the color.
Their input is a stream of numbers which corresponds to a color value.
For example, in the {\bf DeviceRGB} color space the stream must have $3$ numbers, while in the {\bf DeviceCMYK}
color space it must have $4$.

To summarize:

\bthreetable{\it}{\bf}{}
{\bf Operands}&{\bf Operator}&{\bf Description}\cr\noalign{\hrule\vskip2\jot}
name&CS&Sets the current color space for stroking operations.
{\it name} must be a name object ({\bf DeviceGray}, {\bf DeviceRGB}, or {\bf DeviceCMYK}).\cr
name&cs&Sets the current color space for non-stroking operations.
{\it name} must be a name object.\cr
$c_1\ \dots\ c_n$&SC&Set the color to be used for stroking operations.
Each value $c_i$ must be a number between $0.0$ and $1.0$.
The number of operators $n$ depends on the current color space: $1$ for {\bf DeviceGray}, $3$ for
{\bf DeviceRGB}, $4$ for {\bf DeviceCMYK}.\cr
$c_1\ \dots\ c_n$&sc&Same as {\bf SC} but for non-stroking operations.\cr
gray&G&The same as doing {\tt /DeviceGray CS} followed by {\it gray \tt SC}.\cr
gray&g&The same as doing {\tt /DeviceGray cs} followed by {\it gray \tt sc}.\cr
r g b&RG&The same as doing {\tt /DeviceRGB CS} followed by {\it r g b \tt SC}.\cr
r g b&rg&The same as doing {\tt /DeviceRGB cs} followed by {\it r g b \tt sc}.\cr
c m y k&K&The same as doing {\tt /DeviceCMYK CS} followed by {\it c m y k \tt SC}.\cr
c m y k&k&The same as doing {\tt /DeviceCMYK cs} followed by {\it c m y k \tt sc}.\cr
\ethreetable

