To create a PDF object, we use the \macro\pdfobj{} primitive.
Its usage is
\getmacrousage{\pdfobj reserveobjnum | [useobjnum <number>] [stream ^^O[<attr spec>]^^C] <general text>}
\noindent If {\tt reserveobjnum} is given, an object number is reserved, which can be accessed by
\macro\pdflastobj.
After doing {\tt reserveobjnum}, you can do {\tt useobjnum} to create a PDF object with that object number.

If {\tt stream} is specified, then the object created is a stream object, whose stream dictionary is specified
by {\it attr spec}.

The object is kept in memory, and not written to the PDF file immediately, until its object number is passed
to \macro\pdfrefobj{} or if \macro\pdfobj{} is preceeded by \macro\immediate.
Unlike form XObjects, \macro\pdfrefobj{} does not paint anything, it simply writes the PDF object to the file
at shipout time.

The object number can be accessed by \macro\pdflastobj{} after \macro\pdfobj.

\subsection{Tiling patterns}

Recall that tiling pattern's pattern cell is defined by a stream object of type {\bf Pattern}.
So for example, if we want to create an uncolored tiling pattern whose pattern cell is a circle.
We first create the pattern cell, which is just a hollowed out circle:

\blisting
\immediate\pdfobj stream attr{
    /Type /Pattern
    /PatternType 1
    /PaintType 2
    /TilingType 1
    /BBox [0 0 20 20] 
    /XStep 20
    /YStep 20
    /Resources << >>
}{
    q
    1 0 0 1 10 10 cm
    \circle{10}
    \circle{5} f*
    Q
}
\elisting

Now we create the outline of the pattern:

\blisting
\bgroup\setbox0=\hbox to200bp{\vrule width0bp height200bp depth0bp%
\pdfliteral{
    q
    \pttrans
    1 0 0 1 100 100 cm
    /CSP cs
    1 0 1 /P scn
    \circle{100}
    \circle{50} f*
    Q
}\hfil}
\pdfxform resources{
    /ColorSpace << /CSP [ /Pattern /DeviceRGB ] >>
    /Pattern << /P \the\pdflastobj\space 0 R >>
}0
\xdef\donutdonut{\pdfrefxform\the\pdflastxform\relax}
\egroup
\elisting

\immediate\pdfobj stream attr{
    /Type /Pattern
    /PatternType 1
    /PaintType 2
    /TilingType 1
    /BBox [0 0 20 20] 
    /XStep 20
    /YStep 20
    /Resources << >>
}{
    q
    1 0 0 1 10 10 cm
    \circle{10}
    \circle{5} f*
    Q
}
\bgroup\setbox0=\hbox to200bp{\vrule width0bp height200bp depth0bp%
\pdfliteral{%
    q
    \pttrans
    1 0 0 1 100 100 cm
    /CSP cs
    1 0 1 /P scn
    \circle{100}
    \circle{50} f*
    Q
}\hfil}
\pdfxform resources{%
    /ColorSpace << /CSP [ /Pattern /DeviceRGB ] >>
    /Pattern << /P \the\pdflastobj\space 0 R >>
}0
\xdef\donutdonut{\pdfrefxform\the\pdflastxform\relax}
\egroup

\centerline{\donutdonut}

\subsection{PostScript functions}

PostScript functions (type 4 functions) must be written in content streams, and thus require a stream
PDF object.
For example, suppose we want to create a functional shading which has a weird effect.
We define the function $f\colon[0,1]^2\to[0,1]$ by
$$ f(x,y) = \frac{1+\sin(1800(x-.5)^2)\sin(1800(y-.5)^2)}2 $$
(Trigonometric functions are computed in degrees.)
Then our shading will be determined by $f$.
To define the function we do:

\blisting
\immediate\pdfobj stream attr{
    /Type /Function
    /FunctionType 4
    /Domain [0 1 0 1]
    /Range [0 1]
}{{
    .5 sub
    exch
    .5 sub
    dup
    mul
    1800 mul
    sin
    exch
    dup
    mul
    1800 mul
    sin
    mul
    1 add
    2 div
}}
\elisting

Then the actual shading is done by

\blisting
\bgroup
\setbox0=\hbox to100bp{\vrule width0bp height100bp depth0bp%
\pdfliteral{
    /Sh sh
}\hfil}
\pdfxform resources {
    /Shading << /Sh <<
        /ShadingType 1
        /ColorSpace /DeviceGray
        /Domain [0 1 0 1]
        /Matrix [100 0 0 100 0 0]
        /Function \the\pdflastobj\space0 R
    >> >>
}0
\egroup
\elisting

This creates

\immediate\pdfobj stream attr{
    /Type /Function
    /FunctionType 4
    /Domain [0 1 0 1]
    /Range [0 1]
}{{
    .5 sub
    exch
    .5 sub
    dup
    mul
    1800 mul
    sin
    exch
    dup
    mul
    1800 mul
    sin
    mul
    1 add
    2 div
}}

\bgroup
\setbox0=\hbox to100bp{\vrule width0bp height100bp depth0bp%
\pdfliteral{
    /Sh sh
}\hfil}
\pdfxform resources {
    /Shading << /Sh <<
        /ShadingType 1
        /ColorSpace /DeviceGray
        /Domain [0 1 0 1]
        /Matrix [100 0 0 100 0 0]
        /Function \the\pdflastobj\space0 R
    >> >>
}0

\centerline{\pdfrefxform\pdflastxform}
\egroup

