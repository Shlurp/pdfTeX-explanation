The most important primitive for our uses that pdf\TeX{} provides is the \macro\pdfliteral{} macro.
The \macro\pdfliteral{} primitive inserts a whatsit (explained in the \TeX book, but not really important to
our discussion) into the output stream containing raw PDF code.
During the shipout of the current box, the PDF code is injected into the current content stream preceeding
the whatsit.
The exact usage of \macro\pdfliteral{} is as follows:

\getmacrousage{\pdfliteral [shipout] [direct | page] <general text>}

Where {\it general text} means material of the form \getmacrousageB where {\it balanced
text} is text with balanced curly braces.
A vanilla \macro\pdfliteral{} (meaning one without {\tt shipout} or {\tt direct} or {\tt page}) will insert
{\it general text} (which is PDF code) into the document at the current position.
It does the following:
\benum
    \item if in a text object, end the text object;
    \item set the CTM to the current position in the document;
    \item insert the PDF code.
\eenum
So for example, the result of

\blisting
Some text \pdfliteral{0 0 m 30 0 l S}\hbox to30bp{\hfil hi!\hfil}
\elisting

Will be the PDF code

\blisting \setsyntax{pdf}
BT
/F1 9.9626 Tf 91.925 759.927 Td [(Some)-333(text)]TJ
ET
1 0 0 1 139.248 759.927 cm
0 0 m 30 0 l S
1 0 0 1 -139.248 -759.927 cm
BT
/F1 9.9626 Tf 148.713 759.927 Td [(hi!)]TJ
ET
\elisting

\noindent We have here the following:
\benum
    \item a text object, with the text ``Some text'';
    \item a transformation operation ({\bf cm}), which transform CTM to the current position in the document;
    \item the inserted PDF code;
    \item a transformation operation, which moves the CTM back to where it was previously.
\eenum

Normally, {\it general text} is fully expanded when the whatsit is created.
This can be averted by using the {\tt shipout} keyword, which delays expansion to when the box is shipped out.

Furthermore, instead of ending a text object (if in one), if the {\tt direct} keyword is given, pdf\TeX{} will
add the PDF code to the text object.
It will also not transform user space (i.e. the code is placed verbatim into the PDF, no matter where).
Alternatively, {\tt page} will not transform user space, but it will exit a text object.

\bnote
    By default, units in a PDF are $1/72$ of an inch.
    These are called {\it big points} in \TeX, abbreviated to {\tt bp}.
    Normal \TeX{} {\tt pt}s are $1/72.27$ of an inch.
    So in a \macro\pdfliteral{} it is often useful to transform user space so that $1{\rm\ unit}=1{\tt pt}$.
    This can be done by scaling space by $72/72.27\approx.996264$.

    We define the macro
\blisting
\def\pttrans{.996264 0 0 .996264 0 0 cm }
\elisting
    \noindent and we will use it throughout this part.
    Note the space after the {\tt cm}, this is because operators must be followed with a space in a PDF.
    (E.g. you can't do \inlinecode|10 0 0 10 0 0 cm3 w|.)

    We also define the following macro which accepts a dimension expression, and expands to its result without
    the trailing {\tt pt}:
\blisting
\bgroup\lccode`?=`p\lccode`!=`t%
\lowercase{\egroup%
    \def\rmpt#1?!{#1}%
}%
\def\nopt#1{\expandafter\rmpt\the\dimexpr#1\relax}%
\elisting
    So, for example \inlinecode|\nopt{10pt+20pt}| will expand to {\tt30}.
    This is a useful macro when dealing with \macro\pdfliteral s (since their operands don't accept units).
    This macro uses the trick that \macro\lowercase{} doesn't expand its input, and converts character codes
    without altering category codes (and {\tt?} and {\tt!} have category code 12).
\eppbox

So suppose we'd like a macro which draws a dashed box around input text.
We could do:

\blisting
\def\dashbox#1{\hbox{%
    \setbox0=\hbox{#1}%
    \pdfliteral{
        q
        \pttrans
        .3 w
        [3] 0 d
        0 -\nopt{\dp0} \nopt{\wd0} \nopt{\dp0+\ht0} re
        S
        Q
    }%
    \box0%
}}
\elisting

This will give, for example:
\def\dashbox#1{\hbox{%
    \setbox0=\hbox{#1}%
    \pdfliteral{
        q
        \pttrans
        .3 w
        [3] 0 d
        0 -\nopt{\dp0} \nopt{\wd0} \nopt{\dp0+\ht0} re
        S
        Q
    }%
    \box0%
}}

\centerline{\dashbox{Heya there!}}

\noindent This could of course be done in Knuth-\TeX, but with significantly more difficulty

We can also create a macro which draws a circle around text.
We will use create an approximate circle with $4$ cubic B\'ezier curves.
The optimal control points are each about $0.5523$ units away from each endpoint (on a unit circle).

\blisting
\def\maxdim#1#2{%
    \ifdim\the\dimexpr#1\relax>\the\dimexpr#2\relax%
        \dimexpr#1\relax%
    \else%
        \dimexpr#2\relax%
    \fi%
}
\def\mult#1#2{\nopt{#1\dimexpr#2pt\relax}}
\def\circle#1{
    #1 0 m
    #1 \mult{.5523}{#1} \mult{.5523}{#1} #1 0 #1 c
    \mult{-.5523}{#1} #1 -#1 \mult{.5523}{#1} -#1 0 c
    -#1 \mult{-.5523}{#1} \mult{-.5523}{#1} -#1 0 -#1 c
    \mult{.5523}{#1} -#1 #1 \mult{-.5523}{#1} #1 0 c
    h
}
\def\circbox#1{\hbox{%
    \setbox0=\hbox{#1}%
    \edef\radius{\nopt{.5\maxdim{\wd0}{\ht0+\dp0}}}%
    \vrule width0pt depth\dimexpr\radius pt +.5\dp0-.5\ht0\relax height\dimexpr\radius pt+.5\ht0-.5\dp0\relax
    \pdfliteral{
        q
        .3 w
        [3] 0 d
        1 0 0 1 \nopt{.5\wd0} \nopt{.5\ht0-.5\dp0} cm
        \circle{\radius} S
        Q
    }\box0%
}}
\elisting

\def\maxdim#1#2{%
    \ifdim\the\dimexpr#1\relax>\the\dimexpr#2\relax%
        \dimexpr#1\relax%
    \else%
        \dimexpr#2\relax%
    \fi%
}
\def\mult#1#2{\nopt{#1\dimexpr#2pt\relax}}
\def\circle#1{
    #1 0 m
    #1 \mult{.5523}{#1} \mult{.5523}{#1} #1 0 #1 c
    \mult{-.5523}{#1} #1 -#1 \mult{.5523}{#1} -#1 0 c
    -#1 \mult{-.5523}{#1} \mult{-.5523}{#1} -#1 0 -#1 c
    \mult{.5523}{#1} -#1 #1 \mult{-.5523}{#1} #1 0 c
    h
}
\def\circbox#1{\hbox{%
    \setbox0=\hbox{#1}%
    \edef\radius{\nopt{.5\maxdim{\wd0}{\ht0+\dp0}}}%
    \vrule width0pt depth\dimexpr\radius pt +.5\dp0-.5\ht0\relax height\dimexpr\radius pt+.5\ht0-.5\dp0\relax
    \pdfliteral{
        q
        .3 w
        [3] 0 d
        1 0 0 1 \nopt{.5\wd0} \nopt{.5\ht0-.5\dp0} cm
        \circle{\radius} S
        Q
    }\copy0%
}}

\centerline{\circbox{Hey there!}}

Now, we can ask ourselves, why is our definition of \macro\circle{} so complicated?
Why not just define a parameterless \macro\unitcircle{} which draws a unit circle and then scale it?
Because then this will also scale the width of the line drawn, which will lead to undesirable results.
Also of course, this macro isn't perfect: the radius should really be $\sqrt{w^2+(h+d)^2}$ where $w,h,d$ are
the width, height, and depth of the box respectively.
But this requires being able to compute the square root, which is beyond the scope of this article.

Finally, the reason for the \macro\vrule{} is to ensure that the box is given the correct height.
\TeX{} is oblivious to the inserted PDF code, it doesn't know its dimensions, so it can't update the box
accordingly.

