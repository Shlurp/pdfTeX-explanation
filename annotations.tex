An annotation is an interactive feature of a PDF, which allows the user to interact with the PDF using their
keyboard and/or mouse.
In this section we discuss various (but not all) annotation types.
A page object has an optional {\bf Annots} field, which is an array of indirect references to all the
annotation objects occurring on the page.
Annotation objects are dictionaries which represent the annotation on the specified page.
An annotation may be referenced from a single page's {\bf Annots} field only; it cannot be referenced by two
pages simultaneously.

The fields common to all annotations are

\bdicttable
Type&name&(Optional) the type of PDF object this dictionary describes; must be {\bf Annot}.\cr
Subtype&name&(Required) the type of annotation this dictionary describes (see below).\cr
Rect&array&(Required) a rectangle (array of $4$ numbers $[x\ y\ w\ h]$) which specifies the location of the
annotation on the page.\cr
Contents&string&(Optional) text that shall be displayed for the annotation (if relevant for the type of
annotation); or an alternative description of the annotation's contents for accessibility readers.\cr
P&dictionary&(Optional) an indirect reference to the page object associated with this annotation.\cr
AP&dictionary&(Optional) an {\it appearance dictionary} which specifies
how the annotation should be presented on the page.\cr
AS&name&(Required if the annotation's appearance dictionary contains subdictionaries) the annotation's
{\it appearance state}, which selects the applicable appearance stream from an appearance subdictionary.\cr
Border&array&(Optional) an array specifying the characteristics of the annotation's border, which is a rounded
rectangle.
The array may have three or four elements, of the form $[r_h\ r_v\ w\ [o_n\ o_{\it ff}]]$ (the last element
may be omitted).
$r_h$ and $r_v$ are the horizontal and vertical radii, respectively.
$w$ is the stroke width of the border.
$[o_n\ o_{\it ff}]$ is the dash array of the border, it draws dashes of length $o_n$ with spaces of length
$o_{\it ff}$ (the phase is $0$).\cr
C&array&(Optional) an array of numbers in $[0,1]$ which represent the color to be used for
\blist
    \item the background of the annotation's icon when closed;
    \item the title bar of the annotation's popup window;
    \item the border of a {\bf Link} annotation.
\elist
The number of array elements determines the color space in which the color is to be interpreted.
\expandafter\def\csname enumstyle1\endcsname#1{#1}
\benum
    \global\advancecounter{enum count}{-1}
    \item no color; transparent;
    \item {\bf DeviceGray};
    \global\advancecounter{enum count}{1}
    \item {\bf DeviceRGB};
    \item {\bf DeviceCMYK};
\eenum
\edicttable

\subsection{Border styles}

An annotation may be surrounded by a border, printed within its annotation rectange, when displayed or printed.
Some annotations (see below) may use the {\bf BS} field in place of the {\bf Border} field to specify a
{\it border style dictionary} (instead of an array of numbers), which contains information on how the border
should be displayed.
The border style dictionary may have the following fields:

\bdicttable
Type&name&(Optional) the type of PDF object this dictionary describes; must be {\bf Border}.\cr
W&number&(Optional) the border width in points.
Default value: $1$.\cr
S&name&(Optional) the border style:
\blist
    \item {\bf S}: (solid) a solid rectangle;
    \item {\bf D}: (dashed) a dashed rectangle, the dash pattern may be specified by the {\bf D} entry;
    \item {\bf B}: (beveled) a simulated embossed rectangle that appears to be raised above the surface of the
    page.
    \item {\bf I}: (inset) a simulated engraved rectangle that appears to be recessed below the surface of the
    page.
    \item {\bf U}: (underline) a single line along the bottom of the annotation rectangle.
\elist
Default value: {\bf S}.\cr
D&array&A {\it dash array} of the form $[o_n\ o_{\it ff}]$ which specifies the pattern the dash pattern should
take on (on for $o_n$ units, off for $o_{\it ff}$ units).
Default value: $[3]$ (equivalent to $[3\ 3]$).
\edicttable

\subsection{Appearance streams}

An annotation may also specify {\it appearance streams}, which form alternatives to the border and color
attributes available via the {\bf AP} and {\bf AS} fields.
{\bf AP}, the appearance dictionary, may specify the following fields:

\bdicttable
N&stream or dictionary&(Required) the annotation's normal appearance.\cr
R&stream or dictionary&(Optional) the annotation's rollover appearance.
Default value: the value of the {\bf N} entry.\cr
D&stream or dictionary&(Optional) the annotation's down appearance.
Default value: the value of the {\bf N} entry.
\edicttable

\noindent The annotation has the appearance determined by {\bf N}; until the user's cursor hovers over the
annotation's active area, and the appearance changes to {\bf R}; when the user presses down the mouse button
within the annotation's active area, the appearance changes to {\bf D}.
The values of each entry may be a stream, or an {\it appearance subdictionary}.
In the latter case, the appearance subdictionary defines multiple appearance streams according to different
{\it appearance states} of the annotation.
In the latter case, the {\bf AS} must specify which appearance state is to be displayed by default.

For example, if we have a checkbox, it may have two states: {\bf On} and {\bf Off}.
We have appearances for normal and down, but we don't need one for rollover.
Then the appearance dictionary and appearance state may be:

\blisting
/AP <<
    /N <<
        /On 1 0 R
        /Off 2 0 R
    >>
    /D <<
        /On 3 0 R
        /Off 4 0 R
    >>
>>
/AS /Off
\elisting

\subsection{Text annotations}

A text annotation appears as some object on the page (according to its appearance dictionary), and when the
user hovers over it, a ``sticky note'' with specified text on it appears.
A text annotation's dictionary may have the following additional fields:

\bdicttable
Subtype&name&(Required) the type of annotation this dictionary describes; must be {\bf Text}.\cr
Open&boolean&(Optional) a flag specifying whether the annotation (the sticky note) should be displayed open
initially.
Default: {\tt false}.\cr
Name&name&(Optional) the name of an icon that should be used to display the annotation.
This field is only used if the appearance dictionary {\bf AP} is not specified.
Valid names are: {\bf Comment}, {\bf Key}, {\bf Note}, {\bf Help}, {\bf NewParagraph}, {\bf Paragraph},
{\bf Insert}.
Default: {\bf Note}.
\edicttable

For example, take the following annotation: 

\centerline{\pdfannot width10bp height10bp depth0pt{
    /Subtype /Text
    /Contents (This text should display when you hover over the icon)
    /Name /Help
}}

\noindent It can be created using the following code:

\blisting
<<
    /Type /Annot
    /Subtype /Text
    /Contents (This text should display when you hover over the icon)
    /Name /Help 
    /Rect [72 769.89 82 779.89]
>>
\elisting

\noindent We can also create a text annotation which alters its state depending on the whether the user's
cursor is hovering over it, or clicking, or not.
First we need to create its three states:

\blisting
1 0 obj
<<
/Type /XObject
/Subtype /Form
/BBox [0 0 50 50]
/FormType 1
/Matrix [1 0 0 1 0 0]
/Resources 2 0 R
/Length 645       
>>
stream
q 50 0 0 50 0 0 cm
BI
/BPC 8
/CS /DeviceRGB
/F [ /AHx ]
/W 9
/H 9
ID
... % code for smiley
EI
Q
endstream
endobj
4 0 obj
<<
/Type /XObject
/Subtype /Form
/BBox [0 0 50 50]
/FormType 1
/Matrix [1 0 0 1 0 0]
/Resources 5 0 R
/Length 645       
>>
stream
q 50 0 0 50 0 0 cm
BI
/BPC 8
/CS /DeviceRGB
/F [ /AHx ]
/W 9
/H 9
ID
... % code for straight face
EI
Q
endstream
endobj
6 0 obj
<<
/Type /XObject
/Subtype /Form
/BBox [0 0 50 50]
/FormType 1
/Matrix [1 0 0 1 0 0]
/Resources 7 0 R
/Length 645       
>>
stream
q 50 0 0 50 0 0 cm
BI
/BPC 8
/CS /DeviceRGB
/F [ /AHx ]
/W 9
/H 9
ID
... % code for frowny
EI
Q
endstream
endobj
\elisting

Then our annotation is simply
\blisting
% 8 0 obj
<<
    /Type /Annot
    /Subtype /Text
    /Contents (This is a face!)
    /AP <<
        /N 6 0 R
        /R 4 0 R
        /D 1 0 R
    >>
    /Rect [72 769.89 122 819.89]
>>
\elisting

You can try it out here:

\bgroup
\setbox0=\hbox to50bp{\vrule width0bp height50bp depth0bp%
\pdfliteral{
q
50 0 0 50 0 0 cm
BI
    /BPC 8
    /CS /DeviceRGB
    /F [ /AHx ]
    /W 9
    /H 9
ID
    ff0000 ff0000 ff0000 ff0000 ff0000 ff0000 ff0000 ff0000 ff0000 
    ff0000 ffffff ffffff ffffff ffffff ffffff ffffff ffffff ff0000 
    ff0000 ffffff 00ff00 00ff00 ffffff 00ff00 00ff00 ffffff ff0000 
    ff0000 ffffff 00ff00 00ff00 ffffff 00ff00 00ff00 ffffff ff0000 
    ff0000 ffffff ffffff ffffff ffffff ffffff ffffff ffffff ff0000 
    ff0000 ffffff 0000ff ffffff ffffff ffffff 0000ff ffffff ff0000 
    ff0000 ffffff ffffff 0000ff 0000ff 0000ff ffffff ffffff ff0000 
    ff0000 ffffff ffffff ffffff ffffff ffffff ffffff ffffff ff0000 
    ff0000 ff0000 ff0000 ff0000 ff0000 ff0000 ff0000 ff0000 ff0000 >
EI
Q
}\hfil}
\immediate\pdfxform0
\edef\Smiley{\the\pdflastxform}

\setbox0=\hbox to50bp{\vrule width0bp height50bp depth0bp%
\pdfliteral{
q
50 0 0 50 0 0 cm
BI
    /BPC 8
    /CS /DeviceRGB
    /F [ /AHx ]
    /W 9
    /H 9
ID
    ff0000 ff0000 ff0000 ff0000 ff0000 ff0000 ff0000 ff0000 ff0000 
    ff0000 ffffff ffffff ffffff ffffff ffffff ffffff ffffff ff0000 
    ff0000 ffffff 00ff00 00ff00 ffffff 00ff00 00ff00 ffffff ff0000 
    ff0000 ffffff 00ff00 00ff00 ffffff 00ff00 00ff00 ffffff ff0000 
    ff0000 ffffff ffffff ffffff ffffff ffffff ffffff ffffff ff0000 
    ff0000 ffffff ffffff ffffff ffffff ffffff ffffff ffffff ff0000 
    ff0000 ffffff 0000ff 0000ff 0000ff 0000ff 0000ff ffffff ff0000 
    ff0000 ffffff ffffff ffffff ffffff ffffff ffffff ffffff ff0000 
    ff0000 ff0000 ff0000 ff0000 ff0000 ff0000 ff0000 ff0000 ff0000 >
EI
Q
}\hfil}
\immediate\pdfxform0
\edef\Straight{\the\pdflastxform}

\setbox0=\hbox to50bp{\vrule width0bp height50bp depth0bp%
\pdfliteral{
q
50 0 0 50 0 0 cm
BI
    /BPC 8
    /CS /DeviceRGB
    /F [ /AHx ]
    /W 9
    /H 9
ID
    ff0000 ff0000 ff0000 ff0000 ff0000 ff0000 ff0000 ff0000 ff0000 
    ff0000 ffffff ffffff ffffff ffffff ffffff ffffff ffffff ff0000 
    ff0000 ffffff 00ff00 00ff00 ffffff 00ff00 00ff00 ffffff ff0000 
    ff0000 ffffff 00ff00 00ff00 ffffff 00ff00 00ff00 ffffff ff0000 
    ff0000 ffffff ffffff ffffff ffffff ffffff ffffff ffffff ff0000 
    ff0000 ffffff ffffff ffffff ffffff ffffff ffffff ffffff ff0000 
    ff0000 ffffff ffffff 0000ff 0000ff 0000ff ffffff ffffff ff0000 
    ff0000 ffffff 0000ff ffffff ffffff ffffff 0000ff ffffff ff0000 
    ff0000 ff0000 ff0000 ff0000 ff0000 ff0000 ff0000 ff0000 ff0000 >
EI
Q
}\hfil}
\immediate\pdfxform0
\edef\Frowny{\the\pdflastxform}

\centerline{\vrule width0bp height50bp depth0bp%
\pdfannot width50bp height50bp depth0bp{
    /Subtype /Text
    /Contents (This is a face!)
    /AP <<
        /N \Frowny\space0 R
        /R \Straight\space0 R
        /D \Frowny\space0 R
    >>
    /Border [1 1 1 [3]]
    /C [1 1 0]
}}
\egroup

\bnote
    Some annotations will only work on certain PDF viewers.
    For example, the above annotation will not change states on Sumatra PDF.
    It will on Adobe Acrobat (except it doesn't seem to support the down state).
\eppbox

There exist other annotation types, like caret annotations and stamp annotations, that do function well with
\TeX.
But since I don't see a use in them, and similar results can be obtained by text annotations, I will not
discuss them here.

\subsection{Link annotations}

A link annotation represents either a hypertext link to a destination elsewhere in the document, or an action
to be performed (explained below).
A link annotation may have the following additional entries in its dictionary:

\bdicttable
Subtype&name&(Required) the type of annotation this dictionary describes; must be {\bf Link}.\cr
A&dictionary&(Optional) an action dictionary (see below) describing the action to be performed when the link
annotation is clicked.\cr
Dest&array or name or string&(Optional; not permitted if {\bf A} is present) a destination (see below) that
will be jumped to when the link annotation is clicked.\cr
H&name&(Optional) the annotation's {\it highlighting mode}, which describes the visual effect to be displayed
upon the annotation being clicked.
{\bf H} may be one of the following:
\blist
    \item {\bf N} (none): no highlighting;
    \item {\bf I} (invert): invert the colors of the contents in the annotation rectangle (default value);
    \item {\bf O} (outline): invert the color of the annotation's border;
    \item {\bf P} (push): display the annotation as if it were being pushed down below the surface of the page.
\elist\cr
BS&dictionary&(Optional) a border style dictionary (see above) which describes the border (alternative to
{\bf Border}).
\edicttable

\subsubsection{Destinations}

A destination defines a particular view of a document, and consists of the following:
\benum
    \item the page of the document to be displayed;
    \item the location on the page to display;
    \item the magnification (zoom) factor.
\eenum

When a destination is specified, this can be either given as an {\it explicit destination} (which is an array),
a name (which is defined in the {\bf Dests} entry of the document catalog), or a string (whose value is defined
in the {\bf Dests} entry of the name dictionary of the document ({\bf Names} entry of the catalog)).
In the end these must all resolve to explicit destinations, whose syntax is described below:

\bgroup
\def\[{\hbox{\rm[}}\def\]{\hbox{\rm]}}
\btwotable{\it}{}
{\bf Syntax}&{\bf Meaning}\cr\hline
\[page {\bf /XYZ} left top zoom\]&Display the page designated by {\it page} (a page object), with the
coordinates $({\it left},{\it top})$ positioned in the upper-left corner of the window.
The page's contents shall be magnified by a factor of {\it zoom}.
A null value for any of the parameters (other than {\it page}) means that its value will remain unchanged.
A {\it zoom} of $0$ is equivalent to a null value.\cr
\[page {\bf/Fit}\]&Display the page designated by {\it page}, zoomed so that the page fits within the window
in its smaller dimension and centered in the other.\cr
\[page {\bf/FitH} top\]&Display the page {\it page} zoomed so that the page just fits into the window
horizontally, and positioned so that {\it top} is at the top of the page.\cr
\[page {\bf/FitV} left\]&Like {\bf FitH} but vertically.\cr
\[page {\bf/FitR} left bottom right top\]&Display the page {\it page} zoomed so that the rectangle defined by
{\it left bottom right top} fills the window.
If the magnification required for horizontal and vertical dimensions are different, use the smaller zoom factor
and center with the other dimension.
\etwotable
\egroup

\subsubsection{Actions}

An action is something that the PDF consumer can perform when something is triggered (the mouse clicking on
a region, moving over a region, etc.).
While actions are quite broad, we will discuss only a few of them and only a few trigger events.

An action is specified by an action dictionary, which may have the following fields (and more, according to
the type of action):

\bdicttable
Type&name&(Optional) the type of PDF object this dictionary describes; must be {\bf Action}.\cr
S&name&(Required) the type of action this dictionary describes (see below).\cr
Next&dictionary or array&(Optional) the next action, or actions, to execute after finishing this one.
\edicttable

Essentially each action dictionary forms a tree, whose value is the action to execute and children are the
next actions to execute (in order).
We discuss the following action types:

\subsubsubsection{Go-To actions}

A {\it go-to action} changes the view to a specified destionation.
The additional entries to a go-to action dictionary are as follows:

\bdicttable
S&name&(Required) the type of action this dictionary describes; must be {\bf GoTo}.\cr
D&name, string, or array&(Required) the destination to jump to.
\edicttable

\subsubsubsection{Launch actions}

A {\it launch action} opens an application or a document.
The additional entries to a launch action dictionary are as follows:

\bdicttable
S&name&(Required) the type of action this dictionary describes; must be {\bf Launch}.\cr
F&string&(Required) the application to launch or document to open.\cr
NewWindow&boolean&(Optional) if the file specified by {\bf F} is a PDF document, {\tt true} if it should be
opened in a new window, and {\tt false} if it should open it in the same window.
\edicttable

\subsubsubsection{URI actions}

A {\it uniform resource identifier} (URI) is a string that identifies (resolves to) a resource (either locally
or on the Internet).
A {\it URI action} causes a URI to be resolved.
The additional entries to a URI action dictionary are as follows:

\bdicttable
S&name&(Required) the type of action this dictionary describes; must be {\bf URI}.\cr
URI&string&(Required) the URI to resolve (encoded in UTF-8).
\edicttable
