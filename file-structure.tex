\subsection{File structure}

A ``canonical'' PDF file (pdf\TeX{} only creates canonical PDF files.
PDF files can be updated to create non-canonical PDFs, but we will not deal with those)
has the following structure:
\benum
    \item a one-line {\it header} which specifies the PDF version of the file;
    \item a {\it body} containing all the objects which make up the file;
    \item a {\it cross-reference stream} containing information regarding the indirect objects in the file;
    \item a {\it trailer} which specifies the location of the cross-reference stream.
\eenum

For example, if we were to compile the following \TeX{} file:
{\setsyntax{TeX}\listfile{hello-world.tex}}
we'd get the following PDF:
\listfile{hello-world.pdf}
I have truncated the PDF to save space and increase readability, but we still end up with over a hundred lines.
Oh well.

Let's go over this file:

\listfile{hello-world.pdf}[1-1]
\noindent this is the header, it specifies that this file conforms to PDF version 1.5.

Lines 3 through 96 comprise the body of the PDF file.

Lines 97 through 111 comprise the cross-reference table.
This contains information on how to randomly-access any indirect object within the file.
This information (in PDF-1.5) is stored in the cross-reference stream (lines 108--110).
The cross-reference stream is preceded by its stream dictionary, which may have the following fields:

\bdicttable
Type&name&(Required) the type of PDF object that the dictionary describes.
Must be {\bf XRef}.\cr
Size&integer&(Required; must not be indirect) the highest object number used in the file, plus one.\cr
Root&dictionary&(Required; must be indirect) an indirect reference to the catalog of the PDF (see below).\cr
Info&dictionary&(Optional; must be indirect) an indirect reference to the document's information dictionary.\cr
ID&array&(Optional) an array of two byte-strings (hexadecimal strings) which uniquely identifies the file.\cr
Index&array&(Optional) an array of two non-negative integers.
The first is the first object number in the file; the second is the number of objects.
The default value is then $[0\ {\bf Size}]$.\cr
W&array&(Required) an array of integers corresponding to the size of the fields in a single cross-reference
entry.
We won't develop this further.
\edicttable

The cross reference stream in this file is
\listfile{hello-world.pdf}[97-111]
\noindent and the trailer, which just specifies the byte offset of the cross reference stream, is
\listfile{hello-world.pdf}[112-114]

\subsection{Document structure}

\subsubsection{The document catalog}

The {\bf Root} field in the file's cross-reference stream is an indirect reference to the document's
{\it catalog}.
The document catalog stores information for objects in the body of the file which define the document's
content, outline, etc.
It also contains information on how the document should be displayed.

We see on line 103 that the {\bf Root} of the cross-reference stream is object {\tt10 0}.
This is the document catalog:
\listfile{hello-world.pdf}[89-93]
\noindent We see that the document catalog is a dictionary, and here it only has two entries.
But in general it can have more:

\bdicttable
Type&name&(Required) the type of PDF object the dictionary describes.
Must be {\bf Catalog}.\cr
Pages&dictionary&(Required; must be indirect) the {\it page tree node} (see below) that is the root of the
document's {\it page tree} (see below).\cr
PageLabels&number tree&(Optional) a number tree defining the page labeling for the document.
The keys in the tree are page indicies, and the values are {\it page label dictionaries} (see below).\cr
Names&dictionary&(Optional) The document's name tree (see below).\cr
Dests&dictionary&(Optional) A dictionary of names and corresponding destinations (see below when we discuss
hyperlinks).\cr
URI&dictionary&(Optional) A dictionary containing information for URI actions (see below when we discuss
hyperlinks).\cr
\edicttable

\subsubsection{The page tree}

The pages of a document are accessed through a structure known as the {\it page tree}.
This defines the structure of the pages in a document.
A page tree has two types of nodes: intermediate nodes and leaf nodes.
The most simple structure would be a single root intermediate node which contains as children all the pages
in the document as leaf nodes.
This is not efficient, so instead the tree is kept balanced generally.

A page tree node (intermediate node) is a dictionary with the following fields:

\bdicttable
Type&name&(Required) the type of PDF object that this dictionary describes.
Must be {\bf Pages}.\cr
Parent&dictionary&(Required except in root; must be indirect) a reference to the parent of the tree node.\cr
Kids&array&(Required) an array of indirect references to the immediate children of the node.\cr
Count&interger&(Required) the number of leaf nodes (page objects) that are descendants of this node.
\edicttable

Note that the page tree does not necessarily reflect the logical structure of the document (chapters, sections,
etc.).

A page object is a leaf in the page tree.
It is a dictionary with the fields listed below.
Some of the fields (those which are listed as such) may be inherited by ancestor nodes in the page tree.

\bdicttable
Type&name&(Required) the type of PDF object that this dictionary describes.
Must be {\bf Page}.\cr
Parent&dictionary&(Required; must be indirect) a reference to the parent of the page object in the page tree.
\cr
Resources&dictionary&(Required; inheritable) a dictionary containing any resources required by the page.
Omitting this indicates that it should be inherited.\cr
MediaBox&array&(Required; inheritable) an array of four numbers defining the boundaries of the page.\cr
Contents&stream&(Optional) a {\it content stream} (see below) which contains the contents of the page.\cr
Group&dictionary&(Optional) a {\it group attributes dictionary} specifying the attributes of the page's
page group for use in transparency (see below).\cr
Annots&array&(Optional) an array of {\it annotation dictionaries} representing annotations associated with
the page (see below).
\edicttable

\noindent More fields exist, but we ignore them for the sake of brevity.
In order to inherit an attribute, place it in a page tree node, and all page objects which are its descendants
will inherit the attribute.

We see in our document's catalog that the page tree root is object {\tt6 0}:
\listfile{hello-world.pdf}[83-88]
\noindent We see that there is a single page, whose object is {\tt2 0}:
\listfile{hello-world.pdf}[43-50]
\noindent The contents of the page are in object {\tt3 0}.
This is a content stream, which is a stream with operators telling the renderer how to display the page:
\listfile{hello-world.pdf}[3-13]
\noindent and the resources of the page are in object {\tt1 0}:
\listfile{hello-world.pdf}[51-55]
\noindent The resources dictionary defines {\tt/F1} to be a font which is a reference to object {\tt4 0},
which is a dictionary defining how to access the font.

\subsubsection{The name dictionary}

A document may have a {\it name dictionary}, specified in the {\bf Names} field of the document catalog.
This dictionary contains fields which themselves are name trees (see above) which map strings to objects.
For example, the name dictionary may have a name tree which maps strings to locations within the document.
These names may then be referred to (when specified) in other objects.

The name dictionary may have the following fields:

\bdicttable
Dests&name tree&(Optional) a name tree which maps strings to destinations (see below).\cr
AP&name tree&(Optional) a name tree which maps strings to annotation appearance streams (see below).
\edicttable

\noindent Other fields exist, but we will not discuss them.

\subsubsection{Resource Dictionaries}

Operands supplied to operators in content streams may only be direct objects.
This places a heavy restriction that must be somehow overcome.
For this reason, a content stream has a {\it resource dictionary}, defined by the {\bf Resources} entry
associated with it, in one of the following ways:
\blist
    \item for a content stream that is the value of a page's {\bf Contents} entry, the resource dictionary
    is named by the page's {\bf Resources} entry.
    \item for other content streams, the stream dictionary's {\bf Resources} specifies the resource dictionary.
    \item A form XObject (see below) may omit the {\bf Resources} entry, in which case the resources are looked
    up in the {\bf Resources} entry of the page it is used in.
\elist
The entries in the {\bf Resources} dictionary are as follows (all fields are optional).
All entries are explained in more depth in their respective sections below.

\bdicttable
ExtGState&dictionary&A dictionary that maps resource names to graphics state parameter dictionaries.\cr
ColorSpace&dictionary&A dictionary that each resource names to either the name of a device-dependent color
space or an array describing a color space (see below).\cr
Pattern&dictionary&A dictionary that maps resource names to pattern objects (see below).\cr
Shading&dictionary&A dictionary that maps resource names to shading dictionaries (see below).\cr
XObject&dictionary&A dictionary that maps resource names to XObjects. (see below).\cr
Font&dictionary&A dictionary that maps resource names to font dictionaries.\cr
ProcSet&array&An array of predefined procedure set names.\cr
Properties&dictionary&A dictionary that maps resource names to property list dictionaries for marked content.
\edicttable

Each field is explained in more depth later in this article.

