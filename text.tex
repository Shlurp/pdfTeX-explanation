We begin by distinguishing between two concepts: {\it characters} and {\it glyphs}.
Characters are abstract concepts (like the character ``A''), while a glyph is a specific rendering of a
character (e.g. {\it A}, {\tt A}, {\bf A} are all glyphs of the character ``A'').
A {\it font} defines the glyphs for a specific character set: for example computer modern roman defines the
glyphs for a set of standard Latin characters.
Fonts are defined in a form of program, called a {\it font program}, usually stored in a separate file, and
written in a specialized language such as the {\it Type 1} or {\it TrueType} font formats.
PDFs utilize a special kind of dictionary, called a {\it font dictionary}, to identify the font program
and additional information about the font.

A content stream can paint glyphs on a page by first specifying a font dictionary and then a string object
which is interpreted as a sequence of character codes specifying glyphs in the font.
The {\it glyph description}, stored in the font program, provides a sequence of graphics operators to execute
in order to paint the glyph.
A PDF consumer renders glyphs by then executing this sequence (for performance, glyphs can be cached).

\subsection{Glyph structure}

A glyph has a {\it glyph coordinate system}, which is the coordinate system in which it is defined.
All graphics operations related to painting a glyph are done relative to the glyph coordinate system.
Except for Type 3 fonts, units in the glyph space are $1/1000$th of a user space unit.

A glyph has the following structure:

\bigskip
\bgroup
\catcode`_=11
\def\wbuf{2}
\setbox1=\hbox{\setscale{8pt}origin }
\edef\dxl{\_nopt{\wd1}}
\setbox2=\hbox{ \vbox{\setscale{8pt}\halign{\hfil#\hfil\cr next\cr glyph\cr origin\cr}}}
\edef\dxr{\_nopt{\wd2}}
\setbox0=\hbox{\font\@@=ptmr7t\@@ g}
\centerline{\hbox to\dimexpr\wd0+\dxl pt+\dxr pt\relax{\vrule width\z@ height10\ht0 depth10\dp0%
\place{0}{\_nopt{.5\ht1}}{\box1}%
\place{\_nopt{\dxl pt+10\wd0}}{\_nopt{.5\ht2}}{\box2}%
\setbox2=\hbox{\setscale{8pt}width}%
\setbox1=\vtop to\dimexpr10\dp0+6pt+\ht2\relax{\vfil\hbox{width}}%
\place{\_nopt{\dxl pt+5\wd0-.5\wd2}}{0}{\box1}%
\pdfliteral{%
q
    \nattrans,
    1 0 0 1 \dxl\space 0 cm
    1 j 1 J
    .3 w
    q
    [3] 0 d
    \wbuf\space -\_nopt{10\dp0} \_nopt{10\wd0-\wbuf pt * 2} \_nopt{10\dp0+10\ht0} re S
    Q
    0 3 m
    0 -\_nopt{10\dp0+6pt} l S
    0 0 m
    \_nopt{10\wd0} 0 l
    \_nopt{10\wd0} 3 l
    \_nopt{10\wd0} -\_nopt{10\dp0+6pt} l S
    0 -\_nopt{10\dp0+3pt} m
    \_nopt{10\wd0} -\_nopt{10\dp0+3pt} l S
    1 Tr
    .1 w
    10 0 0 10 0 0 cm
}%
\rlap{\box0}%
\pdfliteral{Q}\hfil%
}}
\egroup
\vskip1cm

\noindent The {\it bounding box} is the dashed box around the glyph.
It is the smallest box which encompasses the glyph shape.
The {\it origin} of the glyph is point $(0,0)$ in the glyph coordinate system.
The {\it glyph displacement} is a vector from the origin which gives the coordinate where the next glyph's
origin should be placed.
A glyph displacement that is horizontal is generally called the {\it width}.
We deal here with only horizontal writing modes.

\subsection{Text state}

The current {\it text state} is composed of nine parameters which affect only the painting of text.
These are:

\btwotable{}{}
{\bf Parameter}&{\bf Description}\cr\hline
$T_c$&Character spacing\cr
$T_w$&Word spacing\cr
$T_h$&Horizontal scaling\cr
$T_l$&Leading\cr
$T_f$&Text font\cr
$T_{\it fs}$&Text font size\cr
$T_{\it mode}$&Text rendering mode\cr
$T_{\it rise}$&Text rise\cr
$T_k$&Text knockout
\etwotable

$T_f$ and $T_{\it fs}$ are self-explanatory.
We will not discuss $T_k$.
Within a text object, two further parameters are defined: $T_m$ the text matrix; $T_{\it lm}$ the text line
matrix; and $T_{\it rm}$ the text rendering matrix.

The parameters can be altered using the following operators:

\bthreetable{\it}{\bf}{}
{\bf Operands}&{\bf Operator}&{\bf Description}\cr\hline
charSpace&Tc&Sets the character space $T_c$ to {\it charSpace} which is a unit in unscaled text space units.
This is used by {\bf Tj}, {\bf TJ}, and {\bf '} operators.
Initial value: $0$.\cr
wordSpace&Tw&Sets the word space $T_w$ to {\it wordSpace} which is a unit in unscaled text space units.
This is used by {\bf Tj}, {\bf TJ}, and {\bf '} operators.
Initial value: $0$.\cr
scale&Tz&Set the horizontal scaling $T_h$ to ${\it scale}/100$.
{\it scale} is a number specifying the percentage of the normal width.
Initial value: $100$.\cr
leading&TL&Set the text leading $T_l$ to {\it leading}, a number expressed in unscaled text space units.
Text leading is used by {\bf T*}, {\bf'}, and {\bf"} operators.\cr
font size&Tf&Set the text font $T_f$ to {\it font} and font size $T_{\it fs}$ to {\it size}.
{\it font} is the name of a font resource in the {\bf Font} subdictionary of the current resource dictionary.
There is no initial value for either of these parameters; they must be specified before painting glyphs.\cr
render&Tr&Set the text rendering mode $T_{\it mode}$ to {\it render}, an integer.
Initial value: $0$.\cr
rise&Ts&Set the text rise $T_{\it rise}$ to {\it rise}, which is a number in unscaled text space units.
Initial value: $0$.
\ethreetable

A parameter in {\it unscaled} text space units means it is specified in a coordinate system defined by the
text matrix $T_m$, but not scaled by the font size $T_{\it fs}$.

\subsubsection{Character spacing}

The character spacing parameter $T_c$ specifies the amount of spacing between characters.
It is specified in unscaled text space, but is subject to scaling by the horizontal scaling parameter
$T_h$.
$T_c$ is added to the horizontal component of the widths displacement.
For example:

\moveright1cm\hbox{$T_c=0$: Character}
\moveright1cm\hbox{$T_c=2$: \setbox0=\hbox{Character}\vrule width\z@ depth\dp0 height\ht0%
\pdfliteral{q 2 Tc}\rlap{Character}\pdfliteral{Q}}

\subsubsection{Word spacing}

Word spacing acts similarly to character spacing, but only affects the space character, code $32$.
\TeX{} treats spaces differently, and thus this parameter doesn't have an effect on strings written using
\TeX.

\subsubsection{Horizontal scaling}

$T_h$ specifies how much the horizontal component of the glyphs should be stretched as a percentage.
This also affects $T_c$ (character spacing) and $T_w$ (word spacing).

\moveright1cm\hbox{$T_h=100$: Word}
\moveright1cm\hbox{$T_h=50$: \setbox0=\hbox{WordWord}\vrule width\z@ depth\dp0 height\ht0%
\pdfliteral{q 50 Tz}\rlap{WordWord}\pdfliteral{Q}}

\subsubsection{Leading}

$T_l$ specifies the distance between baselines (like \macro\baselineskip).
This is not used by \TeX.

\subsubsection{Text rendering mode}

The text rendering mode $T_{\it mode}$ determines how glyphs are shown.
It can cause glyphs to be stroked, filled, or added to the clipping path (or some combination of the three).
The following table desribes the text rendering modes:

\bthreetable{}{}{}
{\bf Mode}&{\bf Example}&{\bf Description}\cr\hline
0&\demotext{R}{0}&Fill text\cr
1&\demotext{R}{1}&Stroke text\cr
2&\demotext{R}{2}&Fill, then stroke text\cr
3&\demotext{R}{3}&Neither fill nor stroke text (invisible)\cr
4&\democlip{R}{0}{3}&Fill text and add to path for clipping\cr
5&\democlip{R}{3}{1}&Stroke text and add to path for clipping\cr
6&\democlip{R}{0}{1}&Fill, then stroke text, and add to path for clipping\cr
7&\democlip{R}{3}{3}&Add to path for clipping
\ethreetable

The text rendering mode must not change inside a text object (see below).
The glyphs are accumulated inside a text object, and at the end form a path where the outlines of each glyph
are considered as subpaths.
Then text, if specified, is filled using the nonzero winding number rule.
The current clipping path, if specified, is intersected with this path (after all the stroking and filling
operations).
This remains in effect until an invocation of the {\bf Q} operator.

\subsubsection{Text rise}

Text rise, $T_{\it rise}$, specifies the distance in unscaled text space units, to move the baseline up or
down from its default location.
For example:

\blisting
(This ) Tj
-5 Ts
(text ) Tj
5 Ts
(moves ) Tj
0 Ts
(around) Tj
\elisting

will produce

\medskip
\hbox{This \pdfliteral direct{-5 Ts}text \pdfliteral direct{5 Ts}moves \pdfliteral direct{0 Ts}around}
\medskip

\subsection{Text objects}

A PDF text object consists of operators that can show text strings, move the text position, and set text state
and other parameters.
Additionally, three parameters are defined only within a text object and do not persist between text objects:
\benum
    \item the {\it text matrix}, $T_m$;
    \item the {\it text line matrix}, $T_{\it lm}$;
    \item the {\it text rendering matrix}, $T_{\it rm}$.
        This is actually just an intermediate result that combines the effects of text state parameters,
        the text matrix, and the CTM.
\eenum
\noindent A text object has the form
\blisting
BT
    ... text or other allowed operators ...
ET
\elisting
\noindent{\bf BT} begins a text object, which initializes $T_m$ and $T_{\it lm}$ to the identity matrix.
{\bf ET} ends the text object.
Text objects cannot be nested.

\subsubsection{Text space details}

Text is shown in {\it text space}, which is defined by a combination of the text matrix $T_m$, and the text
state parameters $T_{\it fs}$ (font size), $T_h$ (horizontal scaling), and $T_{\it rise}$ (text rise).
This determines how text coordinates are transformed into user space.
The transformation from text space to user space is represented by an intermediate object, the {\it text
rendering matrix}, $T_{\it rm}$, which is:
$$ T_{\it rm} = {\it CTM}\cdot T_m\cdot\pmatrix{T_{\it fs}\cdot T_h&0&0\cr0&T_{\it fs}&T_{\it rise}\cr0&0&1} $$
That is, first we map a point $(x,y)$ according to the text state parameters, to
$$ \pmatrix{x\cr y}\mapsto\pmatrix{T_{\it fs}\cdot T_h\cdot x\cr T_{\it fs}\cdot y+T_{\it rise}} $$
then we map this according to the text matrix and the CTM to user space.
$T_{\it rm}$ is an intermediate result, ostensibly recomputed before each glyph is painted.

After a glyph is painted, the text matrix is updated according to the glyph's displacement and any spacing
parameters which may apply.
We define
$$ t_x= \left(\left(w-\frac{T_j}{1000}\right)\cdot T_{\it fs}+T_c+T_w\right)\cdot T_h $$
where
\blist
    \item $w$ is the width of the glyph;
    \item $T_j$ is a positioning adjustment (specified in a {\bf TJ} array, see below);
    \item $T_{\it fs},T_h$ are text state parameters corresponding to font size and horizontal scaling,
        respectively.
    \item $T_c,T_w$ are text state parameters corresponding to the current character- and word-spacing
        parameters in the graphics state (only applied if necessary; e.g. $T_w$ is only applied between
        words).
\elist

In addition to $T_m$, another parameter is stored: $T_{\it lm}$.
This is the value of $T_m$ at the beginning of the line.
This makes it easier to move to the next line: simply set $T_m$ to $T_{\it lm}$ displaced by some displacement.
In particular, if you want to move to the next line, displaced by $(\Delta_x,\Delta_y)$, set
$$ T_m = T_{\it lm} = T_{\it lm}\cdot\pmatrix{1&0&\Delta_x\cr0&1&\Delta_y\cr0&0&1} $$

\subsubsection{Text placement operators}

The following operators alter $T_m$ and $T_{\it lm}$:

\bthreetable{\it}{\bf}{}
{\bf Operands}&{\bf Operator}&{\bf Description}\cr\hline
$\Delta_x\ \Delta_y$&Td&Move to the start of the next line, offset from the start of the current line by
$\Delta=(\Delta_x,\Delta_y)$.
$\Delta_x,\Delta_y$ are expressed in unscaled text space units.
This has the effect of doing
$$ T_m = T_{\it lm} = T_{\it lm}\cdot\pmatrix{1&0&\Delta_x\cr0&1&\Delta_y\cr0&0&1} $$\cr
$\Delta_x\ \Delta_y$&TD&Move to the start of the next line, offset from the start of the current line
by $\Delta=(\Delta_x,\Delta_y)$.
As a side effect, sets the leading parameter in the text state.
This has the same effect as doing {\tt$-\Delta_y$ TL} and then {\tt$\Delta_x\ \Delta_y$ Td}.\cr
a b c d e f&Tm&Sets the text matrix and text line matrix
$$ T_m = T_{\it lm} = \pmatrix{a&c&e\cr b&d&f\cr 0&0&1} $$
The operands are all numbers.
The matrices are not composed with the previous values of $T_m$ or $T_{\it lm}$.\cr
---&T*&Move to the start of the next line.
This is the same as doing {\tt0 $T_l$ Td}, where $T_l$ is the current leading parameter in the text state.
\ethreetable

\subsubsection{Text-showing operators}

The following operators show text on the page.
They reposition the glyphs as they do so.

\bthreetable{\it}{\bf}{}
{\bf Operands}&{\bf Operator}&{\bf Description}\cr\hline
string&Tj&Show a text string.\cr
string&{\tt'}&Move to the next line and show a text string.
This has the same effect as doing {\tt T*} and then {\it string \tt Tj}.\cr
$a_w\ a_c$ string&{\tt"}&Move to the next line and show a text string, using $a_w$ for word spacing and $a_c$
for character spacing.
This has the same effect as doing {\tt $a_w$ Tw}, {\tt $a_c$ TC}, and then {\it string \tt '}.\cr
array&TJ&Show one or more text strings, allowing individual glyph positioning.
Each element of {\it array} can be a string or a number.
If a string, the operator shows the string.
If a number, the operator adjusts the text position by that amount.
The number is expressed in one-one thousandth of a unit in unscaled text space (this is the value $T_j$ above).
A positive value has the effect of moving a glyph to the left, and a negative to the right.
This is hard to interact with in \TeX.
\ethreetable

