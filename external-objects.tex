An external object (XObject) is a graphics object whose content is defined by a self-contained content stream,
separate from the content stream in which it is used.
There are three types of XObjects, but we won't discuss the third (PostScript XObjects) as their use is
discouraged.
The other two types of XObjects are
\benum
    \item {\it image XObjects} which represent images;
    \item {\it form XObjects} which is are self-contained descriptions of an arbitrary sequence of
    graphics objects.
    Two subtypes of form XObjects are {\it group XObjects} and {\it reference XObjects}.
\eenum

If an XObject is named by {\it name} in the {\bf XObject} subdictionary of the {\bf Resource} dictionary of
a content stream, it can be painted using the {\bf Do} operator.
For example, we may define the following form XObject:

\blisting
1 0 obj
<<
    /Type /XObject
    /Subtype /Form
    /BBox [0 0 10 1]
    /FormType 1
    /Matrix [1 0 0 1 0 0]
    /Resources 7 0 R
    /Length ...
>>
stream
1 w
1 0 0 1 0 .5 cm
0 0 m 10 0 l S
endstream
endobj
\elisting

\noindent which simply paints a line of length $10$.
Now, in another content stream we can paint this XObject using the {\bf Do} operator:

\blisting
4 0 obj
<<
    /Length 34        
>>
stream
q
1 0 0 1 72 765.905 cm
/Fm1 Do
Q
endstream
endobj
\elisting

\noindent We see that {\bf Do} references the name {\bf Fm1}, which is defined in the page's {\bf Resource}
dictionary to be a reference to object {\tt1 0}.

\blisting
% 2 0 obj
<<
    /XObject << /Fm1 1 0 R >>
    /ProcSet [ /PDF ]
>>
\elisting

\subsection{Images}

There are two ways to insert images into a PDF.
One is using image XObjects, and another is using inline images where the content of the image is represented
inline directly in a content stream.
Image XObjects will be discussed later, with their respective pdf\TeX{} primitives.

An inline image is declared in a content stream in the following way:

\blisting
BI  % begin inline image object
... key-value pairs ...
ID  % begin image data
... image data ...
EI  % end inline image object
\elisting

\noindent That is, the {\bf BI} operator begins the inline image dictionary (except its just a stream of
key-value pairs, not wrapped in \inlinecode|<<...>>|), {\bf ID} ends it and begins the image data (which
specifies how to color each pixel), and {\bf EI} ends the inline image object.
The fields that can placed in between {\bf BI} and {\bf ID} are

\bdicttable
BitsPerComponent&integer&(Required) the number of bits used to represent each color component (a single value
in a color value).
Valid values are $1,2,4,8,16$.\cr
Width&integer&(Required) the width of the image, in samples.\cr
Height&integer&(Required) the height of the image, in samples.\cr
ColorSpace&name&(Required) the color space that the image is defined in.\cr
Filter&array&(Required) an array of filters to be applied to the image data to decode it.
\edicttable

We can use the following abbreviations for the fields:
\blist
    \item {\bf BitsPerComponent} --- {\bf BPC}
    \item {\bf Width} --- {\bf W}
    \item {\bf Height} --- {\bf H}
    \item {\bf ColorSpace} --- {\bf CS}
    \item {\bf Filter} --- {\bf F}
\elist

So if {\bf CS} requires $n$ values per color (e.g. $3$ for RGB), {\bf W} is $w$, and {\bf H} is $h$,
the image data should have $n\cdot w\cdot h$ samples.
Each sample must have a size of {\bf BPC} after being decoded by the filter.
The only filter we will discuss is {\bf AHx} (or {\bf ASCIIHexDecode}), which decodes ASCII hex codes.
That is, it takes two ASCII characters (which are either $0-9$ or $a-z$ or $A-Z$), and computes the hex value.
Whitespace is ignored.

\bnote
The stream must be ended by a {\tt>} character, as seen below.
\eppbox

For example, we may have the inline image:

\blisting
q
.996264 0 0 .996264 0 0 cm
100 0 0 100 0 0 cm
BI
    /BPC 8
    /CS /DeviceRGB
    /F [ /AHx ]
    /W 9
    /H 9
ID
    ff0000 ff0000 ff0000 ff0000 ff0000 ff0000 ff0000 ff0000 ff0000 
    ff0000 ffffff ffffff ffffff ffffff ffffff ffffff ffffff ff0000 
    ff0000 ffffff 00ff00 00ff00 ffffff 00ff00 00ff00 ffffff ff0000 
    ff0000 ffffff 00ff00 00ff00 ffffff 00ff00 00ff00 ffffff ff0000 
    ff0000 ffffff ffffff ffffff ffffff ffffff ffffff ffffff ff0000 
    ff0000 ffffff 0000ff ffffff ffffff ffffff 0000ff ffffff ff0000 
    ff0000 ffffff ffffff 0000ff 0000ff 0000ff ffffff ffffff ff0000 
    ff0000 ffffff ffffff ffffff ffffff ffffff ffffff ffffff ff0000 
    ff0000 ff0000 ff0000 ff0000 ff0000 ff0000 ff0000 ff0000 ff0000 >
EI
Q
\elisting

Which draws the following:

\centerline{\hbox to100pt{\vrule width\z@ height100pt depth\z@%
\pdfliteral{
q
.996264 0 0 .996264 0 0 cm
100 0 0 100 0 0 cm
BI
    /BPC 8          % each value is a byte
    /CS /DeviceRGB  % interpret colors as RGB
    /F [ /AHx ]     % the stream is given in ASCII representing hexadecimal values
    /W 9            % each line is given by 9 samples
    /H 9            % 9 lines
ID
    ff0000 ff0000 ff0000 ff0000 ff0000 ff0000 ff0000 ff0000 ff0000 
    ff0000 ffffff ffffff ffffff ffffff ffffff ffffff ffffff ff0000 
    ff0000 ffffff 00ff00 00ff00 ffffff 00ff00 00ff00 ffffff ff0000 
    ff0000 ffffff 00ff00 00ff00 ffffff 00ff00 00ff00 ffffff ff0000 
    ff0000 ffffff ffffff ffffff ffffff ffffff ffffff ffffff ff0000 
    ff0000 ffffff 0000ff ffffff ffffff ffffff 0000ff ffffff ff0000 
    ff0000 ffffff ffffff 0000ff 0000ff 0000ff ffffff ffffff ff0000 
    ff0000 ffffff ffffff ffffff ffffff ffffff ffffff ffffff ff0000 
    ff0000 ff0000 ff0000 ff0000 ff0000 ff0000 ff0000 ff0000 ff0000 >
EI
Q
}\hfil}}

\bigskip

Recall that an image is by default one unit by one unit, so we must scale it to get the desired size.

\subsection{Form XObjects}

A form XObject is a self-contained content stream which can be painted within other content streams.
The graphics defined in a form XObject are defined relative to its {\it form coordinate space}, which can
then be scaled to user space using its {\bf Matrix} field.

Form XObjects may have the following fields in their stream dictionary, as well as fields common to all
content streams:

\bdicttable
Type&name&(Optional) the type of PDF object the dictionary describes; must be {\bf XObject}.\cr
Subtype&name&(Required) the type of XObject the dictionary describes; must be {\bf Form}.\cr
BBox&array&(Required) a bounding box which defines the boundary of the XObject in form space.
The XObject is clipped to this boundary.\cr
Matrix&array&(Optional) a matrix which represents the transformation from form to user space.\cr
Resources&dictionary&(Required) a dictionary specifying the resources required by the form XObject.\cr
Group&dictionary&(Optional) a {\it group attributes dictionary} indicating that the contents of the form
XObject are to be treated as a groip and specifying the attributes of that group (see below).
\edicttable

\subsubsection{Group XObjects}

A {\it group XObject} is a type of form XObject that can be used to group graphical elements together for
various purposes.
It is declared by adding the {\bf Group} field to a form XObject's dictionary.
This is a dictionary called the {\it group attributes dictionary}, which may have the following fields:

\bdicttable
Type&name&(Optional) the type of PDF object that this dictionary describes; must be {\bf Group}.\cr
S&name&(Required) the {\it group subtype}, which identifies the kind of group the dictionary describes.
The only available group subtype is {\bf Transparency}.
Transparency group XObjects are discussed later.\cr
\edicttable

