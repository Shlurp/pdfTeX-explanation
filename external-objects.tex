An external object (XObject) is a graphics object whose content is defined by a self-contained content stream,
separate from the content stream in which it is used.
There are three types of XObjects, but we won't discuss the third (PostScript XObjects) as their use is
discouraged.
The other two types of XObjects are
\benum
    \item {\it image XObjects} which represent images;
    \item {\it form XObjects} which is are self-contained descriptions of an arbitrary sequence of
    graphics objects.
    Two subtypes of form XObjects are {\it group XObjects} and {\it reference XObjects}.
\eenum

If an XObject is named by {\it name} in the {\bf XObject} subdictionary of the {\bf Resource} dictionary of
a content stream, it can be painted using the {\bf Do} operator.
For example, we may define the following form XObject:

\blisting
1 0 obj
<<
    /Type /XObject
    /Subtype /Form
    /BBox [0 0 10 1]
    /FormType 1
    /Matrix [1 0 0 1 0 0]
    /Resources 7 0 R
    /Length ...
>>
stream
1 w
1 0 0 1 0 .5 cm
0 0 m 10 0 l S
endstream
endobj
\elisting

\noindent which simply paints a line of length $10$.
Now, in another content stream we can paint this XObject using the {\bf Do} operator:

\blisting
4 0 obj
<<
    /Length 34        
>>
stream
q
1 0 0 1 72 765.905 cm
/Fm1 Do
Q
endstream
endobj
\elisting

\noindent We see that {\bf Do} references the name {\bf Fm1}, which is defined in the page's {\bf Resource}
dictionary to be a reference to object {\tt1 0}.

\blisting
% 2 0 obj
<<
    /XObject << /Fm1 1 0 R >>
    /ProcSet [ /PDF ]
>>
\elisting

\subsection{Images}

There are two ways to insert images into a PDF.
One is using image XObjects, and another is using inline images where the content of the image is represented
inline directly in a content stream.

\subsubsection{Image XObjects}

An external image file may be read into a PDF and referenced via an image XObject.
An image XObject is a content stream: the stream itself represents the image data, and the dictionary (the
image dictionary) contains the image's metadata.

An image is two-dimensional array of samples, each sample representing a color.
Each sample consists of as many color components as are needed for the color space in which they are specified
(1 for {\bf DeviceGray}, 3 for {\bf DeviceRGB}, 4 for {\bf DeviceCMYK}).
Each component may be a $1-,2-,4-,8-,16-$ bit integer.

Fields in the image dictionary may include:

\bdicttable
Type&name&(Optional) the type of PDF object this dictionary describes; must be {\bf XObject}.\cr
Subtype&name&(Required) the type of XObject this dictionary describes;\hfil\break must be {\bf Image}.\cr
Width&integer&(Required) the width of the image, in samples.\cr
Height&integer&(Required) the height of the image, in samples.\cr
ColorSpace&name&(Required for images; not permitted for image masks (see below)) the color space in
which the color samples reside.\cr
BitsPerComponent&integer&(Required except for image masks) the number of bits used to represent each color
component.
This may be $1,2,4,8,$ or $16$.\cr
ImageMask&boolean&(Optional) a flag which indicates whether this image shall be an image mask (see below).
If {\tt true}, {\bf BitsPerComponent} must be $1$, and {\bf ColorSpace} and {\bf Mask} must be left
unspecified.
Default: {\tt false}.\cr
Mask&stream or array&(Optional; not permitted for image masks) an image XObject defining an image mask to
apply to this image; or an array specifying a range of colors to be applied as a color key mask (see below).
If {\bf ImageMask} is {\tt true}, must be left undefined.\cr
Decode&array&(Optional) an array of numbers describing how to map image samples into the range of values
appropriate for the image's color space.
If {\bf ImageMask} is {\tt true}, must be either $[0\ 1]$ or $[1\ 0]$.
Otherwise it must be an array of length $2n$ where $n$ is the number of color components in the color space.
The default value is $[0\ 1\ \dots\ 0\ 1]$.\cr
SMask&stream&(Optional) an image XObject which defines a soft-mask image for the current image.
The alpha value is determined by the luminosity of each pixel, and affects the shape or opacity depending on
the {\bf AIS} flag in the current graphics state.\cr
\edicttable

{\bf Decode} decodes the samples so that the color components fall into the range required for the color space.
Given a sample of $c_1\ \dots\ c_n$ and a decode array of $[D_{11}\ D_{12}\ \dots\ D_{n1}\ D_{n2}]$, the
color components are decoded to be
$$ c'_i = \interpolate(c_i;0,2^b-1,D_{i1},D_{i2}) $$
where $b$ is the {\bf BitsPerComponent}.

\subsubsubsection{Masked images}

Generally, in the opaque image model (see below on transparency), an image takes a rectangular area on the
page.
Every pixel in the rectangular area is painted according to the image.
But in PDF, an image may specify a {\bf Mask}, which is itself an image which determines which samples are
drawn (unmasked) and which are not (masked) or an array determining which colors in the image to show.
If {\bf Mask} specifies an image, it must be an {\it image mask}; an image whose {\bf ImageMask} is {\tt true}.

An image mask is an image where every sample is a bit ({\bf BitsPerComponent} is $1$), and {\bf ColorSpace}
must be left unspecified (since the samples don't represent colors).
A ``black'' bit is unmasked; the underlying image will ``bleed'' through, changing the color of the
area of the sample.
A ``white'' bit is masked; the underlying image will not change the color of the area of the sample.

The {\bf Decode} array of an image mask must be either $[0\ 1]$; in which case $0$ is a ``black'' bit and
$1$ a ``white'' bit.
Or it may be $[1\ 0]$ in which case $0$ is ``white'' and $1$ is ``black''.

An image mask may not have the same {\bf Width} and {\bf Height} as the image it is masking.
This is okay, since images are painted in a $1\times1$ square, so their areas will line up on the page
regardless.

\subsubsubsection{Color key masking}

Instead of an image mask being specified for {\bf Mask}, instead an array of $2n$ integers may be specified,
where $n$ is the number of color components in the {\bf ColorSpace} of the image.
Let this array be $[m_1\ M_1\ \dots\ m_n\ M_n]$ where $m_i,M_i\in[0,2^b-1]$ ($b$ is {\bf BitsPerComponent}).
A sample of value $c_1\ \dots\ c_n$ (before being decoded) is masked (not painted) if all of its components
lie in $[m_i,M_i]$, i.e. it is not painted if $m_i\leq c_i\leq M_i$ for all $i\in[1,n]$.

For example, we may have the following image:

\blisting
1 0 obj
<<
    /Type /XObject
    /Subtype /Image
    /Width 9
    /Height 9
    /Filter /ASCIIHexDecode
    /ColorSpace /DeviceRGB
    /BitsPerComponent 8
    /Mask [255 255 255 255 255 255]
    /Length 570
>>
stream
    ff0000 ff0000 ff0000 ff0000 ff0000 ff0000 ff0000 ff0000 ff0000 
    ff0000 ffffff ffffff ffffff ffffff ffffff ffffff ffffff ff0000 
    ff0000 ffffff 00ff00 00ff00 ffffff 00ff00 00ff00 ffffff ff0000 
    ff0000 ffffff 00ff00 00ff00 ffffff 00ff00 00ff00 ffffff ff0000 
    ff0000 ffffff ffffff ffffff ffffff ffffff ffffff ffffff ff0000 
    ff0000 ffffff 0000ff ffffff ffffff ffffff 0000ff ffffff ff0000 
    ff0000 ffffff ffffff 0000ff 0000ff 0000ff ffffff ffffff ff0000 
    ff0000 ffffff ffffff ffffff ffffff ffffff ffffff ffffff ff0000 
    ff0000 ff0000 ff0000 ff0000 ff0000 ff0000 ff0000 ff0000 ff0000 >
endstream
endobj
\elisting

The image is painted with all white pixels masked out.
We can then paint it in the following XObject, with a black background:

\blisting
2 0 obj
<<
    /Type /XObject
    /Subtype /Form
    /BBox [0 0 100 100]
    /FormType 1
    /Matrix [1 0 0 1 0 0]
    /Resources 9 0 R
    /Length 52        
>>
stream
0 0 100 100 re f
q 100 0 0 100 0 0 cm /SIm Do Q
endstream
endobj
\elisting

This results in

\bgroup
\immediate\pdfobj stream attr{
    /Type /XObject
    /Subtype /Image
    /Width 9
    /Height 9
    /Filter /ASCIIHexDecode
    /ColorSpace /DeviceRGB
    /BitsPerComponent 8
    /Mask [255 255 255 255 255 255]
}{
    ff0000 ff0000 ff0000 ff0000 ff0000 ff0000 ff0000 ff0000 ff0000 
    ff0000 ffffff ffffff ffffff ffffff ffffff ffffff ffffff ff0000 
    ff0000 ffffff 00ff00 00ff00 ffffff 00ff00 00ff00 ffffff ff0000 
    ff0000 ffffff 00ff00 00ff00 ffffff 00ff00 00ff00 ffffff ff0000 
    ff0000 ffffff ffffff ffffff ffffff ffffff ffffff ffffff ff0000 
    ff0000 ffffff 0000ff ffffff ffffff ffffff 0000ff ffffff ff0000 
    ff0000 ffffff ffffff 0000ff 0000ff 0000ff ffffff ffffff ff0000 
    ff0000 ffffff ffffff ffffff ffffff ffffff ffffff ffffff ff0000 
    ff0000 ff0000 ff0000 ff0000 ff0000 ff0000 ff0000 ff0000 ff0000 >
}

\setbox0=\hbox to100bp{\rlap{\vrule width100bp height100bp depth0bp}%
\pdfliteral{q 100 0 0 100 0 0 cm /SIm Do Q}\kern100bp}
\pdfxform resources{
    /XObject << /SIm \the\pdflastobj\space0 R >>
}0

\centerline{\pdfrefxform\pdflastxform}
\egroup

\subsubsection{Inline images}

As opposed to an image XObject, which is defined as an external (and reusable) object, an inline image may be
defined entirely within the content stream in which it is used.
An inline image is declared in a content stream in the following way:

\blisting
BI  % begin inline image object
... key-value pairs ...
ID  % begin image data
... image data ...
EI  % end inline image object
\elisting

\noindent That is, the {\bf BI} operator begins the inline image dictionary (except its just a stream of
key-value pairs, not wrapped in \inlinecode|<<...>>|), {\bf ID} ends it and begins the image data (which
specifies how to color each pixel), and {\bf EI} ends the inline image object.
The fields that can placed in between {\bf BI} and {\bf ID} are a subset of the fields which may appear in
an image XObject dictionary:

\bdicttable
BitsPerComponent&integer&(Required) the number of bits used to represent each color component (a single value
in a color value).
Valid values are $1,2,4,8,16$.\cr
Width&integer&(Required) the width of the image, in samples.\cr
Height&integer&(Required) the height of the image, in samples.\cr
ColorSpace&name&(Required) the color space that the image is defined in.\cr
Filter&array&(Required) an array of filters to be applied to the image data to decode it (like any stream).
\edicttable

We can use the following abbreviations for the fields (only in inline images, not image XObjects):
\blist
    \item {\bf BitsPerComponent} --- {\bf BPC}
    \item {\bf Width} --- {\bf W}
    \item {\bf Height} --- {\bf H}
    \item {\bf ColorSpace} --- {\bf CS}
    \item {\bf Filter} --- {\bf F}
\elist

So if {\bf CS} requires $n$ values per color (e.g. $3$ for RGB), {\bf W} is $w$, and {\bf H} is $h$,
the image data should have $n\cdot w\cdot h$ samples.
Each sample must have a size of {\bf BPC} after being decoded by the filter.
The only filter we will discuss is {\bf AHx} (or {\bf ASCIIHexDecode}), which decodes ASCII hex codes.
That is, it takes two ASCII characters (which are either $0-9$ or $a-z$ or $A-Z$), and computes the hex value.
Whitespace is ignored.

\bnote
The stream must be ended by a {\tt>} character, as seen below.
\eppbox

For example, we may have the inline image:

\blisting
q
.996264 0 0 .996264 0 0 cm
100 0 0 100 0 0 cm
BI
    /BPC 8
    /CS /DeviceRGB
    /F [ /AHx ]
    /W 9
    /H 9
ID
    ff0000 ff0000 ff0000 ff0000 ff0000 ff0000 ff0000 ff0000 ff0000 
    ff0000 ffffff ffffff ffffff ffffff ffffff ffffff ffffff ff0000 
    ff0000 ffffff 00ff00 00ff00 ffffff 00ff00 00ff00 ffffff ff0000 
    ff0000 ffffff 00ff00 00ff00 ffffff 00ff00 00ff00 ffffff ff0000 
    ff0000 ffffff ffffff ffffff ffffff ffffff ffffff ffffff ff0000 
    ff0000 ffffff 0000ff ffffff ffffff ffffff 0000ff ffffff ff0000 
    ff0000 ffffff ffffff 0000ff 0000ff 0000ff ffffff ffffff ff0000 
    ff0000 ffffff ffffff ffffff ffffff ffffff ffffff ffffff ff0000 
    ff0000 ff0000 ff0000 ff0000 ff0000 ff0000 ff0000 ff0000 ff0000 >
EI
Q
\elisting

Which draws the following:

\centerline{\hbox to100pt{\vrule width\z@ height100pt depth\z@%
\pdfliteral{
q
.996264 0 0 .996264 0 0 cm
100 0 0 100 0 0 cm
BI
    /BPC 8          % each value is a byte
    /CS /DeviceRGB  % interpret colors as RGB
    /F [ /AHx ]     % the stream is given in ASCII representing hexadecimal values
    /W 9            % each line is given by 9 samples
    /H 9            % 9 lines
ID
    ff0000 ff0000 ff0000 ff0000 ff0000 ff0000 ff0000 ff0000 ff0000 
    ff0000 ffffff ffffff ffffff ffffff ffffff ffffff ffffff ff0000 
    ff0000 ffffff 00ff00 00ff00 ffffff 00ff00 00ff00 ffffff ff0000 
    ff0000 ffffff 00ff00 00ff00 ffffff 00ff00 00ff00 ffffff ff0000 
    ff0000 ffffff ffffff ffffff ffffff ffffff ffffff ffffff ff0000 
    ff0000 ffffff 0000ff ffffff ffffff ffffff 0000ff ffffff ff0000 
    ff0000 ffffff ffffff 0000ff 0000ff 0000ff ffffff ffffff ff0000 
    ff0000 ffffff ffffff ffffff ffffff ffffff ffffff ffffff ff0000 
    ff0000 ff0000 ff0000 ff0000 ff0000 ff0000 ff0000 ff0000 ff0000 >
EI
Q
}\hfil}}

\bigskip

Recall that an image is by default one unit by one unit, so we must scale it to get the desired size.

\subsection{Form XObjects}

A form XObject is a self-contained content stream which can be painted within other content streams.
The graphics defined in a form XObject are defined relative to its {\it form coordinate space}, which can
then be scaled to user space using its {\bf Matrix} field.

Form XObjects may have the following fields in their stream dictionary, as well as fields common to all
content streams:

\bdicttable
Type&name&(Optional) the type of PDF object the dictionary describes; must be {\bf XObject}.\cr
Subtype&name&(Required) the type of XObject the dictionary describes; must be {\bf Form}.\cr
BBox&array&(Required) a bounding box which defines the boundary of the XObject in form space.
The XObject is clipped to this boundary.\cr
Matrix&array&(Optional) a matrix which represents the transformation from form to user space.\cr
Resources&dictionary&(Required) a dictionary specifying the resources required by the form XObject.\cr
Group&dictionary&(Optional) a {\it group attributes dictionary} indicating that the contents of the form
XObject are to be treated as a groip and specifying the attributes of that group (see below).
\edicttable

\subsubsection{Group XObjects}

A {\it group XObject} is a type of form XObject that can be used to group graphical elements together for
various purposes.
It is declared by adding the {\bf Group} field to a form XObject's dictionary.
This is a dictionary called the {\it group attributes dictionary}, which may have the following fields:

\bdicttable
Type&name&(Optional) the type of PDF object that this dictionary describes; must be {\bf Group}.\cr
S&name&(Required) the {\it group subtype}, which identifies the kind of group the dictionary describes.
The only available group subtype is {\bf Transparency}.
Transparency group XObjects are discussed later.\cr
\edicttable

