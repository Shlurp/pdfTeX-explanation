\subsection{Datatypes}

PDF supports the following datatypes:
\blist
    \item booleans;
    \item integers;
    \item real numbers;
    \item strings;
    \item names;
    \item arrays;
    \item dictionaries;
    \item streams;
    \item the null object.
\elist

\subsubsection{Booleans}

Booleans are determined by the keywords {\tt true} and {\tt false}.
Integers are written with or without a sign, and a decimal point turns a number into a real number.

\subsubsection{Strings}

There are two ways to write a string: enclosing characters in parentheses, or, as hexadecimal data wrapped in
single angle brackets ({\tt<...>}).
So for example, {\tt(hello world)} represents the string ``hello world''.
So does {\tt<68656C6C6F776F726C64>}.
To add special characters (line feed, unbalanched parentheses, etc) you can add a backslash before (as one
would normally).

\subsubsection{Names}

A name begins with a forward slash {\tt/}, and may contain any characters except for whitespace and delimiter
characters (brackets, parentheses and friends, forward slash, or a percent sign).
You can add any non-null character to a name (including special characters) by preceding its hexadecimal code
with \inlinecode|#|.
So for example, the following are names:

\centerline{\inlinecode|/name|, \inlinecode|/name*with_special&characters|, \inlinecode|/#28parentheses#29|}

\subsubsection{Arrays}

Arrays are one-dimensional array objects, written in square brackets.
They are delimited by whitespace, and can include any other object as a member.
For example, the following is an array:

\centerline{\inlinecode|[1 1.2 true (S. Lurp) /A_Name [true false]]|}
As displayed in the above example, an array can contain an array as well, thus allowing multi-dimensional
arrays.

\subsubsection{Dictionaries}

A dictionary is a mapping between name objects and instances of any datatype.
A dictionary is enclosed in double angle brackets, like so:

\blisting
<<
    /Type /A_Dictionary
    /Subtype /Example
    /IntItem 12
    /NumItem 1.2
    /StringItem (hello world)
    /ArrItem [12 1.2 (hello world)]
    /DictItem <<
        /Item1 true
        /Item2 false
    >>
>>
\elisting

Dictionaries are of extreme importantce in PDFs.
They will show up a lot later.

\subsubsection{Streams}

A stream is similar to a string, except for a few key differences:
\benum
    \item a string must be read in its entirety, while a stream may be read incrementally;
    \item a string has a maximum limit based on implementation, while a stream has no such limit (which is
        why larger data like images or pages are stored in streams).
\eenum

A stream is structured as follows:

\blisting
<< dictionary >>
stream
...
endstream
\elisting

A stream must be an \gotoanchor{indirect-objects}{{\it indirect object}}.
The stream's dictionary which precedes the \inlinecode|stream...endstream| must have a \inlinecode|/Length|
field which is equal to the byte length of the stream's content ({\tt...}).

\subsubsection{Null}

The null object is an object referencable by the keyword {\tt null}.
It is not equal to any other object.
An \gotoanchor{indirect-objects}{indirect object} which references a non-existent object is equivalent to the
null object.
And giving a dictionary entry the value {\tt null} is equivalent to omitting the entry.

\subsubsection{Indirect Objects} \anchor{indirect-objects}

Any object can be labeled as an {\it indirect object}.
This allows other objects to reference it via a unique {\it object identifier}.
The object identifier has two parts:
\benum
    \item a unique positive integer called the {\it object number};
    \item a (not necessarily unique) non-negative integer called the {\it generation number}.
        For our purposes, these are always zero.
\eenum

An indirect object is declared using the {\tt obj} postfix operator.
Preceding it is the object identifier.
The object's value itself is written between the {\tt obj} and {\tt endobj} keywords.
For example:

\blisting
12 0 obj
    [1 2 3]
endobj
\elisting

Creates an array indirect object of value {\tt[1 2 3]}.
Its object number is 12, and its generation number is 0.

To reference this object, simply use the {\it indirect reference} \inlinecode|12 0 R|.

For example, to create an indirect stream object, you could do:

\blisting
7 0 obj
    << /Length 8 0 R >>
stream
    BT
        /F1 12 Tf
        72 712 Td
        (A stream with an indirect length) Tj
    ET
endstream
endobj

8 0 obj
    77
endobj
\elisting

This defines object {\tt7 0 } to be a stream object containing the above contents.
Its length is an indirect reference to object {\tt8 0}, whose value is $77$.

\subsection{Trees}

A tree is a composite datatype made up of other datatypes.
Its purpose is similar to a dictionary: it maps keys to values, but by different means.
They vary in some ways:
\benum
    \item the keys in a tree are either strings (name trees) or numbers (number trees);
    \item the keys are ordered;
    \item the values associated with the keys may be of any type (including {\tt null});
    \item a tree can represent an arbitrarily large map, and can be read in parts, unlike a dictionary.
\eenum

There are two types of trees: name trees and number trees.
The difference between them is the datatype of their keys: name trees use strings as keys while number trees
use numbers.

Every tree is constructed from nodes, which are dictionary objects.
There are three kinds of nodes: a root node, intermediate nodes, and leaf nodes.
The meaning of these are self-explanatory.

\subsubsection{Number trees}

A number tree node is a dictionary with the following fields:

\bdicttable
Kids&array&(Root and intermediate nodes only; present in root only if {\bf Names} isn't)
an array of indiret references to the children of this node (either intermediate or leaf nodes).\cr
Names&array&(Root and leaf nodes; present in root only if {\bf Kids} isn't)
an array of the form
$$ [{\it key}_1\ {\it value}_1\ \dots\ {\it key}_n\ {\it value}_n] $$
where each ${\it key}_i$ is a string, and the ${\it value}_i$ is the associated value.
The keys are sorted by lexical value as explained below.\cr
Limits&array&(Intermediate and leaf nodes only) an array of two strings, specifying the lexically least and
greatest keys included in the {\bf Names} array of a leaf node, or in the case of an intermediate node, the
{\bf Names} of the children nodes.\cr
\edicttable

For example, we may have the following tree which describes the grades of students, say:
\blisting
1 0 obj     % root
<<
    /Kids
      [ 2 0 R
        3 0 R ]
>>
endobj

2 0 obj     % intermediate
<<
    /Limits [(Andrew) (Gordon)]
    /Kids
      [ 4 0 R
        5 0 R
        6 0 R ]
>> 
endobj

3 0 obj     % intermediate
<<
    /Limits [(Howard) (Zack)]
    /Kids
      [ 7 0 R
        8 0 R ]
>>
endobj

4 0 obj     % leaf
<<
    /Limits [(Andrew) (Avery)]
    /Names
      [ (Andrew) 100
        (Avery) 80 ]
>>
endobj

5 0 obj     % leaf
<<
    /Limits [(Bob) (Dylan)]
    /Names
      [ (Bob) 90
        (Chris) 100
        (Drew) 100
        (Dylan) 60 ]
>>
endobj

6 0 obj     % leaf
<<
    /Limits [(Fred) (Gordon)]
    /Names
      [ (Fred) 50
        (Gordon) 85 ]
>>
endobj

7 0 obj     % leaf
<<
    /Limits [(Howard) (Howard)]
    /Names
      [ (Howard) 10 ]
>>
endobj

7 0 obj     % leaf
<<
    /Limits [(Zack) (Zack)]
    /Names
      [ (Zack) 70 ]
>>
endobj
\elisting

\subsubsection{Number trees}

A number tree is similar to a name tree except that its keys are integers instead of strings, and are
sorted in ascending numerical order.
And instead of the key-value array being named {\bf Names}, it is named {\bf Nums}.

\subsection{Functions}

An important quote from the PDF reference:

\rightline{``{\setfont{it}PDF is not a programming language, and a PDF file is not a program.}''}

Despite this, PDF provides the ability to define certain kinds of functions.
Though of course their use is limited and restricted.

All PDF functions are pure functions ${\bb R}^m\to{\bb R}^n$ (pure meaning they have no side-effects).
Importantly, their inputs and outputs must be numbers, not just any PDF datatype.
PDF functions must have a domain defined in their definition.
If a function has a domain of, say, $[-1,1]$ and is called with input $6$, the input will be clipped to the
domain; so the function will be called with input $1$.
Similarly some functions may define a range, and the output may be similarly clipped.

A function may be either a dictionary or a stream.
A {\it function dictionary} refers either directly to the function (if it is a dictionary) or to the stream
dictionary (if it is a stream).
The dictionary must provide a {\bf FunctionType} entry, which is one of $0,2,3,4$.
For a function of type ${\bb R}^m\to{\bb R}^n$, the function dictionary may have the following fields
(in addition to fields specific to the function type).

\bdicttable
FunctionType&integer&(Required) the function type.\cr
Domain&array&(Required) an array of $2m$ numbers.
For $0\leq i\leq m-1$, ${\bf Domain}_{2i}$ must be less than or equal to ${\bf Domain}_{2i+1}$.
The domain of the function is
$$ \prod_{0\leq i\leq m-1}[{\bf Domain}_{2i},{\bf Domain}_{2i+1}] $$\cr
Range&array&(Required for type 0 and 4 functions, otherwise required) an array of $2n$ numbers.
Similar to the domain, for every $0\leq i\leq n-1$, ${\bf Range}_{2i}\leq{\bf Range}_{2i+1}$.
The range (codomain) of the function is
$$ \prod_{0\leq i\leq n-1}[{\bf Range}_{2i},{\bf Range}_{2i+1}] $$\cr
\edicttable

\subsubsection{Type 0 (Sampled) Functions}

Type 0 functions use a sequence of sampled values (which are contained in a stream) to approximate a function
whose domain and range are both bounded.
In addition to the fields already listed, the function dictionary of a type 0 function may include the
following fields as well:

\bdicttable
Size&array&(Required) an array of $m$ positive integers which specifies the number of samples in each
input dimension.\cr
BitsPerSample&integer&(Required) the number of bits used to represent each sample.
Valid values are $1,2,4,8,16,24,32$.\cr
Order&integer&(Optional) the order of interpolation between samples.
Valid values are $1$ and $3$, specifying linear and cubic spline interpolation, respectively.
The default value is $1$.\cr
Encode&array&(Optional) an array of $2m$ numbers specifying a linear mapping of input values into the domain
of the function's sample table.
The default value is $[0\ ({\bf Size}_0-1)\ 0\ ({\bf Size}_1-1)\ \dots]$.\cr
Decode&array&(Optional) an array of $2n$ numbers specifying a linear mapping of sample values into the range
appropriate appropriate for the function's output values.
\edicttable

\noindent The dictionary may include other fields common to stream objects.
Given an input dimension of $m$, we must have $\prod_{0\leq i<m}{\bf Size}_i$ values in the stream.
In order, these give the multi-dimensional array
$$ g(0,\dots,0),g(1,\dots,0),\dots,g({\bf Size}_0-1,0,\dots,0),g({\bf Size}_0-1,1,\dots,0),\dots,
g({\bf Size}_0-1,\dots,{\bf Size}_{m-1}-1) $$
We now describe how to use $g$ to compute $f$, the type 0 function.

To explain how the function is calculated, we first define the following function:
$$ \interpolate(x;x_0,x_1,y_0,y_1) = y_0 + \left((x-x_0)\cdot\frac{y_1-y_0}{x_1-x_0}\right) $$
this simply projects $x$ onto the line between $(x_0,y_0)$ and $(x_1,y_1)$.

When a sampled function is called with input values $(x_0,\dots,x_{m-1})$, the following steps are taken in
order to compute the result:
\benum
    \item Each $x_i$ is clipped to the domain:
        $$ x'_i = \min(\max(x_i,\dom_{2i}),\dom_{2i+1}) $$
    \item The input value is then encoded:
        $$ e_i = \interpolate(x'_i;\dom_{2i},\dom_{2i+1},{\bf Encode}_{2i},{\bf Encode}_{2i+1}) $$
        That is, given an input $x$, we project it onto the line whose endpoints are
        $(\dom_{2i},{\bf Encode}_{2i})$ and $(\dom_{2i+1},{\bf Encode}_{2i+1})$.
        The effect is that the lower end of the domain is mapped to ${\bf Encode}_{2i}$ and the higher end
        is mapped to ${\bf Encode}_{2i+1}$.
    \item Then the input value is clipped
        $$ e'_i = \min(\max(e_i,0),{\bf Size}_i-1) $$
        This gives us a real matrix $(e'_m,\dots,e'_{m-1})$ which is the encoding of the input vector.
    \item We can then use interpolation (of order ${\bf Order}$) to compute $g(e')=r$.
    \item We interpolate $r$ in order to decode it:
        $$ r'_j = \interpolate(r_j;0,2^{\bf BitsPerSample}-1,{\bf Decode}_{2j},{\bf Decode}_{2j+1}) $$
    \item And then we clip the output to the range:
        $$ y_j = \min(\max(r'_j,\rng_{2j}),\rng_{2j+1}) $$
        This gives us the output: $f(x)=y$.
\eenum

So for example, suppose we have a sampled function ${\bb R}^2\to{\bb R}$ whose domain is $[-1,1]^2$.
The sampling contains $21$ columns and $31$ rows.
So we must encode the input to $[0,20]\times[0,30]$, and then decode to $[-1,1]$.
So the code will be

\blisting
14 0 obj
<<
    /FunctionType 0
    /Domain [-1.0 1.0 -1.0 1.0]
    /Size [21 31]
    /Encode [0 20 0 30]
    /BitsPerSample 4
    /Range [-1.0 1.0]
    /Decode [-1.0 1.0]
    /Length ...
>>
stream
... 651 sampled values ...
endstream
endobj
\elisting

\subsubsection{Type 2 (Exponential Interpolation) Functions}

Type 2 functions are functions ${\bb R}\to{\bb R}^n$.
Exponential interpolation is given by the following parameters $c_0,c_1\in{\bb R}^n$ and $N\in{\bb R}$.
The value of the function at $x$ is
$$ f(x;c_0,c_1,n) = c_0 + (c_1-c_0)x^N $$
The interpretation of the parameters is as follows: $c_0,c_1$ are the values of $f$ at $x=0$ and $x=1$
respectively.
$N$ is the interpolation exponent, which dictates how the curve behaves.
When $N=1$ this is simply linear interpolation.

A type 2 function dictionary may include the following additional fields:

\bdicttable
C0&array&(Optional) an array of $n$ numbers, defining the parameter $c_0$.
Its default value is $[0.0]$.\cr
C1&array&(Optional) an array of $n$ numbers, defining the parameter $c_1$.
Its default value is $[1.0]$.\cr
N&number&(Required) the interpolation exponent.
\edicttable

The values of $\dom$ must constrain $x$ such that if $N$ is not an integer, $x\geq0$.
And if $N$ is negative $x\neq0$.

\subsubsection{Type 3 (Stitching) Functions}

Type 3 functions define a stitching of $k$ one-input functions together.
Suppose you're given a sequence of functions $f_0,\dots,f_{k-1}\colon{\bb R}\to{\bb R}^n$ with domains
$[d_{00},d_{01}],\dots,[d_{k-1,0},d_{k-1,1}]$.
Now we define another vector of {\it bounds}, ${\bf Bounds}=[b_0,\dots,b_{k-2}]$, so that if
$b_{i-1}\leq x<b_i$ we want to input $x$ into $f_i$ ($b_{-1}$ and $b_{k-1}$ are the endpoints of the domain
specified for $f$).
But $[b_i,b_{i+1}]$ may not align with the domain of $f_i$ ($[d_{i0},d_{i1}]$), and we don't necessarily want
it to anyway.
So we now we want to linearly interpolate $[b_i,b_{i+1}]$ into $[d_{i0},d_{i1}]$.
We do so by specifying two points $e_{i0},e_{i1}\in[d_{i0},d_{i1}]$ which will be the new endpoints.

Explicitly, we have
\benum
    \item a new domain $[d_0,d_1]$;
    \item $k$ one-input functions ${\bf Functions}=[f_0,\dots,f_{k-1}]$ each with a domain $[d_{i0},d_{i1}]$;
    \item a vector of bounds ${\bf Bounds}=[b_0,\dots,b_{k-2}]$ (and define $b_{-1}=d_0$ and $b_{k-1}=d_1$);
    \item a vector of encoding values ${\bf Encode}=[e_{00},e_{01},\dots,e_{k-1,0},e_{k-1,1}]$;
\eenum
and given an input value $x$, if $b_{i-1}\leq x<b_i$ (the right inequality is weak if $i=k-1$), then we
compute $x'=\interpolate(x;b_{i-1},b_i,e_{i0},e_{i1})$.
Then we output $f_i(x')$.

Of course, the ranges of each $f_i$ must be compatible with the range specified for $f$ (if specified).

\bdicttable
Functions&array&(Required) the array of $k$ one-input functions to be stitched together.
Each function must have the same output dimensionality $n$.\cr
Bounds&array&(Required) the array of $k-1$ numbers determining the bounds for which to map into each function.%
\cr
Encode&array&(Required) the array of $2k$ numbers which determines the mapping of bounds into domains of each
function.
\edicttable

Notice that if we have a function $f\colon{\bb R}\to{\bb R}^n$, and we want to compute $g(x)=f(1-x)$.
We can compute this by defining $g$ as a stitching function, where the ${\bf Encode}$ array is $[1\ 0]$.

\subsubsection{Type 4 (PostScript Calculator) Functions}

A type 4 function utilizes a small subset of PostScript code to compute values.
The following PostScript operators can be used in type 4 functions:

\blist
    \item Arithemtic operators: {\bf abs, cvi, floor, mod, sin, add, cvr, idiv, mul, sqrt, atan, div, ln,
        neg, sub, ceiling, exp, log, round, truncate, cos}
    \item Boolean and bitwise operators: {\bf and, false, le, not, true, bitshift, ge, lt, or, xor, eq, gt, ne}
    \item Conditional operators: {\bf if, ifelse}
    \item Stack operators: {\bf copy, exch, pop, dup, index, roll}
\elist

The operand syntax for type 4 functions follows PDF conventions rather than PostScript ones.
The entire code defining the function must be wrapped in curly braces \inlinecode|{...}|.
Braces are also used to delimit expressions executed conditionally in {\bf if} and {\bf ifelse} operators.

I may later update this document to provide information on how to use PostScript operators (specifically in
PDFs).

