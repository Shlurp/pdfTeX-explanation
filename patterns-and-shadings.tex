This section should probably belong to the ``Graphics'' section; but it's a big enough of a concept that I
(and hopefully you) will let letting it have its own section slide.
A pattern is a repeating graphical figure or smooth gradient used to fill regions of space.
Take for example the two patterns below:

\immediate\pdfobj stream attr{
    /Type /Pattern
    /PatternType 1
    /PaintType 1
    /TilingType 1
    /BBox [0 0 20 20] 
    /XStep 20
    /YStep 20
    /Resources << >>
}{
    q
    1 0 0 rg 0 0 10 10 re 10 10 10 10 re f
    0 0 .5 rg 10 0 10 10 re 0 10 10 10 re f
    Q
}

\bigskip
\bgroup\setbox0=\hbox to150pt{\vrule width\z@ height100pt depth\z@%
\pdfliteral{%
    q
    \nattrans,
    .5 0 0 .5 0 0 cm
    1 j 1 J
    /CSP cs
    /P scn
    0 100 m
    25 150 100 200 150 200 c
    200 150 250 150 300 100 c
    275 50 250 0 y
    200 0 175 0 150 50 c
    125 50 100 50 0 100 c
    h
    B
    Q
}\hfil}
\pdfxform resources{%
    /ColorSpace << /CSP [ /Pattern /DeviceRGB ] >>
    /Pattern << /P \the\pdflastobj\space 0 R >>
}0
\edef\A{\the\pdflastxform}
\setbox0=\hbox to150pt{\vrule width\z@ height100pt depth\z@%
\pdfliteral{%
    q
    \nattrans,
    .5 0 0 .5 0 0 cm
    1 j 1 J
    /Pattern cs
    /P scn
    0 100 m
    25 150 100 200 150 200 c
    200 150 250 150 300 100 c
    275 50 250 0 y
    200 0 175 0 150 50 c
    125 50 100 50 0 100 c
    h
    B
    Q
}\hfil}%
\pdfxform resources{%
    /Pattern << /P <<
        /Type /Pattern
        /PatternType 2
        /Shading <<
            /ShadingType 2 /ColorSpace /DeviceRGB /Domain [0 1] /Coords [0 0 150 0]
            /Function << /FunctionType 2 /Domain [0 1] /C0 [1 0 0] /C1 [0 0 .5] /N 1 >>
            /Extend [false false]
        >>
    >> >>
}0
\centerline{\pdfrefxform\A\hfil\pdfrefxform\pdflastxform}\egroup
\bigskip

There are two types of patterns:
\benum
    \item {\it Tiling patterns} which repeat a small graphical figure (a {\it pattern cell}) over fixed
    horizontal and vertical intervals to fill the area to be painted.
    (The above left example is one of a tiling pattern.)
    \item {\it Shading patterns} form a gradient that smoothly transitions between colors and fills the area
    to be painted.
    (The above right example is one of a shading pattern.)
\eenum

Every pattern has a {\it pattern matrix}, which is a transformation matrix which maps the pattern's internal
coordinate system to the default coordinate system of the pattern's parent content stream (the content stream
from which the pattern was called).
The multiplication of the pattern matrix with the transformation matrix of the parent content stream
establishes the {\it pattern coordinate space}.
Importantly this is the multiplication of the transformation matrix of the parent stream, and not the CTM.
Thus after a pattern is defined in the resources of a content stream, its pattern coordinate space is not
affected by changes to the CTM.

\subsection{Tiling patterns}

A tiling pattern takes a small graphical figure, called the {\it pattern cell}, and repetitively paints it
horizontally and vertically.
The pattern cell is a content stream which may contain anything a content stream can contain (paths, images,
XObjects, etc.).
When painting using a tiling pattern, the pattern cell is tiled as many times as needed to fill the region
specified.

A pattern cell is is defined by a content stream, and thus has a stream dictionary.
Along with fields common to all stream dictionaries, a pattern dictionary may have the following fields:

\bgroup
\expandafter\def\csname enumstyle1\endcsname#1{#1}
\bdicttable
Type&name&(Optional) the type of PDF object the dictionary describes; must be {\bf Pattern}.\cr
PatternType&integer&(Required) a code identifying what kind of pattern this dictionary describes.
Must be $1$ for a tiling pattern.\cr
PaintType&integer&(Required) a code that determines how the color of the pattern cell is to be specified:
\benum
    \item {\it Colored tiling pattern}: the pattern's content stream specifies the colors to be used to paint
    the pattern cell.
    When the pattern cell's content stream begins execution, the current color is the one that was in effect
    in the parent content stream.
    \item {\it Uncolored tiling pattern}: the pattern's\hfil\break
    content stream does not specify any color information.
    Instead, the entire pattern cell is painted a separately specified color each time the pattern is used.
\eenum\cr
TilingType&integer&(Required) a code that controls adjustments to the spacing of tiles relative to the
device pixel grid:
\benum
    \item {\it Constant spacing}: pattern cells are spaced consistently, i.e. by a miltiple of a device pixel.
    To achieve this, the application is allowed to distort the pattern cell slightly by making small
    adjustments to {\bf XStep}, {\bf YStep}, and the transformation matrix.
    This distortion does not exceed one device pixel.
    \item {\it No distortion}: the pattern cell is not distorted, but the spacing between the pattern cells
    may vary by as much as one device pixel.
    \item {\it Constant spacing and faster tiling}: pattern cells are spaced consistently as with {\it constant
    spacing}, but additional distortion is permitted to enable faster tiling.
\eenum\cr
BBox&array&(Required) an array of four numbers specifying the bottom left and top right corners of the bounding
box of the pattern cell.
That is, {\bf BBox} is of the form $[x_0\ y_0\ x_1\ y_1]$ where $(x_0,y_0)$ is the bottom left corner of the
pattern cell (relative to its own coordinates), and $(x_1,y_1)$ is the top right corner.\cr
XStep&number&(Required) the desired horizontal spacing between pattern cells, measured in the pattern
coordinate system.\cr
YStep&number&(Required) the desired vertical spacing between pattern cells, measured in the pattern
coordinate system.\cr
Resources&dictionary&(Required) a resource dictionary containing all the named resources used by the pattern's
content stream.\cr
Matrix&array&(Optional) an array of six numbers representing the pattern matrix.
\edicttable
\egroup

We use the following operators to paint with patterns:

\bthreetable{\it}{\bf}{}
{\bf Operands}&{\bf Operator}&{\bf Description}\cr
$c_1\dots c_n\ {\it name}$&SCN&Sets the current pattern to be stroked, if the current color space is
{\bf Pattern}, or if {\it name} refers to a shading pattern.
{\it name} must refer to a pattern defined in the {\bf Pattern} subdictionary of the current resource
dictionary.
If this is an uncolored tiling pattern ({\bf PatternType} of $1$ and {\bf PaintType} of $2$), $c_1,\dots,c_n$
are parameters in the specified underlying color space (see below) which set the color to use for the tiling.
\cr
$c_1\dots c_n\ {\it name}$&scn&Same as {\bf SCN} but for non-stroking operations.
\ethreetable

For example, to draw the colored tiling pattern at the beginning of the section, the following PDF code was
produced:

\noindent First we have the pattern object, which defines the pattern cell and necessary data for tiling:

\blisting
1 0 obj
<<
    /Type /Pattern      % pattern dictionary
    /PatternType 1      % tiling pattern
    /PaintType 1        % colored tiling pattern
    /TilingType 1       % constant spacing
    /BBox [0 0 20 20]   % bounding box
    /XStep 20           % xstep and ystep are precisely
    /YStep 20           % the size of the pattern cell
    /Resources << >>    % no resources
    /Length 84
>>
stream
q
1 0 0 rg                        % set color to red
0 0 10 10 re 10 10 10 10 re f   % draw the two red rectangles
0 0 .5 rg                       % set color to blue
10 0 10 10 re 0 10 10 10 re f   % draw the two blue rectangles
Q 
endstream
endobj
\elisting

\noindent Then we have the content stream (which here is an XObject) in which the pattern is painted.

\blisting
2 0 obj
<<
/Type /XObject              % fields for the XObject...
/Subtype /Form
/BBox [0 0 149.44 99.626]
/FormType 1
/Matrix [1 0 0 1 0 0]
/Resources 9 0 R            % resources are in object 9 0
/Length 193       
>>
strea
q
.996264 0 0 .996264 0 0 cm  % this will be explained later
.5 0 0 .5 0 0 cm            % scale down by half
1 j 1 J                     % rounded lines
/Pattern cs                 % set current non-stroking color space to be a pattern
/P scn                      % set the current non-stroking pattern
0 100 m                     % draw the region to paint...
25 150 100 200 150 200 c
200 150 250 150 300 100 c
275 50 250 0 y
200 0 175 0 150 50 c
125 50 100 50 0 100 c
h
B                           % fill
Q 
endstream
endobj
\elisting

\noindent The resources for this XObject are in object {\tt9 0}.
This object must define the name {\tt/P}.
Indeed it does:

\blisting
% 9 0 obj
<<
    /Pattern << /P 1 0 R >> 
    /ProcSet [ /PDF ]
>>
\elisting

\noindent So {\tt/P} is the pattern defined in object {\tt1 0}.

For a uncolored tiling pattern, we must also specify what color we want to paint it with.
So when we call {\bf CS} or {\bf cs}, we need to tell PDF that we're switching to a {\bf Pattern} color space,
as well as the underlying color space (e.g. {\bf DeviceRGB}).
Then when we call {\bf SCN} or {\bf scn}, we need to tell PDF what color we want to paint the pattern with.
For example, suppose we have the following uncolored pattern:

\bigskip
\immediate\pdfobj stream attr{
    /Type /Pattern
    /PatternType 1
    /PaintType 2
    /TilingType 1
    /BBox [0 0 10 10] 
    /XStep 10
    /YStep 10
    /Resources << >>
}{
    5 10 m
    0 5 l 5 0 l 10 5 l h
    5 7.5 m
    2.5 5 l 5 2.5 l 7.5 5 l h
    f*
}

\bgroup\setbox0=\hbox to150pt{\vrule width\z@ height100pt depth\z@%
\pdfliteral{%
    q
    \nattrans,
    1 j 1 J
    /CSP cs
    1 0 0 /P scn
    0 0 150 100 re
    B
    Q
}\hfil}
\pdfxform resources{%
    /ColorSpace << /CSP [ /Pattern /DeviceRGB ] >>
    /Pattern << /P \the\pdflastobj\space 0 R >>
}0
\centerline{\pdfrefxform\pdflastxform}\egroup
\bigskip

The pattern cell here is simply a hollowed-out diamond.
The PDF code for the pattern cell is as follows:

\blisting
1 0 obj
<<
    /Type /Pattern          % pattern cell
    /PatternType 1          % tiling pattern
    /PaintType 2            % uncolored tiling pattern
    /TilingType 1           % constant spacing
    /BBox [0 0 10 10]       % bounding box
    /XStep 10               % xstep and ystep are precisely
    /YStep 10               % the size of the pattern cell
    /Resources << >>        % no resources
    /Length 66        
>>
stream
5 10 m
0 5 l 5 0 l 10 5 l h        % draw outer diamond
5 7.5 m
2.5 5 l 5 2.5 l 7.5 5 l h   % draw inner diamond
f*                          % fill using even-odd
                            % (so that the diamond is hollowed out)
endstream
endobj
\elisting

\noindent The code which draws the pattern is as follows:

\blisting
2 0 obj
<<
    /Type /XObject
    /Subtype /Form
    /BBox [0 0 149.44 99.626]
    /FormType 1
    /Matrix [1 0 0 1 0 0]
    /Resources 9 0 R
    /Length 78        
>>
stream
q
.996264 0 0 .996264 0 0 cm      % to be explained later
1 j 1 J                         % rounded lines
/CSP cs                         % set color space to /CSP
1 0 0 /P scn                    % set pattern to /P with color 1 0 0
0 0 150 100 re                  % draw rectangle
B                               % and fill
Q 
endstream
endobj
\elisting

\noindent We see that here we set the (non-stroking) color space to a name {\bf CSP}.
This name is defined in the resource dictionary, in object {\tt9 0}:

\blisting
% 9 0 obj
<<
    /ColorSpace << /CSP [ /Pattern /DeviceRGB ] >>
    /Pattern << /P 1 0 R >> 
    /ProcSet [ /PDF ]
>>
\elisting

\noindent Here {\bf CSP} is defined to be a {\bf ColorSpace}.
But instead of being defined as a name like {\bf DeviceRGB}, it is defined to be an array of two names:
{\bf Pattern} and {\bf DeviceRGB}.
This is because since {\bf P} is an uncolored tiling pattern, we must specify the color to color it with.
In object {\tt2 0}, we specified the color code {\tt1 0 0}, but in order to interpret this color code we need
to know what the underlying color space is.
That's why the second color space in the array is {\bf DeviceRGB}, telling us to set the underlying color space
to be {\bf DeviceRGB}.

\subsection{Shading Patterns}

A shading pattern is one which defines a gradient between colors.
It is a pattern for all intents and purposes, and painting it is done in the same way as a normal pattern.
Unlike a tiling pattern, a shading pattern is defined by a dictionary instead of a content stream.
Fields in the pattern dictionary for a shading pattern are as follows:

\bdicttable
Type&name&(Optional) the type of PDF object the dictionary describes; must be {\bf Pattern}.\cr
PatternType&integer&(Required) the code defining the type of pattern this dictionary describes.
Must be $2$ for a shading pattern.\cr
Shading&dictionary or stream&(Required) a shading object (see below) defining the shading pattern gradient.\cr
Matrix&array&(Optional) An array of six numbers defining the pattern matrix.\cr
ExtGState&dictionary&(Optional) a graphics state parameter dictionary to be put into effect when the shading
is painted.
\edicttable

The most important field is {\bf Shading}, which defines the shading object; the object which controls the
look of the gradient.
A shading object may be a content stream or a dictionary, and in either case we refer to the dictionary (either
the object itself or the content stream's dictionary) as the shading dictionary.
The shading dictionary may have the following fields, which are common to all types of shadings (but different
types of shadings will have additional fields):

\bgroup
\expandafter\def\csname enumstyle1\endcsname#1{#1}
\bdicttable
ShadingType&integer&(Required) the shading type (discussed below):
\benum
    \item function-based shading;
    \item axial shading;
    \item radial shading;
    \item free-form Gourand-shaded triangle mesh;
    \item lattice-form Gourand-shaded triangle mesh;
    \item coons patch mesg;
    \item tensor-product patch mesh.
\eenum\cr
ColorSpace&name&(Required) the color space in which the color values are to be interpreted.
This cannot be a {\bf Pattern} space.\cr
Background&array&(Optional) an array containing a color value (appropriate to {\bf ColorSpace}) determining
a background color value.
If it is present, this is used to color any region outside of the region defined by the shading object.\cr
BBox&array&(Optional) an array of the form $[x_0\ y_0\ x_1\ y_1]$ which defines the bottom-left and top-right
coordinates of the bounding box to clip the shading to.
\edicttable
\egroup

We will focus on shading types $1$ through $3$, as $4$ through $7$ are much more complicated.
If you're interested, consult the PDF reference.

\bnote
    The term {\it target coordinate space} means the pattern coordinate space, unless the shading is
    invoked by the shading operator {\bf sh}, in which case it refers to the current user space.
    Recall that the pattern coordinate space is {\it not} affected by changes to the CTM.
\eppbox

\subsubsection{Type 1 (Function-Based) Shadings}

A function-based shading is one where the color at every point on the gradient is defined by a specified
function.
This is the most general form of shading.
If a shading object is specified to be of type 1, it may have the following fields:

\bdicttable
Domain&array&(Optional) an array of four numbers $[x_0\ x_1\ y_0\ y_1]$ specifying the rectangular domain
of coordinates over which the color function(s) are defined.
Default is $[0\ 1\ 0\ 1]$.\cr
Matrix&array&(Optional) an array of six numbers specifying a transformation mapping the coordinate space
specified by {\bf Domain} to the shading's target coordinate space.
For example, to map the default {\bf Domain} to a one-inch square starting at coordinate $(100,100)$, the
{\bf Matrix} would be $[72\ 0\ 0\ 72\ 100\ 100]$ (since the default value of {\bf UserUnit} is $1/72$th of
an inch).\cr
Function&function&A $2$-in, $n$-out function; or an array of $n$ $2$-in $1$-out functions (where $n$ is the
number of color values required by the {\bf ColorSpace} specified in the shading dictionary).
Each function's domain must be a superset of {\bf Domain}.
\edicttable

For example, suppose we want to shade using the following function, which simply computes the average of its
two inputs:

\blisting
1 0 obj
<<
    /FunctionType 4
    /Domain [0 1 0 1]
    /Range [0 1] 
    /Length 13        
>>
stream
{
    add
    2 div
}
endstream
endobj
\elisting

\noindent We create a shading object which uses this function:

\blisting
% 2 0 obj
<<
    /ShadingType 1              % functional shading
    /ColorSpace /DeviceGray     % use grayscale coloring
    /Domain [0 1 0 1]           % domain is [0,1] x [0,1]
    /Matrix [100 0 0 100 0 0]   % scale gradient by 100
    /Function 1 0 R             % use above function
>> 
\elisting

\noindent and a shading pattern object which sets this as its {\bf Shading} attribute

\blisting
% 4 0 obj
<<
    /Type /Pattern  % pattern object
    /PatternType 2  % shading pattern
    /Shading 2 0 R  % shading object is above dictionary
>> 
\elisting

\noindent The object in which the pattern is drawn is

\blisting
5 0 obj
<<
    /Type /XObject
    /Subtype /Form
    /BBox [0 0 99.626 99.626]
    /FormType 1
    /Matrix [1 0 0 1 0 0]
    /Resources 11 0 R
    /Length 42        
>>
stream
q
/Pattern cs     % set color space to pattern
/P scn          % set current pattern
0 0 100 100 re  % draw rectangle
B               % fill
Q 
endstream
endobj
\elisting

\noindent Notice that we don't need to change the color space, as the color space is determined by the
shading pattern ({\bf P} is defined in the resources of this object, object {\tt11 0} as a reference to object
{\tt4 0}).
The resulting pattern is:

\bgroup
\immediate\pdfobj stream attr{
    /FunctionType 4
    /Domain [0 1 0 1]
    /Range [0 1]
}{{
    add
    2 div
}}

\immediate\pdfobj{
<<
    /ShadingType 1
    /ColorSpace /DeviceGray
    /Domain [0 1 0 1]
    /Matrix [100 0 0 100 0 0]
    /Function \the\pdflastobj\space0 R
>>
}

\immediate\pdfobj{
<<
    /Type /Pattern
    /PatternType 2
    /Shading \the\pdflastobj\space 0 R
>>
}

\setbox0=\hbox to 100pt{\vrule width\z@ height100pt depth\z@%
    \pdfliteral{
        q
        /Pattern cs
        /P scn
        0 0 100 100 re
        B
        Q
    }\hfil%
}
\pdfxform resources{/Pattern << /P \the\pdflastobj\space 0 R >>}0

\centerline{\pdfrefxform\pdflastxform}
\egroup

\subsubsection{Type 2 (Axial) Shadings}

Axial shadings are shadings which vary along a linear axis between two endpoints, and extends indefinitely
perpendicular to that axis.
The shading may also extend indefinitely beyond either or both endpoints by continuing the boundary colors.
The following table lists the fields specific to type 2 shadings:

\bdicttable
Coords&array&(Required) an array of the form $[x_0\ y_0\ x_1\ y_1]$ which specifies the start and endpoints
of the axis, expressed in the shading's target coordinate space.\cr
Domain&array&(Optional) an array of two numbers $[t_0\ t_1]$ which specifies the limiting values of the
parameteric value $t$ (i.e. $t_0\leq t\leq t_1$).
Default value is $[0\ 1]$.\cr
Function&function&(Required) a $1$-in $n$-out function, or an array of $n$ $1$-in $1$-out functions (where
$n$ is the number of color values required by the {\bf Color Space}).
These functions are used to determine the color along the axis.
Each function's domain must be a superset of {\bf Domain}.\cr
Extend&array&(Optional) an array of two boolean values which specifies whether to extend the shading beyond
the start and end points of the axis, respectively.
Default value is $[{\tt false}\ {\tt false}]$.
\edicttable

Suppose we want to create a linear gradient between red and blue.
We can use the following linear interpolation function 

\blisting
% 1 0 obj
<<
    /FunctionType 2 % exponential interpolation function
    /Domain [0 1]   % domain is [0,1]
    /C0 [1 0 0]     % interpolate between red ...
    /C1 [0 0 .5]    % ... and blue
    /N 1            % using linear interpolation
>> 
\elisting

\noindent Now we define the shading object

\blisting
% 3 0 obj
<<
    /ShadingType 2          % axial shading
    /ColorSpace /DeviceRGB  % use RGB color sapce
    /Coords [0 0 100 0]     % axis between origin and (100,0)
    /Domain [0 1]           % domain of function is [0 1]
    /Function 1 0 R         % function is above linear interpolation function
    /Extend [false false]   % do not extend
>> 
\elisting

\noindent and the pattern object

\blisting
% 4 0 obj
<<
    /Type /Pattern  % pattern object
    /PatternType 2  % shading pattern
    /Shading 3 0 R  % using the above shading object
>> 
\elisting

\noindent And the actual content stream where the gradient is drawn is

\blisting
5 0 obj
<<
/Type /XObject
/Subtype /Form
/BBox [0 0 99.626 99.626]
/FormType 1
/Matrix [1 0 0 1 0 0]
/Resources 11 0 R
/Length 94        
>>
stream
q
/Pattern cs
/P scn
50 100 m
100 100 100 50 v 100 0 50 0 v 0 0 0 50 v 0 100 50 100 v % draw circle
h
B                                                       % fill
Q 
endstream
endobj
\elisting

\noindent Again, {\bf P} is defined in the {\bf Resources} of the content stream as a reference to the pattern
object defined.
The resulting pattern is

\immediate\pdfobj{
<<
    /FunctionType 2
    /Domain [0 1]
    /C0 [1 0 0]
    /C1 [0 0 .5]
    /N 1
>>
}

\immediate\pdfobj{
<<
    /ShadingType 2
    /ColorSpace /DeviceRGB
    /Coords [0 0 100 0]
    /Domain [0 1]
    /Function \the\pdflastobj\space0 R
    /Extend [false false]
>>
}

\immediate\pdfobj{
<<
    /Type /Pattern
    /PatternType 2
    /Shading \the\pdflastobj\space 0 R
>>
}

\bgroup
\setbox0=\hbox to 100pt{\vrule width\z@ height100pt depth\z@%
    \pdfliteral{
        q
        /Pattern cs
        /P scn
        50 100 m
        100 100 100 50 v
        100 0 50 0 v
        0 0 0 50 v
        0 100 50 100 v
        h
        B
        Q
    }\hfil
}
\pdfxform resources{/Pattern << /P \the\pdflastobj\space 0 R >>}0

\centerline{\pdfrefxform\pdflastxform}
\egroup

\subsubsection{Type 3 (Radial) Shadings}

Radial shadings define a gradient that varies between two circles.
This can be used to depict three-dimensional spheres and cones.
Radial shading dictionaries may contain the following fields:

\bdicttable
Coords&array&(Required) an array of the form $[x_0\ y_0\ r_0\ x_1\ y_1\ r_1]$, which specifies the centers
and radii of the starting and ending circles.
$r_0$ and $r_1$ must be at least zero.
If one is zero, it is treated as a point, and if both are, nothing is painted.\cr
Domain&array&(Optional) an array of two numbers $[t_0\ t_1]$ specifying the limiting values of the parametric
value $t$.
This variable varies linearly between $t_0$ and $t_1$ as it goes between the two circles.
$t$ is the input to the function specified by {\bf Function}.\cr
Function&function&(Required) a $1$-in $n$-out, or an array of $n$ $1$-in $1$-out functions, (where $n$ is the
number of values in the {\bf ColorSpace}) which specify the color to be drawn, called with parametric value
$t$.
Each function's domain must be a superset of {\bf Domain}.\cr
Extend&array&(Optional) an array of two boolean values which specifies whether or not to extend the shading
beyond the starting and ending circles.
Default value $[{\it false}\ {\it false}]$.
\edicttable

For example, if we want to shade a circle with an inner color of red and an outer color of blue, we first
define the interpolation function which linearly interpolates between red and blue (same as our previous
example).
Then we define a shading object:

\blisting
% 3 0 obj
<<
    /ShadingType 3              % radial shading
    /ColorSpace /DeviceRGB      % rgb color space
    /Coords [75 75 0 50 50 50]  % first circle is a point at (75,75),
                                % second is the circle of radius 50 at (50,50)
    /Domain [0 1]               % domain of linear interpolation function
    /Function 1 0 R             % use linear interpolation function
    /Extend [false false]       % do not extend
>> 
\elisting

\noindent The object which draws the pattern is

\blisting
4 0 obj
<<
    /Type /XObject
    /Subtype /Form
    /BBox [0 0 99.626 99.626]
    /FormType 1
    /Matrix [1 0 0 1 0 0]
    /Resources 10 0 R
    /Length 7
>>
stream
/Sh sh
endstream
endobj
\elisting

\noindent Notice here we just use the {\bf sh} operator, which we explain below.
Let's look at the resources dictionary:

\blisting
% 10 0 obj
<<
    /Shading << /Sh 3 0 R >>
    /ProcSet [ /PDF ]
>>
\elisting

\noindent So we define {\bf Sh} to be a {\bf Shading} object, referencing the shading object defined in object
{\tt3 0}.
The resulting pattern is

\immediate\pdfobj{
<<
    /FunctionType 2
    /Domain [0 1]
    /C0 [1 0 0]
    /C1 [0 0 .5]
    /N 1
>>
}

\immediate\pdfobj{
<<
    /ShadingType 3
    /ColorSpace /DeviceRGB
    /Coords [75 75 0 50 50 50]
    /Domain [0 1]
    /Function \the\pdflastobj\space0 R
    /Extend [false false]
>>
}

\bgroup
\setbox0=\hbox to 100pt{\vrule width\z@ height100pt depth\z@%
    \pdfliteral{
        /Sh sh
    }\hfil%
}
\pdfxform resources{/Shading << /Sh \the\pdflastobj\space 0 R >>}0

\centerline{\pdfrefxform\pdflastxform}
\egroup

\subsubsection{The shading operator}

If you have a shading pattern {\it name} and you'd like to draw it as-is (without using it to fill a path),
you can use the shading operator {\bf sh}.
{\it name} must be specified in the {\bf Shading} subdictionary of the resource dictionary of the current
content stream.
This operator should only be applied to bounded or geometrically defined shadings.

\bnote
    The operand to the shading operator {\bf sh} must be a {\it shading object}; not a {\it pattern object}.
\eppbox

