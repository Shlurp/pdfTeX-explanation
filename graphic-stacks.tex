\subsection{Color stacks}

Consider the following code to color some text:

\blisting
\quitvmode\pdfliteral{q 1 0 0 rg 1 0 0 RG}Hello\pdfliteral{Q}Hey!
\elisting

(\macro\quitvmode{} is a pdf\TeX{} primitive which has essentially the same effect as \macro\leavevmode.)
This will result in the following PDF code:

\blisting \setsyntax{pdf}
1 0 0 1 91.925 759.927 cm
q 1 0 0 rg 1 0 0 RG
1 0 0 1 -91.925 -759.927 cm
BT
/F1 9.9626 Tf 91.925 759.927 Td [(Hello)]TJ
ET
1 0 0 1 114.341 759.927 cm
Q
1 0 0 1 -114.341 -759.927 cm
BT
/F1 9.9626 Tf 114.341 759.927 Td [(Hey!)]TJ
ET
\elisting

Let's see what happens here:
\benum
    \item the CTM is translated so that it points at the current position on the page;
    \item the graphics state is pushed onto the stack, and the color is changed to red;
    \item the CTM is translated back (since after the \macro\pdfliteral{} is executed, the CTM must be
    repositioned);
    \item ``Hello'' is printed;
    \item the CTM is translated to the current position for the next \macro\pdfliteral;
    \item the \macro\pdfliteral{} injects {\tt Q} to the content stream;
    \item the CTM is translated back;
    \item ``Hey!'' is printed.
\eenum

Note the issue: {\tt Q} is injected into the content stream, which pops the last graphics state.
This graphics state was had the CTM positioned at $[{\tt1\ 0\ 0\ 1\ 91.925\ 759.927}]$, the placement of
``Hello''.
So the ``Hey!'' will be printed {\it over} ``Hello'', instead of after it!

So we need some way of letting \TeX{} handle the graphics state stack.
This can be done via two new pdf\TeX{} primtives: \macro\pdfcolorstackinit{} and \macro\pdfcolorstack.

\macro\pdfcolorstackinit{} initializes a \TeX{} graphics state stack (also called a color stack), which can
then be manipulated by \macro\pdfcolorstack.
The usage of \macro\pdfcolorstackinit{} is as follows:
\getmacrousage{\pdfcolorstackinit [page] [direct] <general text>}
\noindent this initializes a \TeX{} graphics state stack, whose initial value is determined by {\it general
text}.
A usage of \macro\pdfcolorstackinit{} expands to the number of the \TeX{} graphics state stack, which is used
by \macro\pdfcolorstack{} to manipulate the stack.
If {\tt page} is given, then the \TeX{} graphics state stack is restored at the beginning of each page.
If {\tt direct} is given, operations by \macro\pdfcolorstack{} will not break the current text object in which
they may reside.

The usage of \macro\pdfcolorstack{} is as follows:
\getmacrousage{\pdfcolorstack <stack number> (set | push | pop | current) <general text>}
\noindent This manipulates the stack given by {\it stack number} according the operation specified:
\blist
    \item {\tt set}: the current value of the \TeX{} graphics stack is replaced with {\it general text};
    \item {\tt push}: {\it general text} is pushed to the top of the \TeX{} graphics stack, and becomes the new
    current value;
    \item {\tt pop}: the top value of the \TeX{} graphics state is removed and the new top becomes the
    current graphics state;
    \item {\tt current}: the PDF graphics state is set to the top value of the \TeX{} stack without modifying
    the \TeX{} graphics state stack.
\elist

Importantly, \TeX{} implements its color stacks without using the {\tt q},{\tt Q} operators.
When \macro\pdfcolorstack{} is used, it simply puts the top of the stack into the PDF content stream.
For example:

\blisting
\chardef\tgs=\pdfcolorstackinit direct {0 g 0 G}

\pdfcolorstack\tgs push{1 0 0 rg 1 0 0 RG}Hi
\pdfcolorstack\tgs pop{}there!
\elisting

\noindent Will result in the PDF code

\blisting \setsyntax{pdf}
1 0 0 rg 1 0 0 RG
BT
/F1 9.9626 Tf 91.925 759.927 Td [(Hi)]TJ
0 g 0 G
 [-333(there!)]TJ
ET
\elisting

\noindent Notice that instead of wrapping the first color change and text object in {\tt q...Q}, it simply
places {\tt0 g 0 G} (the head of the stack after the {\tt pop}) after it.
A \TeX{} graphics state can be used for any graphics state parameter, but it should be used for the same set
of parameters for each {\tt push} operation.
That is, if you start with the stack initialized to {\tt0 g 1 Tr}, don't all of the sudden push {\tt0 G}.
The reason being that when you pop, you won't reset the stroking color (because {\tt0 g 1 Tr} doesn't change
the stroke color).
Similarly, don't {\it just} push {\tt1 g}, since if you then push something which uses {\tt Tr}, popping won't
reset it.

\subsection{The issue with transformations}

Another issue is that of transformations.
Suppose we do

\blisting
\chardef\tgs=\pdfcolorstackinit{1 0 0 1 0 0 cm}     % cant change CTM in a text object;
                                                    % cannot be direct

Hi
\pdfcolorstack\tgs push{2 0 0 2 0 0 cm}there,
\pdfcolorstack\tgs pop{}pal!
\elisting

\noindent The resulting PDF code shouldn't be surprising:

\blisting \setsyntax{pdf}
BT
/F1 9.9626 Tf 91.925 759.927 Td [(Hi)]TJ
ET
1 0 0 1 105.486 759.927 cm
2 0 0 2 0 0 cm              % here
1 0 0 1 -105.486 -759.927 cm
BT
/F1 9.9626 Tf 105.486 759.927 Td [(there,)]TJ
ET
1 0 0 1 133.741 759.927 cm
1 0 0 1 0 0 cm              % here
1 0 0 1 -133.741 -759.927 cm
BT
/F1 9.9626 Tf 133.741 759.927 Td [(pal!)]
ET
\elisting

\noindent Notice that when we {\tt pop} from the \TeX{} graphics state stack, it just writes
{\tt1 0 0 1 0 0 cm} to the PDF file.
Which effectively does nothing.
So we're left with:

\bigskip
\hbox{%
\pdfliteral{q}%
\chardef\tgs=\pdfcolorstackinit{1 0 0 1 0 0 cm}
\rlap{Hi
\pdfcolorstack\tgs push{2 0 0 2 0 0 cm}there,
\pdfcolorstack\tgs pop{}pal!}%
\pdfliteral{Q}}

Now suppose we use \macro\pdfliteral s instead:

\blisting
Hi
\pdfliteral{q 2 0 0 2 0 0 cm}there,
\pdfliteral{Q}pal!
\elisting

\noindent Then we get the following result:

\bigskip
\hbox{%
Hi
\pdfliteral{q 2 0 0 2 0 0 cm}\rlap{there},
\pdfliteral{Q}pal!
}

\noindent Which is produced by the following PDF code:

\blisting \setsyntax{pdf}
BT
/F1 9.9626 Tf 91.925 759.927 Td [(Hi)]TJ
ET
1 0 0 1 105.486 759.927 cm      % position for "there,"
q 2 0 0 2 0 0 cm        % insertion here
1 0 0 1 -105.486 -759.927 cm    % move back
BT
/F1 9.9626 Tf 105.486 759.927 Td [(there,)]TJ
ET
1 0 0 1 133.741 759.927 cm      % position for "pal!"
Q                       % insertion here
1 0 0 1 -133.741 -759.927 cm    % move back
BT
/F1 9.9626 Tf 133.741 759.927 Td [(pal!)]TJ
ET
\elisting

\noindent Once again we can see that the issue is we reset the state to the position where ``there,'' is to
be printed.

While pdf\TeX{} does provide the primitives \macro\pdfsetmatrix, \macro\pdfsave, and \macro\pdfrestore, these
do little but make \TeX{} aware of the changes to the CTM so it can properly place link and anchor positions.
They don't do much else (\macro\pdfsave{} injects {\tt q} into the content stream, \macro\pdfrestore{} injects
{\tt Q}).

\subsubsubsection{An important note regarding transformations}

When changing the CTM, make sure that the {\tt q} and {\tt Q} are injected into the PDF at the same 
{\it place} in the document spatially.
Otherwise, as you can see above this will mess up the CTM since the translation back is after the {\tt Q}.
So the result is that the CTM is translated back (in the above example) $-133.741$ {\tt bp}s.
Thus when you wrap a CTM change in {\tt q...Q}, also wrap it in \macro\rlap, like so:
\blisting
Hi,
\pdfliteral{q 2 0 0 2 0 0 cm}\rlap{there,}%
\pdfliteral{Q}pal!
\elisting
So while we still overlap with ``there'', we don't mess subsequent text (which will happen if the CTM
is messed up).
The resulting PDF code is:
\blisting \setsyntax{pdf}
BT
/F1 9.9626 Tf 91.925 759.927 Td [(Hi)]TJ
ET
1 0 0 1 105.486 759.927 cm      % this matches with
q 2 0 0 2 0 0 cm
1 0 0 1 -105.486 -759.927 cm
BT
/F1 9.9626 Tf 105.486 759.927 Td [(there,)]TJ
ET
1 0 0 1 105.486 759.927 cm
Q
1 0 0 1 -105.486 -759.927 cm    % this
BT
/F1 9.9626 Tf 105.486 759.927 Td [(pal!)]TJ
ET
\elisting

