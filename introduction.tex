The Portable Document Format (PDF) is a file format developed by the Adobe corporation in $1992$ to render and
display documents.
It is a rich file format capable of displaying a diverse variety of documents, and its immense popularity has
lead to its use around the world.
Still today it is the leading format for displaying documents in a cross-platform manner.
It has undergone various updates and standardizations, keeping it modern and usable.

\TeX{} is a program and language for typesetting (generally academic) documents.
Historically, it compiled to a {\tt dvi} (device independent) file format, but a more modern \TeX{} engine
called pdf\TeX{} was developed to compile to PDF.
\TeX{} includes a powerful macro-based programming layer, as well as a versatile typesetting engine.
We assume basic knowledge of plain-\TeX{} for this article (if you don't know what plain-\TeX{} is, try reading
the \TeX book by D. Knuth before this article).

Despite the multitude of literature on PDF as well as \TeX, there exists little literature on pdf\TeX.
Specifically, there does not exist much literature on how to utilize pdf\TeX-primitives to create PDF graphics.
The pdf\TeX{} manual lists primitives, but does not give explanations on how to use them, and instead assumes
intimate familiarity with PDF.
In this article we will both explain PDF as well as how to utilize pdf\TeX{} primitives to create PDFs.

We give thanks to the resources which were invaluable for this article:
\blist
    \item The \url{https://mirrors.rit.edu/CTAN/systems/doc/pdftex/manual/pdftex-a.pdf}{pdf\TeX{} manual} by
    H\`an Th\^e\llap{\raise 0.5ex\hbox{\'{}}} Th\`anh and team;
    \item The \url{https://opensource.adobe.com/dc-acrobat-sdk-docs/pdfstandards/pdfreference1.7old.pdf}
    {PDF reference} by Adobe;
    \item \url{https://petr.olsak.net/ftp/olsak/bulletin/pdfgrafika.pdf}
    {Petr Ol\v s\'ak's article on pdf\TeX{} primitives};
    \item The \url{https://ctan.mc1.root.project-creative.net/graphics/pgf/base/doc/pgfmanual.pdf}
    {TikZ and PGF manual}, as well as PGF source code.
\elist

\subsection*{Structure of the article}

This article will be split largely into two parts: an introduction to PDF, and an explanation of pdf\TeX{}
primitives.
The first part will not cover the entirety of the PDF, and we will leave certain subjects for the second part
as well.

